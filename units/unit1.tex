\documentclass[../notes.tex]{subfiles}

\begin{document}
	\chapter{Number Systems}
		\section{Number Properties}
			\subsection{Commutativity}
				For all integers $m$ and $n$, \textit{addition} and \textit{multiplication} are \textbf{commutative}.
				\begin{align*}
					m + n &= n + m \tag*{addition}\\
					mn &= nm \tag*{multiplication}
				\end{align*}
			\subsection{Associativity}
				For all integers $m$, $n$ and $k$, \textit{addition} and \textit{multiplication} are \textbf{associaive}.
				\begin{align*}
					m+(n+k) &= (m+n)+k \tag*{addition}\\
					(m)(nk) &= (mn)k \tag*{multiplication}
				\end{align*}
			\subsection{Distributivity}
				For all integers $m$, $n$ and $k$, \textit{multiplication} is \textbf{distributive} over \textit{addition}.
				\begin{align*}
					m(n + k) &= mn + mk\\
					(n + k)m &= m(n + k)\\
					&= mn + mk\\
					&= nm + km
				\end{align*}
			\subsection{Multiplicative Identity}
				There exists an integer ($1$) that has the property that for every integer $m$, $m\cdot 1 = m$.
			\subsection{Additive Identity}
				There exists an integer ($0$) that has the property that for every integer $m$, $m + 0 = m$.
			\subsection{Linearity}
				For all integers $m$ and $n$, exactly one of the following is true:
				\begin{align*}
					m &< n\\
					m &= n\\
					m &> n
				\end{align*}
			\subsection{Monotocity}
				For all integers $m$, $n$ and $k$,\\
				\-\hspace{2em}If $m = n$, then $m + k = n + k$ and $mk = nk$.\\
				\-\hspace{2em}If $m < n$, then $m + k < n + k$.\\
				\-\hspace{4em}If $k > 0$, then $mk < nk$.\\
				\-\hspace{4em}If $k < 0$, then $mk < nk$.
			\subsection{Transitivity of \texorpdfstring{$=$}{=}, \texorpdfstring{$<$}{<} and \texorpdfstring{$>$}{>}}
				For all integers $m$, $n$ and $k$,\\
				\-\hspace{2em}If $m = n$ and $n = k$, then $m = k$.\\
				\-\hspace{2em}If $m < n$ and $n < k$, then $m < k$.\\
				\-\hspace{2em}If $m > n$ and $n > k$, then $m > k$.
			\subsection{Absence of Zero Divisors}
				For all integers $m$ and $n$,
				\-\hspace{2em}$mn = 0$ if and only if $m = 0$ or $n = 0$.
			\subsection{Additive Inverses}
				For every integer $m$ there exists an integer $n$ such that
				\begin{align*}
					m + n = 0
				\end{align*}
		\noindent\rule{\textwidth}{0.4pt}
\end{document}