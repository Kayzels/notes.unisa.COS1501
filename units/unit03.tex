\documentclass[../notes.tex]{subfiles}

\begin{document}
	\ifSubfilesClassLoaded{\setcounter{chapter}{2}}{}
	\chapter{Sets}
		\section{Subset}
			\begin{definition}{Subset}
				If $A$ and $B$ are sets from a universal set $U$, then $A$ is a \concept{subset} of $B$ if and only if every element of $A$ is also an element of $B$.\\
				Can be abbreviated $A \subseteq B$
			\end{definition}
			\subsection{Proper Subset}
				\begin{definition}{Proper Subset}
					If $C$ and $D$ are sets from a universal set $U$, and every element of $C$ is an element of $D$, but $D$ has some elements that are not in $C$, then $C$ is a \concept{proper subset} of $D$.\\
					Can be abbreviated $C \subset D$.
				\end{definition}
				\begin{sidenote}{Confusion Between Element and Subset}
					Note that $\in$ and $\subset$ are not the same. This becomes significant when dealing with power sets.
				\end{sidenote}
		\pagebreak
		\section{Creating Sets From Other Sets}
			For examples, the following sets will be used:
			\begin{alignat*}{4}
				U &= \{1, 2, 3, 4, 5\}\qquad & A &= \{1, 2, 3\} \qquad & B &= \{2, 3, 4\} \qquad & C &= \{4, 5\}
			\end{alignat*}

			\subsection[Set Union]{Set Union (OR)}
				\begin{definition}{Set Union}
					The \concept{union} of sets $A$ and $B$ is written $A \cup B$, and is the set of all elements that belong to $A$ or $B$ (or both).
				\end{definition}
				\nopagebreak
				\begin{center}
					\begin{venntwo}[][$A \cup B = \bigl\{x \mid x \in A$ or $x \in B\bigr\}$]
						\fillA
						\fillB
					\end{venntwo}
				\end{center}
				\begin{example}[hbox] $
					\begin{aligned}[t]
						A \cup B &= \{1, 2, 3\} \cup \{2, 3, 4\}\\
						&= \{1, 2, 3, 4\}
					\end{aligned} $
				\end{example}
			\subsection[Set Intersection]{Set Intersection (AND)}
				\begin{definition}{Set Intersection}
					The \concept{intersection} of sets $A$ and $B$ is written $A \cap  B$, and is the set of all elements that belong to $A$ and $B$ at the same time.
				\end{definition}
				\nopagebreak
				\begin{center}
					\begin{venntwo}[][$A \cap B = \bigl\{x \mid x \in A$ and $x \in B\bigr\}$]
						\fillACapB
					\end{venntwo}
				\end{center}
				\begin{example}[hbox] $
					\begin{aligned}[t]
						A \cap B &= \{1, 2, 3\} \cap \{2, 3, 4\}\\
						&= \{2, 3\}
					\end{aligned} $
				\end{example}

			\subsection[Set Difference]{Set Difference (MINUS)}
				\begin{definition}{Set Difference}
					The \concept{difference} between sets $A$ and $B$, also called the \concept{complement of $B$ relative to $A$}, is written $A - B$, and is the set of elements that are in $A$ that are not in $B$.
				\end{definition}
				\nopagebreak
				\begin{center}
					\begin{venntwo}[][$A - B = \bigl\{x \mid x \in A$ and $x \notin B\bigr\}$]
						\fillOnlyA
					\end{venntwo}
				\end{center}
				\begin{example}[hbox] $
					\begin{aligned}[t]
						A - B &= \{1, 2, 3\} - \{2, 3, 4\}\\
						&= \{1\}
					\end{aligned} $
				\end{example}

			\subsection[Set Complement]{Set Complement (NOT)}
				\begin{definition}{Set Complement}
					Let $A$ be a subset of a universal set $U$. Then the \concept{complement} of $A$, written $A'$ is the set of all elements that belong to $U$ but do not belong to $A$.
				\end{definition}
				\nopagebreak
				\begin{center}
					\begin{venntwo}[][$A' = \bigl\{x \mid x \notin A\bigr\}$]
						\fillNotA
					\end{venntwo}
				\end{center}
				\begin{example}[hbox] $
					\begin{aligned}[t]
						A' &= U - A\\
						&= \{1, 2, 3, 4, 5\} - \{1, 2, 3\}\\
						&= \{4, 5\}
					\end{aligned} $
				\end{example}

			\pagebreak
			\subsection[Symmetric Set Difference]{Symmetric Set Difference (XOR)}
				\begin{definition}{Symmetric Set Difference}
					The \concept{symmetric difference} between two sets $A$ and $B$, written $A + B$, is the set of elements that belong to $A$ or to $B$, but not to both.
				\end{definition}
				\nopagebreak
				\begin{center}
					\begin{venntwo}[][$A + B = \bigl\{x \mid x \in A$ or $x \in B$, but not both$\bigr\}$]
						\fillOnlyA
						\fillOnlyB
					\end{venntwo}
				\end{center}
				
				\begin{example}[hbox] $
					\begin{aligned}[t]
						A + B &= \{1, 2, 3\} + \{2, 3, 4\}\\
						&= \{1, 4\}
					\end{aligned} $
				\end{example}
		\section{Other Terms Significant For Sets}
			\subsection{The Empty Set}
				\begin{definition}{Empty Set}
					The set that contains no elements is called the \concept{empty set}, and is written $\emptyset$.
				\end{definition}
			\subsection{Set Disjointness}
				\begin{definition}{Disjointness}
					Two sets $A$ and $B$ are called \concept{disjoint} if they have no elements in common. In other words,
					\begin{align*}
						A \cap B = \emptyset
					\end{align*}
				\end{definition}
				\begin{example}[hbox] $
					\begin{aligned}[t]
						A \cap C &= \{1, 2, 3\} \cap \{4, 5\}\\
						&= \emptyset
					\end{aligned} $
				\end{example}
			\pagebreak
			\subsection{Set Cardinality}
				\begin{definition}{Cardinality}
					Let $A$ be a set with $k$ distinct elements that can be counted. The \emph{number of elements} $k$ in $A$ is called the \concept{cardinality} of the set. It can be written as $n(A)$ or $\left\lvert A\right\rvert$.
				\end{definition}
				\begin{example}[hbox] $
					\begin{aligned}[t]
						\left\lvert A\right\rvert &= \bigl\lvert \{1, 2, 3\}\bigr\rvert\\
						&= 3
					\end{aligned} $
				\end{example}
			\subsection{Power Sets}
				\begin{definition}{Power Set}
					Given a set $A$ with $n$ distinct elements, the \concept{power set} of $A$, written $\mathcal{P}(A)$, is the set that has as its members \emph{all} subsets of $A$.
				\end{definition}
				\nopagebreak
				\begin{sidenote}{Every Element of a Power Set is a Set}
					It is important to note that every element of a power set is a \emph{set}!\\ That means if $B$ is a subset of $A$, then $B$ is an element of $\mathcal{P}(A)$, i.e. $B \in \mathcal{P}(A)$.\\
					However, $B$ is \emph{not} a subset of $\mathcal{P}(A)$, i.e. $B \nsubseteq \mathcal{P}(A)$! A set containing $B$, i.e. $\{B\}$ would be a subset of $\mathcal{P}(A)$.
				\end{sidenote}
				\nopagebreak
				\begin{example}[hbox] $
					\begin{aligned}[t]
						\mathcal{P}(C) &= \mathcal{P}\bigl(\{4, 5\}\bigr)\\
						&= \bigl\{\emptyset, \{4\}, \{5\}, \{4, 5\}\bigr\}
					\end{aligned} $
				\end{example}
				The cardinality of a power set $\mathcal{P}(A)$ is $2^{n}$ where $n$ is the number of elements in the set $A$.
				\begin{example}[hbox] $
					\begin{aligned}[t]
						\bigl\lvert \mathcal{P}(A)\bigr\rvert &= \Bigl\lvert\mathcal{P}\bigl(\{1, 2, 3\}\bigr)\Bigr\rvert\\
						&= 2^{3}\\
						&= 8
					\end{aligned} $
				\end{example}
			\pagebreak
			\begin{exercise}{Self Assessment Exercise \thechapter.6}
				\begin{enumerate}
					\item \hfill $
						\begin{alignedat}[t]{3}
							U &= \{1, 2, 3, 4, 5\} \qquad & A &= \{1, 2, 3\} \qquad & B &= \{3, 4, 5\}
						\end{alignedat} $ \hfill \phantom{1 \quad}
						\begin{multicols}{2}
							\begin{enumerate}[label=(\alph*), labelsep=1em]
								\item $
									\begin{aligned}[t]
										A \cup B &= \{1, 2, 3\} \cup \{3, 4, 5\}\\
										&= \{1, 2, 3, 4, 5\}\\
										B \cup A &= \{3, 4, 5\} \cup \{1, 2, 3\}\\
										&= \{1, 2, 3, 4, 5\}
									\end{aligned} $
								\item $
									\begin{aligned}[t]
										A \cap B &= \{1, 2, 3\} \cap \{3, 4, 5\}\\
										&= \{3\}\\
										B \cap A &= \{3, 4, 5\} \cap \{1, 2, 3\}\\
										&= \{3\}
									\end{aligned} $
								\item $
									\begin{aligned}[t]
										A - B &= \{1, 2, 3\} - \{3, 4, 5\}\\
										&= \{1, 2\}\\
										B - A &= \{3, 4, 5\} - \{1, 2, 3\}\\
										&= \{4, 5\}
									\end{aligned} $
								\item $
									\begin{aligned}[t]
										A + B &= \{1, 2, 3\} + \{3, 4, 5\}\\
										&= \{1, 2, 4 , 5\}\\
										B + A &= \{3, 4, 5\} + \{1, 2, 3\}\\
										&= \{1, 2, 4, 5\}
									\end{aligned} $
							\end{enumerate}
						\end{multicols}
					\item \hfill $
						\begin{alignedat}[t]{3}
							U &= \{a, e, i, o, u\} \qquad & A &= \{i, o, u\} \qquad & B &= \{a, e, o, u\}
						\end{alignedat} $ \hfill \phantom{2 \quad}
						\begin{multicols}{2}
							\begin{enumerate}[label=(\alph*), labelsep=1em]
								\item $
									\begin{aligned}[t]
										A' &= \{i, o, u\}'\\
										&= \{a, e, i, o, u\} - \{i, o, u\}\\
										&= \{a, e\}\\
										\left(A'\right)' &= \{a, e, i, o, u\} - \{a, e\}\\
										&= \{i, o, u\}\\
										&= A
									\end{aligned} $
								\item $
									\begin{aligned}[t]
										B' &= \{a, e, o, u\}'\\
										&= \{a, e, i, o, u\} - \{a, e, o, u\}\\
										&= \{i\}\\
										\left(B'\right)' &= \{a, e, i, o, u\} - \{i\}\\
										&= \{a, e, o, u\}\\
										&= B
									\end{aligned} $
								\item $
									\begin{aligned}[t]
										A \cup B &= \{i, o, u\} \cup \{a, e, o, u\}\\
										&= \{a, e, i, o, u\}\\
										(A \cup B)' &= \{a, e, i, o, u\} - \{a, e, i, o, u\}\\
										&= \emptyset
									\end{aligned} $
								\item $
									\begin{aligned}[t]
										A' \cap B' &= \{a, e\} \cap \{i\}\\
										&= \emptyset
									\end{aligned} $
								\item $
									\begin{aligned}[t]
										A \cap B &= \{i, o, u\} \cap \{a, e, o, u\}\\
										&= \{o, u\}\\
										(A \cap B)' &= \{a, e, i, o, u\} - \{o, u\}\\
										&= \{a, e, i\}
									\end{aligned} $
								\item $
									\begin{aligned}[t]
										A' \cup B' &= \{a, e\} \cup \{i\}\\
										&= \{a, e, i\}
									\end{aligned} $
								\item $
									\begin{aligned}[t]
										A - B &= \{i, o, u\} - \{a, e, o, u\}\\
										&= \{i\}\\
										B - A &= \{a, e, o, u\} - \{i, o, u\}\\
										&= \{a,e\}
									\end{aligned} $
								\item $
									\begin{aligned}[t]
										A \cap B' &= \{i, o, u\} \cap \{i\}\\
										&= \{i\}\\
										B \cap A' &= \{a, e, o, u\} \cap \{a, e\}\\
										&= \{a, e\}
									\end{aligned} $
								\item $
									\begin{aligned}[t]
										A + B &= \{i, o, u\} + \{a, e, o, u\}\\
										&= \{a, e, i\}\\
										B + A &= \{a, e, o, u\} + \{i, o, u\}\\
										&= \{a, e, i\}
									\end{aligned} $
							\end{enumerate}
						\end{multicols}
					\item \hfill $
						\begin{alignedat}[t]{3}
							U &= \{1, 2, 3, 4, 5\} \qquad & A &= \{3\} \qquad & B &= \bigl\{\{3\}, 4, 5\bigr\}
						\end{alignedat} $ \hfill \phantom{3 \quad}
						\begin{align*}
							\mathcal{P}(A) &= \mathcal{P}\bigl(\{3\}\bigr)\\
							&= \bigl\{\emptyset, \{3\}\bigr\}\\
							\mathcal{P}(B) &= \mathcal{P}\Bigl(\bigl\{\{3\}, 4, 5\bigr\}\Bigr)\\
							&= \Bigl\{\emptyset, \bigl\{\{3\}\bigr\}, \bigl\{4\bigr\}, \bigl\{5\bigr\}, \bigl\{\{3\}, 4\bigr\}, \bigl\{\{3\}, 5\bigr\}, \bigl\{4, 5\bigr\}, \bigl\{\{3\}, 4, 5\bigr\}\Bigr\}
						\end{align*}
					\pagebreak
					\item \hfill $
						\begin{alignedat}[t]{3}
							U &= \{a, e, i, o, u\} \qquad & A &= \{i, o, u\} \qquad & B &= \{a, e, o, u\}
						\end{alignedat} $ \hfill \phantom{4 \quad}
						\begin{enumerate}[label=(\alph*), labelsep=1em]
							\item $
								\begin{aligned}[t]
									\mathcal{P}(A) &= \bigl\{\emptyset, \{i\}, \{o\}, \{u\}, \{i, o\}, \{i, u\}, \{o, u\}, \{i, o, u\}\bigr\}\\
									\mathcal{P}(B) &= \bigl\{\emptyset, \{a\}, \{e\}, \{o\}, \{u\}, \{a, e\}, \{a, o\}, \{a, u\}, \{e, o\}, \{e, u\}, \{o, u\}, \{a, e, o\},\\
									& \qquad \{a, e, u\}, \{a, o, u\}, \{e, o, u\}, \{a, e, o, u\}\bigr\}
								\end{aligned} $
							\item $
								\begin{aligned}[t]
									\mathcal{P}(A \cap B) &= \mathcal{P}\bigl(\{o, u\}\bigr)\\
									&= \bigl\{\emptyset, \{o\}, \{u\}, \{o, u\}\bigr\}\\
									\mathcal{P}(A) \cap \mathcal{P}(B) &= \bigl\{\emptyset, \{o\}, \{u\}, \{o, u\}\bigr\}
								\end{aligned} $
							\item $
								\begin{aligned}[t]
									\mathcal{P}\left(A'\right) &= \mathcal{P}\bigl(\{a, e\}\bigr)\\
									&= \bigl\{\emptyset, \{a\}, \{e\}, \{a, e\}\bigr\}
								\end{aligned}$

								$\begin{aligned}[t]
									\mathcal{P}(U) &= \bigl\{\emptyset, \{a\}, \{e\}, \{i\}, \{o\}, \{u\}, \{a, e\}, \{a, i\}, \{a, o\}, \{a, u\},\\
									& \qquad \{e, i\}, \{e, o\}, \{e, u\}, \{i, o\}, \{i, u\}, \{o, u\},\\
									& \qquad \{a, e, i\}, \{a, e, o\}, \{a, e, u\}, \{a, i, o\}, \{a, i, u\}, \{a, o, u\},\\
									& \qquad \{e, i, o\}, \{e, i, u\} \{e, o, u\}, \{i, o, u\},\\
									& \qquad \{a, e, i, o\}, \{a, e, i, u\}, \{a, e, o, u\} \{a, i, o, u\}, \{e, i, o, u\}, \{a, e, i, o, u\}\bigr\}
								\end{aligned} $

								$\begin{aligned}[t]
									\bigl(\mathcal{P}(A)\bigr)' &= \bigl\{\{a\}, \{e\}, \{a, e\}, \{a, i\}, \{a, o\}, \{a, u\}, \{e, i\}, \{e, o\}, \{e, u\},\\
									& \qquad \{a, e, i\}, \{a, e, o\}, \{a, e, u\}, \{a, i, o\}, \{a, i, u\}, \{a, o, u\}, \\
									& \qquad \{e, i, o\}, \{e, i, u\},  \{e, o, u\}, \\
									& \qquad \{a, e, i, o\}, \{a, e, i, u\}, \{a, e, o, u\} \{a, i, o, u\}, \{e, i, o, u\}, \{a, e, i, o, u\}\bigr\}
								\end{aligned} $
							\item $
								\begin{aligned}[t]
									\mathcal{P}(A) \cup \mathcal{P}(B) &= \bigl\{\emptyset, \{a\}, \{e\}, \{i\}, \{o\}, \{u\},\\
									& \qquad \{a, e\}, \{a, o\}, \{a, u\}, \{e, o\}, \{e, u\}, \{i, o\}, \{i, u\}, \{o, u\}, \\
									& \qquad \{a, e, o\}, \{a, e, u\}, \{a, o, u\}, \{e, o, u\}, \{i, o, u\}, \{a, e, o, u\}\bigr\}
								\end{aligned}$

								$\begin{aligned}[t]
									\mathcal{P}(A \cup B) &= \mathcal{P}\bigl(\{a, e, i, o, u\}\bigr)\\
									&= \mathcal{P}(U)\\
									&= \bigl\{\emptyset, \{a\}, \{e\}, \{i\}, \{o\}, \{u\}, \{a, e\}, \{a, i\}, \{a, o\}, \{a, u\},\\
									& \qquad \{e, i\}, \{e, o\}, \{e, u\}, \{i, o\}, \{i, u\}, \{o, u\},\\
									& \qquad \{a, e, i\}, \{a, e, o\}, \{a, e, u\}, \{a, i, o\}, \{a, i, u\}, \{a, o, u\},\\
									& \qquad \{e, i, o\}, \{e, i, u\} \{e, o, u\}, \{i, o, u\},\\
									& \qquad \{a, e, i, o\}, \{a, e, i, u\}, \{a, e, o, u\} \{a, i, o, u\}, \{e, i, o, u\}, \{a, e, i, o, u\}\bigr\}
								\end{aligned} $
						\end{enumerate}
				\end{enumerate}
			\end{exercise}
	\rulechapterend
\end{document}
