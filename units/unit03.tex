\documentclass[../notes.tex]{subfiles}

\begin{document}
  \ifSubfilesClassLoaded{\setcounter{chapter}{2}}{}
  \chapter{Sets}
    \section{Subset}
      If $A$ and $B$ are sets from a universal set $U$, then $A$ is a \textbf{subset} of $B$ if and only if every element of $A$ is also an element of $B$.\\
      Can be abbreviated $A \subseteq B$
      \subsection{Proper Subset}
        If $C$ and $D$ are sets from a universal set $U$, and every element of $C$ is an element of $D$, but $D$ has some elements that are not in $C$, then $C$ is a \textbf{proper subset} of $D$.\\
        Can be abbreviated $C \subset D$.
        \begin{sidenote}{Confusion Between Element and Subset}
          Note that $\in$ and $\subset$ are not the same. This becomes significant when dealing with power sets.
        \end{sidenote}
    \pagebreak
    \section{Creating Sets From Other Sets}
      For examples, the following sets will be used:
      \begin{alignat*}{4}
        U &= \{1, 2, 3, 4, 5\}\qquad & A &= \{1, 2, 3\} \qquad & B &= \{2, 3, 4\} \qquad & C &= \{4, 5\}
      \end{alignat*}

      \subsection[Set Union]{Set Union (OR)}
        The \textbf{union} of sets $A$ and $B$ is written $A \cup B$, and is the set of all elements that belong to $A$ or $B$ (or both).
        \nopagebreak
        \begin{align*}
          A \cup B = \bigl\{x \mid x \in A \text{ or } x \in B\bigr\}
        \end{align*}
        \nopagebreak
        \begin{center}
          \begin{venntwo}
            \fillA
            \fillB
          \end{venntwo}
        \end{center}
        \nopagebreak
        \begin{example} \moveup
          \begin{align*}
            A \cup B &= \{1, 2, 3\} \cup \{2, 3, 4\}\\
            &= \{1, 2, 3, 4\}
          \end{align*}
        \end{example}

      \subsection[Set Intersection]{Set Intersection (AND)}
        The \textbf{intersection} of sets $A$ and $B$ is written $A \cap  B$, and is the set of all elements that belong to $A$ and $B$ at the same time.
        \nopagebreak
        \begin{align*}
          A \cap B = \bigl\{x \mid x \in A \text{ and } x \in B\bigr\}
        \end{align*}
        \nopagebreak
        \begin{center}
          \begin{venntwo}
            \fillACapB
          \end{venntwo}
        \end{center}
        \nopagebreak
        \begin{example} \moveup
          \begin{align*}
            A \cap B &= \{1, 2, 3\} \cap \{2, 3, 4\}\\
            &= \{2, 3\}
          \end{align*}
        \end{example}

      \subsection[Set Difference]{Set Difference (MINUS)}
        The \textbf{difference} between sets $A$ and $B$, also called the \textbf{complement of $B$ relative to $A$}, is written $A - B$, and is the set of elements that are in $A$ that are not in $B$.
        \nopagebreak
        \begin{align*}
          A - B = \bigl\{x \mid x \in A \text{ and } x \notin B\bigr\}
        \end{align*}
        \nopagebreak
        \begin{center}
          \begin{venntwo}
            \fillOnlyA
          \end{venntwo}
        \end{center}
        \nopagebreak
        \begin{example} \moveup
          \begin{align*}
            A - B &= \{1, 2, 3\} - \{2, 3, 4\}\\
            &= \{1\}
          \end{align*}
        \end{example}

      \subsection[Set Complement]{Set Complement (NOT)}
        Let $A$ be a subset of a universal set $U$. Then the \textbf{complement} of $A$, written $A'$ is the set of all elements that belong to $U$ but do not belong to $A$.
        \nopagebreak
        \begin{align*}
          A' & = \bigl\{x \mid x \in U \text{ and } x \notin A\bigr\}\\
          & = \bigl\{x \mid x \notin A\bigr\} 
        \end{align*}
        \nopagebreak
        \begin{center}
          \begin{venntwo}
            \fillNotA
          \end{venntwo}
        \end{center}
        \nopagebreak
        \begin{example} \moveup
          \begin{align*}
            A' &= U - A\\
            &= \{1, 2, 3, 4, 5\} - \{1, 2, 3\}\\
            &= \{4, 5\}
          \end{align*}
        \end{example}

      \pagebreak
      \subsection[Symmetric Set Difference]{Symmetric Set Difference (XOR)}
        The \textbf{symmetric difference} between two sets $A$ and $B$, written $A + B$, is the set of elements that belong to $A$ or to $B$, but not to both.
        \nopagebreak
        \begin{align*}
          A + B = \bigl\{x \mid x \in A \text{ or } x \in B\text{, but not both}\bigr\}
        \end{align*}
        \nopagebreak
        \begin{center}
          \begin{venntwo}
            \fillOnlyA
            \fillOnlyB
          \end{venntwo}
        \end{center}
        \nopagebreak
        \begin{example} \moveup
          \begin{align*}
            A + B &= \{1, 2, 3\} + \{2, 3, 4\}\\
            &= \{1, 4\}
          \end{align*}
        \end{example}
    \section{Other Terms Significant For Sets}
      \subsection{The Empty Set}
        The set that contains no elements is called the \textbf{empty set}, and is written $\emptyset$.
      \subsection{Set Disjointness}
        Two sets $A$ and $B$ are called \textbf{disjoint} if they have no elements in common. In other words,
        \begin{align*}
          A \cap B = \emptyset
        \end{align*}
        \begin{example} \moveup
          \begin{align*}
            A \cap C &= \{1, 2, 3\} \cap \{4, 5\}\\
            &= \emptyset
          \end{align*}
        \end{example}
      \subsection{Set Cardinality}
        Let $A$ be a set with $k$ distinct elements that can be counted. The \textit{number of elements} $k$ in $A$ is called the \textbf{cardinality} of the set. It can be written as $n(A)$ or $\left\lvert A\right\rvert$.
        \nopagebreak
        \begin{example} \moveup
          \begin{align*}
            \left\lvert A\right\rvert &= \bigl\lvert \{1, 2, 3\}\bigr\rvert\\
            &= 3
          \end{align*}
        \end{example}
      \pagebreak
      \subsection{Power Sets}
        Given a set $A$ with $n$ distinct elements, the \textbf{power set} of $A$, written $\mathcal{P}(A)$, is the set that has as its members \textit{all} subsets of $A$.

        \begin{sidenote}{Every Element of a Power Set is a Set}
          It is important to note that every element of a power set is a \textit{set}!\\ That means if $B$ is a subset of $A$, then $B$ is an element of $\mathcal{P}(A)$, i.e. $B \in \mathcal{P}(A)$.\\
          However $B$ is \textbf{not} a subset of $\mathcal{P}(A)$, i.e. $B \nsubseteq \mathcal{P}(A)$! A set containing $B$, i.e. $\{B\}$ would be a subset of $\mathcal{P}(A)$.
        \end{sidenote}
      \begin{example} \moveup
        \begin{align*}
          \mathcal{P}(C) &= \mathcal{P}\bigl(\{4, 5\}\bigr)\\
          &= \bigl\{\emptyset, \{4\}, \{5\}, \{4, 5\}\bigr\}
        \end{align*}
      \end{example}
      The cardinality of a power set $\mathcal{P}(A)$ is $2^{n}$ where $n$ is the number of elements in the set $A$.
      \begin{example} \moveup
        \begin{align*}
          \bigl\lvert \mathcal{P}(A)\bigr\rvert &= \Bigl\lvert\mathcal{P}\bigl(\{1, 2, 3\}\bigr)\Bigr\rvert\\
          &= 2^{3}\\
          &= 8
        \end{align*}
      \end{example}
      \pagebreak
      \begin{exercise}{Self Assessment Exercise \thechapter.6}
        \begin{enumerate}
          \item \moveup
            \begin{alignat*}{3}
              U &= \{1, 2, 3, 4, 5\} \qquad & A &= \{1, 2, 3\} \qquad & B &= \{3, 4, 5\}
            \end{alignat*}
            \begin{enumerate}[label=(\alph*)]
              \item \moveup
                \begin{align*}
                  A \cup B &= \{1, 2, 3\} \cup \{3, 4, 5\}\\
                  &= \{1, 2, 3, 4, 5\}\\
                  B \cup A &= \{3, 4, 5\} \cup \{1, 2, 3\}\\
                  &= \{1, 2, 3, 4, 5\}
                \end{align*}
              \item \moveup
                \begin{align*}
                  A \cap B &= \{1, 2, 3\} \cap \{3, 4, 5\}\\
                  &= \{3\}\\
                  B \cap A &= \{3, 4, 5\} \cap \{1, 2, 3\}\\
                  &= \{3\}
                \end{align*}
              \item \moveup
                \begin{align*}
                  A - B &= \{1, 2, 3\} - \{3, 4, 5\}\\
                  &= \{1, 2\}\\
                  B - A &= \{3, 4, 5\} - \{1, 2, 3\}\\
                  &= \{4, 5\}
                \end{align*}
              \item \moveup
                \begin{align*}
                  A + B &= \{1, 2, 3\} + \{3, 4, 5\}\\
                  &= \{1, 2, 4 , 5\}\\
                  B + A &= \{3, 4, 5\} + \{1, 2, 3\}\\
                  &= \{1, 2, 4, 5\}
                \end{align*}
            \end{enumerate}
          \item \moveup
            \begin{alignat*}{3}
              U &= \{a, e, i, o, u\} \qquad & A &= \{i, o, u\} \qquad & B &= \{a, e, o, u\}
            \end{alignat*}
            \begin{enumerate}[label=(\alph*)]
              \item \moveup
                \begin{align*}
                  A' &= \{i, o, u\}'\\
                  &= \{a, e, i, o, u\} - \{i, o, u\}\\
                  &= \{a, e\}\\
                  \left(A'\right)' &= \{a, e, i, o, u\} - \{a, e\}\\
                  &= \{i, o, u\}\\
                  &= A
                \end{align*}
              \item \moveup
                \begin{align*}
                  B' &= \{a, e, o, u\}'\\
                  &= \{a, e, i, o, u\} - \{a, e, o, u\}\\
                  &= \{i\}\\
                  \left(B'\right)' &= \{a, e, i, o, u\} - \{i\}\\
                  &= \{a, e, o, u\}\\
                  &= B
                \end{align*}
              \item \moveup
                \begin{align*}
                  A \cup B &= \{i, o, u\} \cup \{a, e, o, u\}\\
                  &= \{a, e, i, o, u\}\\
                  (A \cup B)' &= \{a, e, i, o, u\} - \{a, e, i, o, u\}\\
                  &= \emptyset
                \end{align*}
              \item \moveup
                \begin{align*}
                  A' \cap B' &= \{a, e\} \cap \{i\}\\
                  &= \emptyset
                \end{align*}
              \item \moveup
                \begin{align*}
                  A \cap B &= \{i, o, u\} \cap \{a, e, o, u\}\\
                  &= \{o, u\}\\
                  (A \cap B)' &= \{a, e, i, o, u\} - \{o, u\}\\
                  &= \{a, e, i\}
                \end{align*}
              \item \moveup
                \begin{align*}
                  A' \cup B' &= \{a, e\} \cup \{i\}\\
                  &= \{a, e, i\}
                \end{align*}
              \item \moveup
                \begin{align*}
                  A - B &= \{i, o, u\} - \{a, e, o, u\}\\
                  &= \{i\}\\
                  B - A &= \{a, e, o, u\} - \{i, o, u\}\\
                  &= \{a,e\}
                \end{align*}
              \item \moveup
                \begin{align*}
                  A \cap B' &= \{i, o, u\} \cap \{i\}\\
                  &= \{i\}\\
                  B \cap A' &= \{a, e, o, u\} \cap \{a, e\}\\
                  &= \{a, e\}
                \end{align*}
              \item \moveup
                \begin{align*}
                  A + B &= \{i, o, u\} + \{a, e, o, u\}\\
                  &= \{a, e, i\}\\
                  B + A &= \{a, e, o, u\} + \{i, o, u\}\\
                  &= \{a, e, i\}
                \end{align*}
            \end{enumerate}
          \item \moveup
            \begin{alignat*}{3}
              U &= \{1, 2, 3, 4, 5\} \qquad & A &= \{3\} \qquad & B &= \bigl\{\{3\}, 4, 5\bigr\}
            \end{alignat*}
            \begin{align*}
              \mathcal{P}(A) &= \mathcal{P}\bigl(\{3\}\bigr)\\
              &= \bigl\{\emptyset, \{3\}\bigr\}\\
              \mathcal{P}(B) &= \mathcal{P}\Bigl(\bigl\{\{3\}, 4, 5\bigr\}\Bigr)\\
              &= \Bigl\{\emptyset, \bigl\{\{3\}\bigr\}, \bigl\{4\bigr\}, \bigl\{5\bigr\}, \bigl\{\{3\}, 4\bigr\}, \bigl\{\{3\}, 5\bigr\}, \bigl\{4, 5\bigr\}, \bigl\{\{3\}, 4, 5\bigr\}\Bigr\}
            \end{align*}
          \pagebreak
          \item \moveup
            \begin{alignat*}{3}
              U &= \{a, e, i, o, u\} \qquad & A &= \{i, o, u\} \qquad & B &= \{a, e, o, u\}
            \end{alignat*}
            \begin{enumerate}[label=(\alph*)]
              \item \moveup
                \begin{align*}
                  \mathcal{P}(A) &= \bigl\{\emptyset, \{i\}, \{o\}, \{u\}, \{i, o\}, \{i, u\}, \{o, u\}, \{i, o, u\}\bigr\}\\
                  \mathcal{P}(B) &= \bigl\{\emptyset, \{a\}, \{e\}, \{o\}, \{u\}, \{a, e\}, \{a, o\}, \{a, u\}, \{e, o\}, \{e, u\}, \{o, u\}, \{a, e, o\},\\
                  & \qquad \{a, e, u\}, \{a, o, u\}, \{e, o, u\}, \{a, e, o, u\}\bigr\}
                \end{align*}
              \item \moveup
                \begin{align*}
                  \mathcal{P}(A \cap B) &= \mathcal{P}\bigl(\{o, u\}\bigr)\\
                  &= \bigl\{\emptyset, \{o\}, \{u\}, \{o, u\}\bigr\}\\
                  \mathcal{P}(A) \cap \mathcal{P}(B) &= \bigl\{\emptyset, \{o\}, \{u\}, \{o, u\}\bigr\}
                \end{align*}
              \item \moveup
                \begin{align*}
                  \mathcal{P}\left(A'\right) &= \mathcal{P}\bigl(\{a, e\}\bigr)\\
                  &= \bigl\{\emptyset, \{a\}, \{e\}, \{a, e\}\bigr\}
                \end{align*}
                \begin{align*}
                  \mathcal{P}(U) &= \bigl\{\emptyset, \{a\}, \{e\}, \{i\}, \{o\}, \{u\}, \{a, e\}, \{a, i\}, \{a, o\}, \{a, u\},\\
                  & \qquad \{e, i\}, \{e, o\}, \{e, u\}, \{i, o\}, \{i, u\}, \{o, u\},\\
                  & \qquad \{a, e, i\}, \{a, e, o\}, \{a, e, u\}, \{a, i, o\}, \{a, i, u\}, \{a, o, u\},\\
                  & \qquad \{e, i, o\}, \{e, i, u\} \{e, o, u\}, \{i, o, u\},\\
                  & \qquad \{a, e, i, o\}, \{a, e, i, u\}, \{a, e, o, u\} \{a, i, o, u\}, \{e, i, o, u\}, \{a, e, i, o, u\}\bigr\}
                \end{align*}
                \begin{align*}
                  \bigl(\mathcal{P}(A)\bigr)' &= \bigl\{\{a\}, \{e\}, \{a, e\}, \{a, i\}, \{a, o\}, \{a, u\}, \{e, i\}, \{e, o\}, \{e, u\},\\
                  & \qquad \{a, e, i\}, \{a, e, o\}, \{a, e, u\}, \{a, i, o\}, \{a, i, u\}, \{a, o, u\}, \\
                  & \qquad \{e, i, o\}, \{e, i, u\},  \{e, o, u\}, \\
                  & \qquad \{a, e, i, o\}, \{a, e, i, u\}, \{a, e, o, u\} \{a, i, o, u\}, \{e, i, o, u\}, \{a, e, i, o, u\}\bigr\}
                \end{align*}
              \item \moveup
                \begin{align*}
                  \mathcal{P}(A) \cup \mathcal{P}(B) &= \bigl\{\emptyset, \{a\}, \{e\}, \{i\}, \{o\}, \{u\},\\
                  & \qquad \{a, e\}, \{a, o\}, \{a, u\}, \{e, o\}, \{e, u\}, \{i, o\}, \{i, u\}, \{o, u\}, \\
                  & \qquad \{a, e, o\}, \{a, e, u\}, \{a, o, u\}, \{e, o, u\}, \{i, o, u\}, \{a, e, o, u\}\bigr\}
                \end{align*}
                \begin{align*}
                  \mathcal{P}(A \cup B) &= \mathcal{P}\bigl(\{a, e, i, o, u\}\bigr)\\
                  &= \mathcal{P}(U)\\
                  &= \bigl\{\emptyset, \{a\}, \{e\}, \{i\}, \{o\}, \{u\}, \{a, e\}, \{a, i\}, \{a, o\}, \{a, u\},\\
                  & \qquad \{e, i\}, \{e, o\}, \{e, u\}, \{i, o\}, \{i, u\}, \{o, u\},\\
                  & \qquad \{a, e, i\}, \{a, e, o\}, \{a, e, u\}, \{a, i, o\}, \{a, i, u\}, \{a, o, u\},\\
                  & \qquad \{e, i, o\}, \{e, i, u\} \{e, o, u\}, \{i, o, u\},\\
                  & \qquad \{a, e, i, o\}, \{a, e, i, u\}, \{a, e, o, u\} \{a, i, o, u\}, \{e, i, o, u\}, \{a, e, i, o, u\}\bigr\}
                \end{align*}
            \end{enumerate}
        \end{enumerate}
      \end{exercise}
  \rulechapterend
\end{document}
