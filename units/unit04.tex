\documentclass[../notes.tex]{subfiles}

\begin{document}
	\chapter{Proofs Involving Sets}
		\section{Venn Diagrams}
			\subsection{Set Equality}
				For any sets $A$ and $B$, if $A \subseteq B$ and $B \subseteq A$, then every element of $A$ is also an element of $B$, and every element of $B$ is also an element of $A$, so $A = B$.
			\subsection{Drawing Complex Venn Diagrams}
			Draw the diagram in stages, including $U$, $A$, $B$ and $C$ in each diagram.
			\begin{examplebox}
				Let $A$, $B$, $C \subseteq U$. Draw the Venn diagram for $\bigl[(A \cup B) - (A \cap B)\bigr] \cup C$
				\begin{center}
					\begin{venndiagram3sets}[shade=circle area]
						\setpostvennhook
						{
							\node[above] at (current bounding box.north) {$A \cup B$};
						}
						\fillAll[fill=white]
						\fillA
						\fillB
					\end{venndiagram3sets}
					\begin{venndiagram3sets}[shade=circle area]
						\setpostvennhook
						{
							\node[above] at (current bounding box.north) {$A \cap B$};
						}
						\fillAll[fill=white]
						\fillACapB
					\end{venndiagram3sets}
					\pagebreak
					\begin{venndiagram3sets}[shade=circle area]
						\setpostvennhook
						{
							\node[above] at (current bounding box.north) {$(A \cup B) - (A \cap B)$};
						}
						\fillAll[fill=white]
						\fillANotB
						\fillBNotA
					\end{venndiagram3sets}
					\begin{venndiagram3sets}[shade=circle area]
						\setpostvennhook
						{
							\node[above] at (current bounding box.north) {$C$};
						}
						\fillAll[fill=white]
						\fillC
					\end{venndiagram3sets}
					\begin{venndiagram3sets}[shade=circle area]
						\setpostvennhook
						{
							\node[above] at (current bounding box.north) {$\bigl[(A \cup B) - (A \cap B)\bigr] \cup C$};
						}
						\fillAll[fill=white]
						\fillANotB
						\fillBNotA
						\fillC
					\end{venndiagram3sets}
				\end{center}
			\end{examplebox}
			\begin{exercisebox}{Self-Assessment Exercise \thechapter.4}
				\begin{enumerate}
					\item
						\begin{enumerate}
							\item $(X \cup Y)'$
								\begin{center}
									\begin{venndiagram2sets}[shade=circle area, labelA=$X$, labelB=$Y$, tikzoptions={scale=0.8}]
										\setpostvennhook
										{
											\node[above] at (current bounding box.north) {$X \cup Y$};
										}
										\fillAll[fill=white]
										\fillA
										\fillB
									\end{venndiagram2sets}
									\begin{venndiagram2sets}[shade=circle area, labelA=$X$, labelB=$Y$, tikzoptions={scale=0.8}]
										\setpostvennhook
										{
											\node[above] at (current bounding box.north) {$(X \cup Y)'$};
										}
										\fillAll[fill=white]
										\fillNotAorB
									\end{venndiagram2sets}
								\end{center}
							\item $X' \cap Y'$
								\begin{center}
									\begin{venndiagram2sets}[shade=circle area, labelA=$X$, labelB=$Y$, tikzoptions={scale=0.8}]
										\setpostvennhook
										{
											\node[above] at (current bounding box.north) {$X'$};
										}
										\fillAll[fill=white]
										\fillNotA
									\end{venndiagram2sets}
									\begin{venndiagram2sets}[shade=circle area, labelA=$X$, labelB=$Y$, tikzoptions={scale=0.8}]
										\setpostvennhook
										{
											\node[above] at (current bounding box.north) {$Y'$};
										}
										\fillAll[fill=white]
										\fillNotB
									\end{venndiagram2sets}
									\begin{venndiagram2sets}[shade=circle area, labelA=$X$, labelB=$Y$, tikzoptions={scale=0.8}]
										\setpostvennhook
										{
											\node[above] at (current bounding box.north) {$X' \cap Y'$};
										}
										\fillAll[fill=white]
										\fillNotAorB
									\end{venndiagram2sets}
								\end{center}
							\item $(X \cap Y)'$
								\begin{center}
									\begin{venndiagram2sets}[shade=circle area, labelA=$X$, labelB=$Y$, tikzoptions={scale=0.8}]
										\setpostvennhook
										{
											\node[above] at (current bounding box.north) {$X \cap Y$};
										}
										\fillAll[fill=white]
										\fillACapB
									\end{venndiagram2sets}
									\begin{venndiagram2sets}[shade=circle area, labelA=$X$, labelB=$Y$, tikzoptions={scale=0.8}]
										\setpostvennhook
										{
											\node[above] at (current bounding box.north) {$(X \cap Y)'$};
										}
										\fillAll[fill=white]
										\fillNotAorNotB
									\end{venndiagram2sets}
								\end{center}
							\pagebreak
							\item $X' \cap Y'$
								\begin{center}
									\begin{venndiagram2sets}[shade=circle area, labelA=$X$, labelB=$Y$, tikzoptions={scale=0.8}]
										\setpostvennhook
										{
											\node[above] at (current bounding box.north) {$X'$};
										}
										\fillAll[fill=white]
										\fillNotA
									\end{venndiagram2sets}
									\begin{venndiagram2sets}[shade=circle area, labelA=$X$, labelB=$Y$, tikzoptions={scale=0.8}]
										\setpostvennhook
										{
											\node[above] at (current bounding box.north) {$Y'$};
										}
										\fillAll[fill=white]
										\fillNotB
									\end{venndiagram2sets}
									\begin{venndiagram2sets}[shade=circle area, labelA=$X$, labelB=$Y$, tikzoptions={scale=0.8}]
										\setpostvennhook
										{
											\node[above] at (current bounding box.north) {$X' \cup Y'$};
										}
										\fillAll[fill=white]
										\fillNotA
										\fillNotB
									\end{venndiagram2sets}
								\end{center}
						\end{enumerate}
					\item
						\begin{enumerate}
							\item $X - (Y \cup Z)$
								\begin{center}
									\begin{venndiagram3sets}[shade=circle area, labelA=$X$, labelB=$Y$, labelC=$Z$, tikzoptions={scale=0.8}]
										\setpostvennhook
										{
											\node[above] at (current bounding box.north) {$X$};
										}
										\fillAll[fill=white]
										\fillA
									\end{venndiagram3sets}
									\begin{venndiagram3sets}[shade=circle area, labelA=$X$, labelB=$Y$, labelC=$Z$, tikzoptions={scale=0.8}]
										\setpostvennhook
										{
											\node[above] at (current bounding box.north) {$Y \cup Z$};
										}
										\fillAll[fill=white]
										\fillB
										\fillC
									\end{venndiagram3sets}
									\begin{venndiagram3sets}[shade=circle area, labelA=$X$, labelB=$Y$, labelC=$Z$, tikzoptions={scale=0.8}]
										\setpostvennhook
										{
											\node[above] at (current bounding box.north) {$X - (Y \cup Z)$};
										}
										\fillAll[fill=white]
										\fillOnlyA
									\end{venndiagram3sets}
								\end{center}
							\item $(X - Y) \cup (X - Z)$
								\begin{center}
									\begin{venndiagram3sets}[shade=circle area, labelA=$X$, labelB=$Y$, labelC=$Z$, tikzoptions={scale=0.8}]
										\setpostvennhook
										{
											\node[above] at (current bounding box.north) {$X - Y$};
										}
										\fillAll[fill=white]
										\fillANotB
									\end{venndiagram3sets}
									\begin{venndiagram3sets}[shade=circle area, labelA=$X$, labelB=$Y$, labelC=$Z$, tikzoptions={scale=0.8}]
										\setpostvennhook
										{
											\node[above] at (current bounding box.north) {$X - Z$};
										}
										\fillAll[fill=white]
										\fillANotC
									\end{venndiagram3sets}
									\begin{venndiagram3sets}[shade=circle area, labelA=$X$, labelB=$Y$, labelC=$Z$, tikzoptions={scale=0.8}]
										\setpostvennhook
										{
											\node[above] at (current bounding box.north) {$(X - Y) \cup (X - Z)$};
										}
										\fillAll[fill=white]
										\fillANotB
										\fillANotC
									\end{venndiagram3sets}
								\end{center}
							\item $X \cap (Y - Z)$
								\begin{center}
									\begin{venndiagram3sets}[shade=circle area, labelA=$X$, labelB=$Y$, labelC=$Z$, tikzoptions={scale=0.8}]
										\setpostvennhook
										{
											\node[above] at (current bounding box.north) {$X$};
										}
										\fillAll[fill=white]
										\fillA
									\end{venndiagram3sets}
									\begin{venndiagram3sets}[shade=circle area, labelA=$X$, labelB=$Y$, labelC=$Z$, tikzoptions={scale=0.8}]
										\setpostvennhook
										{
											\node[above] at (current bounding box.north) {$Y - Z$};
										}
										\fillAll[fill=white]
										\fillBNotC
									\end{venndiagram3sets}
									\begin{venndiagram3sets}[shade=circle area, labelA=$X$, labelB=$Y$, labelC=$Z$, tikzoptions={scale=0.8}]
										\setpostvennhook
										{
											\node[above] at (current bounding box.north) {$X \cap (Y - Z)$};
										}
										\fillAll[fill=white]
										\fillACapBNotC
									\end{venndiagram3sets}
								\end{center}
							\pagebreak
							\item $(X \cap Y) - (X \cap Z)$
								\begin{center}
									\begin{venndiagram3sets}[shade=circle area, labelA=$X$, labelB=$Y$, labelC=$Z$, tikzoptions={scale=0.8}]
										\setpostvennhook
										{
											\node[above] at (current bounding box.north) {$X \cap Y$};
										}
										\fillAll[fill=white]
										\fillACapB
									\end{venndiagram3sets}
									\begin{venndiagram3sets}[shade=circle area, labelA=$X$, labelB=$Y$, labelC=$Z$, tikzoptions={scale=0.8}]
										\setpostvennhook
										{
											\node[above] at (current bounding box.north) {$X \cap Z$};
										}
										\fillAll[fill=white]
										\fillACapC
									\end{venndiagram3sets}
									\begin{venndiagram3sets}[shade=circle area, labelA=$X$, labelB=$Y$, labelC=$Z$, tikzoptions={scale=0.8}]
										\setpostvennhook
										{
											\node[above] at (current bounding box.north) {$(X \cap Y) - (X \cap Z)$};
										}
										\fillAll[fill=white]
										\fillACapBNotC
									\end{venndiagram3sets}
								\end{center}
							\item $X \cap (Y + Z)$
								\begin{center}
									\begin{venndiagram3sets}[shade=circle area, labelA=$X$, labelB=$Y$, labelC=$Z$, tikzoptions={scale=0.8}]
										\setpostvennhook
										{
											\node[above] at (current bounding box.north) {$X$};
										}
										\fillAll[fill=white]
										\fillA
									\end{venndiagram3sets}
									\begin{venndiagram3sets}[shade=circle area, labelA=$X$, labelB=$Y$, labelC=$Z$, tikzoptions={scale=0.8}]
										\setpostvennhook
										{
											\node[above] at (current bounding box.north) {$Y + Z$};
										}
										\fillAll[fill=white]
										\fillBNotC
										\fillCNotB
									\end{venndiagram3sets}
									\begin{venndiagram3sets}[shade=circle area, labelA=$X$, labelB=$Y$, labelC=$Z$, tikzoptions={scale=0.8}]
										\setpostvennhook
										{
											\node[above] at (current bounding box.north) {$X \cap (Y + Z)$};
										}
										\fillAll[fill=white]
										\fillACapBNotC
										\fillACapCNotB
									\end{venndiagram3sets}
								\end{center}
							\item $(X \cap Y) + (X \cap Z)$
								\begin{center}
									\begin{venndiagram3sets}[shade=circle area, labelA=$X$, labelB=$Y$, labelC=$Z$, tikzoptions={scale=0.8}]
										\setpostvennhook
										{
											\node[above] at (current bounding box.north) {$X \cap Y$};
										}
										\fillAll[fill=white]
										\fillACapB
									\end{venndiagram3sets}
									\begin{venndiagram3sets}[shade=circle area, labelA=$X$, labelB=$Y$, labelC=$Z$, tikzoptions={scale=0.8}]
										\setpostvennhook
										{
											\node[above] at (current bounding box.north) {$X \cap Z$};
										}
										\fillAll[fill=white]
										\fillACapC
									\end{venndiagram3sets}
									\begin{venndiagram3sets}[shade=circle area, labelA=$X$, labelB=$Y$, labelC=$Z$, tikzoptions={scale=0.8}]
										\setpostvennhook
										{
											\node[above] at (current bounding box.north) {$(X \cap Y) + (X \cap Z)$};
										}
										\fillAll[fill=white]
										\fillACapBNotC
										\fillACapCNotB
									\end{venndiagram3sets}
								\end{center}
						\end{enumerate}
				\end{enumerate}
			\end{exercisebox}
		\pagebreak
		\section{Proofs}
			For proofs with sets, one needs to prove that the sets have exactly the same elements. For this, one needs to show that each half of the equation is equal to the other half: one needs to show both forwards and backwards. However, this can be abbreviated using iff.
			\begin{examplebox}
				(Long Way:) Prove that for all subsets $A$ and $B$ of $U$, $A \cup B = B \cup A$
				\begin{proof}
					Show $(A \cup B) \subseteq (B \cup A)$
					\begin{tabbing}
						Let $\qquad$ \=$x \in (A \cup B)$\\
						If \>$x \in (A \cup B)$\\
						then \> $x \in A$ or $x \in B$\\
						i.e. \> $x \in B$ or $x \in A$\\
						i.e. \> $x \in (B \cup A)$
					\end{tabbing}
					$\therefore$ if $x \in (A \cup B)$, then $x \in (B \cup A)$,\\
					$\therefore (A \cup B) \subseteq (B \cup A)$.

					\vspace{7mm}
					Show $(B \cup A) \subseteq (A \cup B)$
					\begin{tabbing}
						Let $\qquad$ \=$x \in (B \cup A)$\\
						If \>$x \in (B \cup A)$\\
						then \> $x \in B$ or $x \in A$\\
						i.e. \> $x \in A$ or $x \in B$\\
						i.e. \> $x \in (A \cup B)$
					\end{tabbing}
					$\therefore$ if $x \in (B \cup A)$, then $x \in (A \cup B)$,\\
					$\therefore (B \cup A) \subseteq (A \cup B)$.

					\vspace{5mm}
					$\therefore A \cup B = B \cup A$
				\end{proof}
			\end{examplebox}
			Using iff can shorten this, but be careful!
			\begin{examplebox}
				\begin{tabbing}
					$\quad$ \=$x \in (X \cup Y)'$\\
					iff \>$x \notin (X \cup Y)$\\
					iff \>$x \notin X$ and $x \notin Y$\\
					iff \>$x \in X'$ and $x \in Y'$\\
					iff \>$x \in X' \cap Y'$
				\end{tabbing}
				\begin{tabbing}
					$\quad$ \=$x \in (X \cap Y)'$\\
					iff \> $x \notin (X \cap Y)$\\
					iff \> $x \notin X$ or $x \notin Y$\\
					iff \> $x \in X'$ or $x \in Y'$\\
					iff \> $x \in X' \cup Y'$
				\end{tabbing}
			\end{examplebox}
			\pagebreak
			\begin{exercisebox}{Self-Assessment Exercise \thechapter.6}
				\begin{enumerate}[label=(\alph*)]
					\item $(X')' = X$\\
						Let $x \in (X')'$.
						\begin{tabbing}
							$\quad$ \= $x \in (X')'$\\
							iff \> $x \notin X'$\\
							iff \> $x \in X$\\
							$\therefore$ \> $(X')' = X$
						\end{tabbing}
					\item $X - (Y \cap W) = (X - Y) \cup (X - W)$\\
						Let $x \in X - (Y \cap W)$.
						\begin{tabbing}
							$\quad$ \= $x \in X - (Y \cap W)$\\
							iff \> $x \in X$ and $x \notin (Y \cap W)$\\
							iff \> $x \in X$ and $x \in Y'$ or $x \in W'$\\
							iff \> $\bigl(x \in X$ and $x \in Y'\bigr)$ or $\bigl(x \in X$ and $x \in W'\bigr)$\\
							iff \> $\bigl(x \in (X - Y)\bigr)$ or $\bigl(x \in (X - W)\bigr)$\\
							iff \> $x \in (X - Y) \cup (X - W)$\\
							$\therefore$ \> $X - (Y \cap W) = (X - Y) \cup (X - W)$
						\end{tabbing}
					\item $X \cap (Y \cap W) = (X \cap Y) \cap W$\\
						Let $x \in X \cap (Y \cap W)$.
						\begin{tabbing}
							$\quad$ \= $x \in X \cap (Y \cap W)$\\
							iff \> $x \in X$ and $x \in (Y \cap W)$\\
							iff \> $x \in X$ and $x \in Y$ and $x \in W$\\
							iff \> $(x \in X$ and $x \in Y)$ and $x \in W$\\
							iff \> $x \in (X \cap Y)$ and $x \in W$\\
							iff \> $x \in (X \cap Y) \cap W$\\
							$\therefore$ \> $X \cap (Y \cap W) = (X \cap Y) \cap W$
						\end{tabbing}
					\item $X \cap (Y \cup W) = (X \cap Y) \cup (X \cap W)$\\
						Let $x \in X \cap (Y \cup W)$
						\begin{tabbing}
							$\quad$ \= $x \in X \cap (Y \cup W)$\\
							iff \> $x \in X$ and $x \in (Y \cup W)$\\
							iff \> $x \in X$ and ($x \in Y$ or $x \in W$)\\
							iff \> ($x \in X$ and $x \in Y$) or ($x \in X$ and $x \in W$)\\
							iff \> ($x \in (X \cap Y)$) or ($x \in (X \cap W)$)\\
							iff \> $x \in (X \cap Y) \cup (X \cap W)$\\
							$\therefore$ \> $X \cap (Y \cup W) = (X \cap Y) \cup (X \cap W)$
						\end{tabbing}
				\end{enumerate}
			\end{exercisebox}
			In order to prove that two sets are not equal, one needs to just provide a \textbf{counterexample} - an element that is in one set that is not in the other.
		\section{The Inclusion Exclusion Principle}
			For all finite sets $X$ and $Y$, $\left\lvert X \cup Y\right\rvert = \left\lvert X\right\rvert + \left\lvert Y \right\rvert - \left\lvert X \cap Y\right\rvert $
			\begin{examplebox}
				Let $X = \{a, b, c, 1\}$ and $Y = \{1, 2, 3\}$. Then $X \cap Y = \{1\}$ and $\left\lvert X \cap Y\right\rvert = 1$.\\
				$\left\lvert X\right\rvert = 4$, $\left\lvert Y\right\rvert = 3$, so $\left\lvert X \cup Y\right\rvert = \left\lvert X\right\rvert + \left\lvert Y\right\rvert - \left\lvert X \cap Y\right\rvert = 4 + 3 - 1 = 6$
			\end{examplebox}
			\subsection{Sum Rule}
				If $X$ and $Y$ are disjoint sets ($X \cap Y = \emptyset$), and $\left\lvert X\right\rvert = m$ and $\left\lvert Y\right\rvert = n $, then $\left\lvert X \cup Y\right\rvert = m + n $
			\subsection{Applying the principle to Venn Diagrams}
				\begin{examplebox}
					In a group of 50 learners, 25 play mastermind, 30 play basketball, and 10 play both.
					\begin{enumerate}[label=(\alph*)]
						\item How many learners play Mastermind or basketball, (or both)?
						\item How many students do not play either Mastermind or basketball?
					\end{enumerate}
					$U$ is all the learners, $M$ is those who play Mastermind, and $B$ is those who play basketball.
					\begin{alignat*}{4}
						\left\lvert U\right\rvert &= 50 \qquad & \left\lvert M\right\rvert &= 25 \qquad & \left\lvert B\right\rvert &= 30 \qquad & \left\lvert M \cap B \right\rvert = 10
					\end{alignat*}
					\begin{center}
						\begin{venndiagram2sets}[shade=circle area, showframe=true, radius=2.4cm, overlap=1.2cm, vgap=1cm, labelA={$M$}, labelAB={$10$}, labelOnlyA={$25 - 10 = 15$}, labelOnlyB={$30 - 10 = 20$}]
							\fillAll[fill=white]
							\setpostvennhook
							{
								\node[above] at (current bounding box.south) {$50 - 15 - 20 - 10 = 5$};
							}
						\end{venndiagram2sets}
					\end{center}
					\pagebreak
					\begin{enumerate}
						\item $\left\lvert M \cup B\right\rvert = 15 + 10 + 20 = 45$.\\
							Also by Inclusion Exclusion,
							\begin{align*}
								\left\lvert M \cup B\right\rvert &= \left\lvert M\right\rvert + \left\lvert B\right\rvert - \left\lvert M \cap B\right\rvert\\
								&= 25 + 30 - 10\\
								&= 45
							\end{align*}
						\item $\left\lvert (M \cup B)'\right\rvert = 50 - 45 = 5$
					\end{enumerate}
				\end{examplebox}
				% TODO: COME BACK TO - NOTES ARE NOT DONE
				\noindent\rule{\textwidth}{0.4pt}
	\end{document}