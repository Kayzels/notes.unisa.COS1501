\providecommand{\main}{..}
\documentclass[\main/notes.tex]{subfiles}

\begin{document}
	\ifSubfilesClassLoaded{\setcounter{chapter}{3}}{}
	\addtocontents{toc}{\protect\newpage}
	\chapter{Proofs Involving Sets}
		\section{Venn Diagrams}
			\subsection{Set Equality}
				\begin{definition}{Set Equality}
					For any sets $A$ and $B$, if $A \subseteq B$ and $B \subseteq A$, then every element of $A$ is also an element of $B$, and every element of $B$ is also an element of $A$, so $A = B$.
				\end{definition}
			\subsection{Drawing Complex Venn Diagrams}
			Draw the diagram in stages, including $U$, $A$, $B$ and $C$ in each diagram.
			\begin{example}
				Let $A$, $B$, $C \subseteq U$. Draw the Venn diagram for $\bigl[(A \cup B) - (A \cap B)\bigr] \cup C$
				\begin{center}
					\begin{vennthree}[][$A \cup B$]
						\fillA
						\fillB
					\end{vennthree}
					\begin{vennthree}[][$A \cap B$]
						\fillACapB
					\end{vennthree}
					\pagebreak
					\begin{vennthree}[][$(A \cup B) - (A \cap B)$]
						\fillANotB
						\fillBNotA
					\end{vennthree}
					\begin{vennthree}[][$C$]
						\fillC
					\end{vennthree}
					\begin{vennthree}[][$\bigl{[}(A \cup B) - (A \cap B)\bigr{]} \cup C$]
						\fillANotB
						\fillBNotA
						\fillC
					\end{vennthree}
				\end{center}
			\end{example}
			\begin{exercise}{Self-Assessment Exercise \thechapter.4}
				\begin{enumerate}
					\item
						\begin{enumerate}
							\item $(X \cup Y)'$
								\begin{center}
									\begin{venntwo}[labelA=$X$, labelB=$Y$, tikzoptions={scale=0.8}][$X \cup Y$]
										\fillA
										\fillB
									\end{venntwo}
									\begin{venntwo}[labelA=$X$, labelB=$Y$, tikzoptions={scale=0.8}][$(X \cup Y)'$]
										\fillNotAorB
									\end{venntwo}
								\end{center}
							\item $X' \cap Y'$
								\begin{center}
									\begin{venntwo}[labelA=$X$, labelB=$Y$, tikzoptions={scale=0.8}][$X'$]
										\fillNotA
									\end{venntwo}
									\begin{venntwo}[labelA=$X$, labelB=$Y$, tikzoptions={scale=0.8}][$Y'$]
										\fillNotB
									\end{venntwo}
									\begin{venntwo}[labelA=$X$, labelB=$Y$, tikzoptions={scale=0.8}][$X' \cap Y'$]
										\fillNotAorB
									\end{venntwo}
								\end{center}
							\item $(X \cap Y)'$
								\begin{center}
									\begin{venntwo}[labelA=$X$, labelB=$Y$, tikzoptions={scale=0.8}][$X \cap Y$]
										\fillACapB
									\end{venntwo}
									\begin{venntwo}[labelA=$X$, labelB=$Y$, tikzoptions={scale=0.8}][$(X \cap Y)'$]
										\fillNotAorNotB
									\end{venntwo}
								\end{center}
							\pagebreak
							\item $X' \cap Y'$
								\begin{center}
									\begin{venntwo}[labelA=$X$, labelB=$Y$, tikzoptions={scale=0.8}][$X'$]
										\fillNotA
									\end{venntwo}
									\begin{venntwo}[labelA=$X$, labelB=$Y$, tikzoptions={scale=0.8}][$Y'$]
										\fillNotB
									\end{venntwo}
									\begin{venntwo}[labelA=$X$, labelB=$Y$, tikzoptions={scale=0.8}][$X' \cup Y'$]
										\fillNotA
										\fillNotB
									\end{venntwo}
								\end{center}
						\end{enumerate}
					\item
						\begin{enumerate}
							\item $X - (Y \cup Z)$
								\begin{center}
									\begin{vennthree}[labelA=$X$, labelB=$Y$, labelC=$Z$, tikzoptions={scale=0.8}][$X$]
										\fillA
									\end{vennthree}
									\begin{vennthree}[labelA=$X$, labelB=$Y$, labelC=$Z$, tikzoptions={scale=0.8}][$Y \cup Z$]
										\fillB
										\fillC
									\end{vennthree}
									\begin{vennthree}[labelA=$X$, labelB=$Y$, labelC=$Z$, tikzoptions={scale=0.8}][$X - (Y \cup Z)$]
										\fillOnlyA
									\end{vennthree}
								\end{center}
							\item $(X - Y) \cup (X - Z)$
								\begin{center}
									\begin{vennthree}[labelA=$X$, labelB=$Y$, labelC=$Z$, tikzoptions={scale=0.8}][$X - Y$]
										\fillANotB
									\end{vennthree}
									\begin{vennthree}[labelA=$X$, labelB=$Y$, labelC=$Z$, tikzoptions={scale=0.8}][$X - Z$]
										\fillANotC
									\end{vennthree}
									\begin{vennthree}[labelA=$X$, labelB=$Y$, labelC=$Z$, tikzoptions={scale=0.8}][$(X - Y) \cup (X - Z)$]
										\fillANotB
										\fillANotC
									\end{vennthree}
								\end{center}
							\item $X \cap (Y - Z)$
								\begin{center}
									\begin{vennthree}[labelA=$X$, labelB=$Y$, labelC=$Z$, tikzoptions={scale=0.8}][$X$]
										\fillA
									\end{vennthree}
									\begin{vennthree}[labelA=$X$, labelB=$Y$, labelC=$Z$, tikzoptions={scale=0.8}][$Y - Z$]
										\fillBNotC
									\end{vennthree}
									\begin{vennthree}[labelA=$X$, labelB=$Y$, labelC=$Z$, tikzoptions={scale=0.8}][$X \cap (Y - Z)$]
										\fillACapBNotC
									\end{vennthree}
								\end{center}
							\pagebreak
							\item $(X \cap Y) - (X \cap Z)$
								\begin{center}
									\begin{vennthree}[labelA=$X$, labelB=$Y$, labelC=$Z$, tikzoptions={scale=0.8}][$X \cap Y$]
										\fillACapB
									\end{vennthree}
									\begin{vennthree}[labelA=$X$, labelB=$Y$, labelC=$Z$, tikzoptions={scale=0.8}][$X \cap Z$]
										\fillACapC
									\end{vennthree}
									\begin{vennthree}[labelA=$X$, labelB=$Y$, labelC=$Z$, tikzoptions={scale=0.8}][$(X \cap Y) - (X \cap Z)$]
										\fillACapBNotC
									\end{vennthree}
								\end{center}
							\item $X \cap (Y + Z)$
								\begin{center}
									\begin{vennthree}[labelA=$X$, labelB=$Y$, labelC=$Z$, tikzoptions={scale=0.8}][$X$]
										\fillA
									\end{vennthree}
									\begin{vennthree}[labelA=$X$, labelB=$Y$, labelC=$Z$, tikzoptions={scale=0.8}][$Y + Z$]
										\fillBNotC
										\fillCNotB
									\end{vennthree}
									\begin{vennthree}[labelA=$X$, labelB=$Y$, labelC=$Z$, tikzoptions={scale=0.8}][$X \cap (Y + Z)$]
										\fillACapBNotC
										\fillACapCNotB
									\end{vennthree}
								\end{center}
							\item $(X \cap Y) + (X \cap Z)$
								\begin{center}
									\begin{vennthree}[labelA=$X$, labelB=$Y$, labelC=$Z$, tikzoptions={scale=0.8}][$X \cap Y$]
										\fillACapB
									\end{vennthree}
									\begin{vennthree}[labelA=$X$, labelB=$Y$, labelC=$Z$, tikzoptions={scale=0.8}][$X \cap Z$]
										\fillACapC
									\end{vennthree}
									\begin{vennthree}[labelA=$X$, labelB=$Y$, labelC=$Z$, tikzoptions={scale=0.8}][$(X \cap Y) + (X \cap Z)$]
										\fillACapBNotC
										\fillACapCNotB
									\end{vennthree}
								\end{center}
						\end{enumerate}
				\end{enumerate}
			\end{exercise}
		\pagebreak
		\section{Proofs}
			For proofs with sets, one needs to prove that the sets have exactly the same elements. For this, one needs to show that each half of the equation is equal to the other half: one needs to show both forwards and backwards. However, this can be abbreviated using iff.
			\begin{example}
				(Long Way:) Prove that for all subsets $A$ and $B$ of $U$, $A \cup B = B \cup A$
				\begin{proof}
					Show (i) $(A \cup B) \subseteq (B \cup A)$ and (ii) $(B \cup A) \subseteq (A \cup B)$
					\begin{enumerate}[label=(\roman*)]
						\item Show $(A \cup B) \subseteq (B \cup A)$
							\begin{subproof}[Subproof]
								\removelastskip $ $
								\begin{tabbing}
									Let $\qquad$ \=$x \in (A \cup B)$\\
									If \>$x \in (A \cup B)$\\
									then \> $x \in A$ or $x \in B$\\
									i.e. \> $x \in B$ or $x \in A$\\
									i.e. \> $x \in (B \cup A)$
								\end{tabbing}
								$\therefore$ if $x \in (A \cup B)$, then $x \in (B \cup A)$,\\
								$\therefore (A \cup B) \subseteq (B \cup A)$.
							\end{subproof}
						\item Show $(B \cup A) \subseteq (A \cup B)$
							\begin{subproof}[Subproof]
								\removelastskip $ $
								\begin{tabbing}
									Let $\qquad$ \=$x \in (B \cup A)$\\
									If \>$x \in (B \cup A)$\\
									then \> $x \in B$ or $x \in A$\\
									i.e. \> $x \in A$ or $x \in B$\\
									i.e. \> $x \in (A \cup B)$
								\end{tabbing}
								$\therefore$ if $x \in (B \cup A)$, then $x \in (A \cup B)$,\\
								$\therefore (B \cup A) \subseteq (A \cup B)$.
							\end{subproof}
					\end{enumerate}
					$\therefore A \cup B = B \cup A$
				\end{proof}
			\end{example}
			Using iff can shorten this, but be careful!
			\begin{example}
				$ $\\
				\begin{minipage}{0.4\textwidth}
					\begin{proof}
						$ $
						\begin{tabbing}
							$\quad$ \=$x \in (X \cup Y)'$\\
							iff \>$x \notin (X \cup Y)$\\
							iff \>$x \notin X$ and $x \notin Y$\\
							iff \>$x \in X'$ and $x \in Y'$\\
							iff \>$x \in X' \cap Y'$
						\end{tabbing}
					\end{proof}
				\end{minipage}
				\hfill
				\begin{minipage}{0.4\textwidth}
					\begin{proof}
						$ $ 
						\begin{tabbing}
							$\quad$ \=$x \in (X \cap Y)'$\\
							iff \> $x \notin (X \cap Y)$\\
							iff \> $x \notin X$ or $x \notin Y$\\
							iff \> $x \in X'$ or $x \in Y'$\\
							iff \> $x \in X' \cup Y'$
						\end{tabbing}
					\end{proof}
				\end{minipage}
			\end{example}
			\pagebreak
			\begin{exercise}{Self-Assessment Exercise \thechapter.6}
				\begin{multicols}{2}
					\begin{enumerate}[label=(\alph*), itemsep=0.5em]
						\item $(X')' = X$
							\begin{proof}
								Let $x \in (X')'$.
								\begin{tabbing}
									$\quad$ \= $x \in (X')'$\\
									iff \> $x \notin X'$\\
									iff \> $x \in X$\\
									$\therefore$ \> $(X')' = X$
								\end{tabbing}
							\end{proof}
						\item $X - (Y \cap W) = (X - Y) \cup (X - W)$
							\begin{proof}
								Let $x \in X - (Y \cap W)$.
								\begin{tabbing}
									$\quad$ \= $x \in X - (Y \cap W)$\\
									iff \> $x \in X$ and $x \notin (Y \cap W)$\\
									iff \> $x \in X$ and $x \in Y'$ or $x \in W'$\\
									iff \> $\bigl(x \in X$ and $x \in Y'\bigr)$ or $\bigl(x \in X$ and $x \in W'\bigr)$\\
									iff \> $\bigl(x \in (X - Y)\bigr)$ or $\bigl(x \in (X - W)\bigr)$\\
									iff \> $x \in (X - Y) \cup (X - W)$\\
									$\therefore$ \> $X - (Y \cap W) = (X - Y) \cup (X - W)$
								\end{tabbing}
							\end{proof}
						\columnbreak
						\item $X \cap (Y \cap W) = (X \cap Y) \cap W$
							\begin{proof}
								Let $x \in X \cap (Y \cap W)$.
								\begin{tabbing}
									$\quad$ \= $x \in X \cap (Y \cap W)$\\
									iff \> $x \in X$ and $x \in (Y \cap W)$\\
									iff \> $x \in X$ and $x \in Y$ and $x \in W$\\
									iff \> $(x \in X$ and $x \in Y)$ and $x \in W$\\
									iff \> $x \in (X \cap Y)$ and $x \in W$\\
									iff \> $x \in (X \cap Y) \cap W$\\
									$\therefore$ \> $X \cap (Y \cap W) = (X \cap Y) \cap W$
								\end{tabbing}
							\end{proof}
						\item $X \cap (Y \cup W) = (X \cap Y) \cup (X \cap W)$
							\begin{proof}
								Let $x \in X \cap (Y \cup W)$
								\begin{tabbing}
									$\quad$ \= $x \in X \cap (Y \cup W)$\\
									iff \> $x \in X$ and $x \in (Y \cup W)$\\
									iff \> $x \in X$ and ($x \in Y$ or $x \in W$)\\
									iff \> ($x \in X$ and $x \in Y$) or ($x \in X$ and $x \in W$)\\
									iff \> ($x \in (X \cap Y)$) or ($x \in (X \cap W)$)\\
									iff \> $x \in (X \cap Y) \cup (X \cap W)$\\
									$\therefore$ \> $X \cap (Y \cup W) = (X \cap Y) \cup (X \cap W)$
								\end{tabbing}
							\end{proof}
					\end{enumerate}
				\end{multicols}
			\end{exercise}
			In order to prove that two sets are not equal, one needs to just provide a \concept{counterexample} - an element that is in one set that is not in the other.
			\begin{definition}{Identity}
				An equation which is satisfied by every possible value of the unknown(s) is called an \concept{identity}
			\end{definition}
		\pagebreak
		\begin{exercise}{Self Assessment Exercise \thechapter.8}
			\begin{enumerate}
				\item \question{Is it the case for all $X$, $Y$, $Z \subseteq U$, $X + (Y \cap Z) = (X + Y) \cap (X + Z)$?}
					\begin{center}
						\begin{vennthree}[labelA=$X$, labelB=$Y$, labelC=$Z$, tikzoptions={scale=0.8}][$X$]
							\fillA
						\end{vennthree}
						\begin{vennthree}[labelA=$X$, labelB=$Y$, labelC=$Z$, tikzoptions={scale=0.8}][$Y \cap Z$]
							\fillBCapC
						\end{vennthree}
						\begin{vennthree}[labelA=$X$, labelB=$Y$, labelC=$Z$, tikzoptions={scale=0.8}][$X + (Y \cap Z)$]
							\fillOnlyA
							\fillACapBNotC
							\fillACapCNotB
							\fillBCapCNotA
						\end{vennthree}
						\begin{vennthree}[labelA=$X$, labelB=$Y$, labelC=$Z$, tikzoptions={scale=0.8}][$X + Y$]
							\fillANotB
							\fillBNotA
						\end{vennthree}
						\begin{vennthree}[labelA=$X$, labelB=$Y$, labelC=$Z$, tikzoptions={scale=0.8}][$X + Z$]
							\fillANotC
							\fillCNotA
						\end{vennthree}
						\begin{vennthree}[labelA=$X$, labelB=$Y$, labelC=$Z$, tikzoptions={scale=0.8}][$X + (Y \cap Z)$]
							\fillOnlyA
							\fillBCapCNotA
						\end{vennthree}
					\end{center}
					As the venn diagrams are not the same, it is not the case.\\
					Counterexample: Find an element that is in $X$ and in $Y$, but is not in $Z$.
				\item \question{Find examples of sets $A$ and $B$ such that $\mathcal{P}(A \cup B)$ is not a subset of $\mathcal{P}(A) \cup \mathcal{P}(B)$}.\\
					$A$ and $B$ just need to contain different elements. For example, let $A = \{1\}$ and $B = \{2\}$.
					\begin{align*}
						\mathcal{P}(A \cup B) &= \mathcal{P}\bigl(\{1\} \cup \{2\}\bigr)\\
						&= \mathcal{P}\bigl(\{1, 2\}\bigr)\\
						&= \bigl\{\emptyset, \{1\}, \{2\}, \{1, 2\}\bigr\}\\
						\mathcal{P}(A) \cup \mathcal{P}(B) &= \mathcal{P}\bigl(\{1\}\bigr) \cup \mathcal{P}\bigl(\{2\}\bigr)\\
						&= \bigl\{\emptyset, \{1\}\bigr\} \cup \bigl\{\emptyset, \{2\}\bigr\}\\
						&= \bigl\{\emptyset, \{1\}, \{2\}\bigr\}
					\end{align*}
				\pagebreak
				\item \question{Is it the case that, for all $X, Y \subseteq U,\mathcal{P}(X) \cap \mathcal{P}(Y) = \mathcal{P}(X \cap Y)$?}\\
					Yes.
					\begin{proof}
						Let $S \in P(X) \cap P(Y)$.
						\begin{tabbing}
							$\qquad$ \= $S \in P(X) \cap P(Y)$\\
							iff \> $S \in P(X)$ and $S \in P(Y)$\\
							iff \> $S \subseteq X$ and $S \subseteq Y$\\
							iff \> (The elements of $S$ are all in $X$) and (The elements of $S$ are all in $Y$)\\
							iff \> The elements of $S$ are all in $X \cap Y$\\
							iff \> $S \subseteq X \cap Y$\\
							iff \> $S \in \mathcal{P}(X \cap Y)$
						\end{tabbing}
					\end{proof}
				\item \question{Use Venn diagrams to investigate whether, for all sets $X, Y, Z \subseteq U$ \\
				$X - (Y - Z) = (X - Y) - Z$. If it is true, provide a proof. Else, provide a counterexample.}
					\begin{center}
						\begin{vennthree}[labelA=$X$, labelB=$Y$, labelC=$Z$, tikzoptions={scale=0.8}][$X$]
							\fillA
						\end{vennthree}
						\begin{vennthree}[labelA=$X$, labelB=$Y$, labelC=$Z$, tikzoptions={scale=0.8}][$Y - Z$]
							\fillBNotC
						\end{vennthree}
						\begin{vennthree}[labelA=$X$, labelB=$Y$, labelC=$Z$, tikzoptions={scale=0.8}][$X - (Y - Z)$]
							\fillOnlyA
							\fillACapC
						\end{vennthree}
						\begin{vennthree}[labelA=$X$, labelB=$Y$, labelC=$Z$, tikzoptions={scale=0.8}][$X - Y$]
							\fillANotB
						\end{vennthree}
						\begin{vennthree}[labelA=$X$, labelB=$Y$, labelC=$Z$, tikzoptions={scale=0.8}][$Z$]
							\fillC
						\end{vennthree}
						\begin{vennthree}[labelA=$X$, labelB=$Y$, labelC=$Z$, tikzoptions={scale=0.8}][$(X - Y) - Z$]
							\fillOnlyA
						\end{vennthree}
					\end{center}
					\begin{proof}[Counterexample]
						Let $X = \{1, 2\}$, $Y = \{4\}$, $Z = \{1, 3\}$.
							\begin{align*}
								X - (Y - Z) &= \{1, 2\}\\
								(X - Y) - Z &= \{2\}\\
								\{1, 2\} &\neq \{2\}\\
								X - (Y - Z) &\neq (X - Y) - Z
							\end{align*}
					\end{proof}
				\pagebreak
				\item \question{Use Venn diagrams to investigate whether, for all sets $A, B, C \subseteq U$ \\
				$(A \cap B) + (C \cap A) = (A \cap B') \cup (B - C)$. If it is true, provide a proof. Else, provide a counterexample.}
					\begin{center}
						\begin{vennthree}[tikzoptions={scale=0.8}][$A \cap B$]
							\fillACapB
						\end{vennthree}
						\begin{vennthree}[tikzoptions={scale=0.8}][$C \cap A$]
							\fillACapC
						\end{vennthree}
						\begin{vennthree}[tikzoptions={scale=0.8}][$(A \cap B) + (C \cap A)$]
							\fillACapBNotC
							\fillACapCNotB
						\end{vennthree}
						\begin{vennthree}[tikzoptions={scale=0.8}][$B'$]
							\fillNotB
						\end{vennthree}
						\begin{vennthree}[tikzoptions={scale=0.8}][$A \cap B'$]
							\fillANotB
						\end{vennthree}
						\begin{vennthree}[tikzoptions={scale=0.8}][$B - C$]
							\fillBNotC
						\end{vennthree}
						\begin{vennthree}[tikzoptions={scale=0.8}][$(A \cap B') \cup (B - C)$]
							\fillANotB
							\fillBNotC
						\end{vennthree}
					\end{center}
					\begin{proof}[Counterexample]
						Let $A = \{1, 2\}$, $B = \{2, 3\}$, $C = \{4\}$
						\begin{align*}
							(A \cap B) + (C \cap A) &= \{2\}\\
							(A \cap B') \cup (B - C) &= \{1, 2, 3\}\\
							\{2\} &\neq \{1, 2, 3\}\\
							(A \cap B) + (C \cap A) &\neq (A \cap B') \cup (B - C)
						\end{align*} 
					\end{proof}
			\end{enumerate}
		\end{exercise}
		\pagebreak
		\section{The Inclusion Exclusion Principle}
			\begin{theorem}{Inclusion Exclusion Principle}
				For all finite sets $X$ and $Y$, $\left\lvert X \cup Y\right\rvert = \left\lvert X\right\rvert + \left\lvert Y \right\rvert - \left\lvert X \cap Y\right\rvert $
			\end{theorem}
			\begin{example}
				Let $X = \{a, b, c, 1\}$ and $Y = \{1, 2, 3\}$. Then $X \cap Y = \{1\}$ and $\left\lvert X \cap Y\right\rvert = 1$.\\
				$\left\lvert X\right\rvert = 4$, $\left\lvert Y\right\rvert = 3$, so $\left\lvert X \cup Y\right\rvert = \left\lvert X\right\rvert + \left\lvert Y\right\rvert - \left\lvert X \cap Y\right\rvert = 4 + 3 - 1 = 6$
			\end{example}
			\begin{theorem}{Sum Rule}
				If $X$ and $Y$ are disjoint sets ($X \cap Y = \emptyset$), and $\left\lvert X\right\rvert = m$ and $\left\lvert Y\right\rvert = n $, then $\left\lvert X \cup Y\right\rvert = m + n $
			\end{theorem}
			\subsection{Applying the principle to Venn Diagrams}
				\begin{example}
					In a group of 50 learners, 25 play mastermind, 30 play basketball, and 10 play both.
					\begin{enumerate}[label=(\alph*)]
						\item How many learners play Mastermind or basketball, (or both)?
						\item How many students do not play either Mastermind or basketball?
					\end{enumerate}
					$U$ is all the learners, $M$ is those who play Mastermind, and $B$ is those who play basketball.
					\begin{alignat*}{4}
						\left\lvert U\right\rvert &= 50 \qquad & \left\lvert M\right\rvert &= 25 \qquad & \left\lvert B\right\rvert &= 30 \qquad & \left\lvert M \cap B \right\rvert = 10
					\end{alignat*}
					\begin{center}
						\begin{venntwo}[showframe=true, radius=2.4cm, overlap=1.2cm, vgap=1cm, labelA={$M$}, labelAB={$10$}, labelOnlyA={$25 - 10 = 15$}, labelOnlyB={$30 - 10 = 20$}]
							\setpostvennhook
							{
								\node[above] at (current bounding box.south) {$50 - 15 - 20 - 10 = 5$};
							}
						\end{venntwo}
					\end{center}
					\pagebreak
					\begin{enumerate}
						\item $\left\lvert M \cup B\right\rvert = 15 + 10 + 20 = 45$.\\
							Also, by Inclusion Exclusion,
							\begin{align*}
								\left\lvert M \cup B\right\rvert &= \left\lvert M\right\rvert + \left\lvert B\right\rvert - \left\lvert M \cap B\right\rvert\\
								&= 25 + 30 - 10\\
								&= 45
							\end{align*}
						\item $\left\lvert (M \cup B)'\right\rvert = 50 - 45 = 5$
					\end{enumerate}
				\end{example}
				\begin{example}
					A questionnaire filled in by 100 subscribers to Blue Scalpel Medical Insurance who submitted no claims during 2009 reveals that 45 jog regularly, 30 do aerobics regularly, 20 cycle regularly, 6 jog and do aerobics, 1 jogs and cycles, 5 do aerobics and cycle, and 1 jogs, cycles and does aerobics.
					\begin{enumerate}[label=(\alph*)]
						\item How many of these healthy people do not participate regularly in any of the three activities?
						\item How many only jog?
					\end{enumerate}
					$U$ is the subscribers, $J$ is those who jog, $A$ is those who do aerobics, and $C$ is those who cycle.
					\begin{alignat*}{4}
						\left\lvert U\right\rvert &= 100 \qquad & \left\lvert J\right\rvert &= 45 \qquad & \left\lvert A\right\rvert &= 30 \qquad & \left\lvert C\right\rvert &= 20\\
						\left\lvert J \cap A\right\rvert &= 6 & \left\lvert J \cap C\right\rvert &= 1 & \left\lvert A \cap C\right\rvert &= 5 & \left\lvert J \cap A \cap C\right\rvert &= 1     
					\end{alignat*}
					\begin{center}
						\begin{vennthree}[showframe=true, radius=3.5cm, overlap=2.3cm, vgap=1cm, labelA={$J$}, labelB={$A$}, labelABC={$1$}, labelOnlyAC={$1 - 1 = 0$}, labelOnlyBC={$5-1 = 4$}, labelOnlyAB={$6 - 1 = 5$}, labelOnlyA={$45 - 5 - 1 - 0 = 39$}, labelOnlyB={$30 - 5 - 1 - 4 = 20$}, labelOnlyC={$20 - 0 - 1 - 4 = 15$}]
							\setpostvennhook
							{
								\node[above] at (current bounding box.south) {$100 - 39 - 20 - 15 - 5 - 0 - 4 - 1 = 16$};
							}
						\end{vennthree}
					\end{center}
					\begin{enumerate}[label=(\alph*)]
						\item This would be the value of the people who don't appear in any of the circles, which is $16$.
						\item This would be the value inside the circle $J$, which is $39$.
					\end{enumerate}
				\end{example}
				\pagebreak
				\begin{exercise}{Self Assessment Exercise \thechapter.10}
					\begin{enumerate}
						\item \question{Of $1000$ first year students, $700$ take Mathematics, $400$ take Computer Science, and $800$ take Mathematics or Computer Science.}\\
							$U$ is the first year students, $M$ is those who take Mathematics, and $C$ is those who take Computer Science.
							\begin{alignat*}{5}
								\left\lvert U\right\rvert &= 1000 \qquad & \left\lvert M\right\rvert &= 700 \qquad & \left\lvert C\right\rvert &= 400 \qquad & \left\lvert M \cup C\right\rvert &= 800 \qquad & \left\lvert M \cap C\right\rvert &= x
							\end{alignat*}
							\begin{center}
								\begin{venntwo}[showframe=true, radius=2.4cm, overlap=1.2cm, vgap=1cm, labelA={$M$}, labelB={$C$}, labelAB=$x$, labelOnlyA={$700 - x$}, labelOnlyB={$400 - x$}]
									\setpostvennhook
									{
										\node[above] at (current bounding box.south) {$1000 - 800 = 200$};
									}
								\end{venntwo}
							\end{center}
							\begin{alignat*}{2}
								&\qquad &800 &= (700 - x) + x + (400 - x)\\
								& \Rightarrow &800 &= 700 + 400 - x\\
								& \Rightarrow &- 300 &= -x\\
								& \Rightarrow & x &= 300
							\end{alignat*}
							\begin{enumerate}
								\item \question{How many students take Mathematics and Computer Science?}\\
								This would be $x$, which is $300$.
								\item \question{How many students take Mathematics, but not Computer Science?}
									\begin{align*}
										400 - x &= 400 - 300\\
										&= 100
									\end{align*}
								\item \question{How many students do not take either of the two subjects?}\\
								The number occurring outside the circles, so $200$.
							\end{enumerate}
						\pagebreak
						\item \question{A builder has a team of 64 construction workers. 45 can use at least one of the three equipment types. 22 can operate cranes, 26 can operate backhoes, 4 can operate cranes and bulldozers, 6 can operate backhoes and bulldozers, 8 can operate cranes and backhoes, and 1 can operate all three kinds of machinery. How many can operate bulldozers?}\\
							$U$ is the workers, $C$ is the workers who can operate cranes, $B$ is the workers who can operate backhoes, and $D$ is the workers who can operate bulldozers. 
							\begin{alignat*}{4}
								\left\lvert U\right\rvert &= 64 \qquad & \left\lvert C\right\rvert &= 22 \qquad & \left\lvert B \right\rvert &= 26 \qquad & \left\lvert D\right\rvert &= x\\
								\left\lvert C \cap D\right\rvert &= 4 \qquad & \left\lvert B \cap D \right\rvert &= 6 \qquad & \left\lvert C \cap B \right\rvert &= 8 \qquad & \left\lvert C \cap B \cap D \right\rvert &= 1\\
								\left\lvert C \cup B \cup D\right\rvert &= 45 & & & & & &
							\end{alignat*}
							\begin{center}
								\begin{vennthree}[showframe=true, radius=3.5cm, overlap=2.3cm, vgap=0.8cm, labelA={$C$}, labelB={$B$}, labelC={$D$}, labelABC={$1$}, labelOnlyAB={$8 - 1 = 7$}, labelOnlyC={$x - 3 - 1 - 5 = x - 9$}, labelOnlyAC={$4 - 1 = 3$}, labelOnlyBC={$6 - 1 = 5$}, labelOnlyA={$22 - 7 - 1 - 3 = 11$}, labelOnlyB={$26 - 7 - 1 - 5 = 13$}]
									\setpostvennhook
									{
										\node[above] at (current bounding box.south) {$64 - 45 = 19$};
									}
								\end{vennthree}
							\end{center}
							\begin{alignat*}{2}
								& &45 &= 11 + 13 + (x - 9) + 7 + 3 + 5 + 1\\
								& \Rightarrow \quad & 45 &= 40 + x - 9\\
								& \Rightarrow \quad & x &= 45 - 31\\
								& \Rightarrow \quad & x &= 14
							\end{alignat*}
							The number of people who can only operate bulldozers is $x - 9 = 14 - 9 = 5$.\\
							The number of people who can operate bulldozers is therefore $5 + 3 + 1 + 5 = 14$.
						\item \question{A software company employs 22 software engineers. All of them can use at least one of the three methods. $17$ of them can use a formal method (FM), $9$ can use Unified Modelling Language (UML), and $9$ can use entity-relationship diagrams (ER). $5$ engineers can use both an FM and UML, $4$ both an FM and ER diagrams, and $7$ both UML and ER diagrams.}
						\begin{alignat*}{4}
							\left\lvert U\right\rvert &= 22 \qquad & \left\lvert \mathrm{FM}\right\rvert&= 17 \qquad & \left\lvert \mathrm{UML}\right\rvert &= 9 \qquad & \left\lvert \mathrm{ER}\right\rvert &= 9 \\
							\left\lvert \mathrm{FM} \cap \mathrm{UML}\right\rvert &= 5 \qquad & \left\lvert \mathrm{FM} \cap \mathrm{ER}\right\rvert &= 4 \qquad & \left\lvert \mathrm{UML} \cap \mathrm{ER}\right\rvert &= 7 \qquad & \left\lvert \mathrm{FM} \cap \mathrm{UML} \cap \mathrm{ER}\right\rvert &= x  
						\end{alignat*}
						\begin{center}
							\begin{vennthree}[showframe=true, radius=3.5cm, overlap=2.3cm, labelA={FM}, labelB={UML}, labelC={ER}, labelABC={$x$}, labelOnlyAB={$5 - x$}, labelOnlyAC={$4 - x$}, labelOnlyBC={$7 - x$}, labelOnlyA={$x + 8$}, labelOnlyB={$x - 3$}, labelOnlyC={$x - 2$}]
							\end{vennthree}
						\end{center}
						\begin{description}
							\item[For Only FM:] $
								\begin{aligned}[t]
									17 - (5 - x) - x - (4 - x) &= 17 - 5 + x - x - 4 + x\\
									&= x + 8
								\end{aligned} $
							\item[For Only UML:] $
								\begin{aligned}[t]
									9 - (5 - x) - x - (7 - x) &= 9 - 5 + x - x - 7 + x\\
									&= x - 3
								\end{aligned} $
							\item[For Only ER:] $
								\begin{aligned}[t]
									9 - (4 - x) - x - (7 - x) &= 9 - 4 + x - x - 7 + x\\
									&= x - 2
								\end{aligned} $
						\end{description}
						\begin{alignat*}{2}
							& & 22 &= (x + 8) + (x - 3) + (x - 2) + (5 - x) + (4 - x) + (7 - x) + x\\
							& \Rightarrow \quad & 22 &= (x + x + x - x - x - x + x) + (8 - 3 - 2 + 5 + 4 + 7)\\
							& \Rightarrow \quad & 22 &= x + 19\\
							& \Rightarrow \quad & x &= 22 - 19\\
							& \Rightarrow \quad & x &= 3
						\end{alignat*}
						\begin{enumerate}
							\item \question{How many engineers can use all three diagrams?}\\
								As shown above, $3$ engineers.
							\item \question{How many engineers can use UML only?}
								\begin{align*}
									x - 3 &= 3 - 3\\
									&= 0
								\end{align*}
						\end{enumerate}
					\end{enumerate}
				\end{exercise}
		\section{Proofs on Specific Sets}
			To prove that two sets are equal, prove that each member of the left-hand side belongs to the right-hand side, and vice versa.
			\begin{sidenote}{Any variable can be used for a set description}
				Whether the variable is $x$ or $z$ does not change the members of the set.
			\end{sidenote}
			\begin{example}
				Prove that $\{w \in \mathbb{R} \mid w^{2} - 3w + 2 < 0\} = \{z \in \mathbb{R} \mid 1 < z < 2\}$
				\begin{proof}
					Let $x \in \{w \in \mathbb{R} \mid w^{2} - 3w + 2 < 0\}$
					\begin{tabbing}
						$\qquad$ \= $x \in \{w \in \mathbb{R} \mid w^{2} - 3w + 2 < 0\}$\\
						iff      \> $x \in \mathbb{R}$ and $x^{2} - 3x + 2 < 0$\\
						iff      \> $x \in \mathbb{R}$ and $(x - 2)(x - 1) < 0$\\
						iff      \> $x \in \mathbb{R}$ and either \= $(x - 2) < 0$ and $(x - 1) > 0$ (minus times a plus is a minus) or\\
						         \>                               \> $(x - 2) > 0$ and $(x - 1) < 0$ (plus times a minus is a minus)\\
						iff      \> $x \in \mathbb{R}$ and either \> $(x < 2$ and $x > 1)$ or $(x > 2$ and $x < 1)$\\
										 \>                               \> (there are no real numbers that meet the second option)\\
						iff      \> $x \in \mathbb{R}$ and $(x < 2$ and $x > 1)$\\
						iff      \> $x \in \mathbb{R}$ and $1 < x < 2$\\
						iff      \> $x \in \{x \in \mathbb{R} \mid 1 < x < 2\}$\\
						iff      \> $x \in \{z \in \mathbb{R} \mid 1 < z < 2\}$
					\end{tabbing}
				\end{proof}
			\end{example}
			\begin{sidenote}{Using Or in Proofs}
				Note that if there is an ``or'' that is connecting the statements, then the statement is true if \emph{either} of the statements is true.
			\end{sidenote}
			\pagebreak
			\begin{exercise}{Self Assessment Exercise \thechapter.11}
				\question{Prove the following}
				\begin{enumerate}
					\item \question{$\{y \in \mathbb{Z}^{+} \mid y$ is an even prime number$\} = \{u \in \mathbb{Z}^{+} \mid u^{2} = 4\}$}
						\begin{proof}
							Let $x \in \{y \in \mathbb{Z}^{+} \mid y$ is an even prime number$\}$.
							\begin{tabbing}
								$\qquad$ \= $x \in \{y \in \mathbb{Z}^{+} \mid y$ is an even prime number$\}$\\
								iff      \> $x \in \mathbb{Z}^{+}$ and $x$ is an even prime number\\
								iff      \> $x \in \mathbb{Z}^{+}$ and $x = 2$\\
								iff      \> $x \in \mathbb{Z}^{+}$ and $x^{2} = 4$\\
								iff      \> $x \in \{x \in \mathbb{Z}^{+} \mid x^{2} = 4\}$\\
								iff      \> $x \in \{u \in \mathbb{Z}^{+} \mid u^{2} = 4\}$
							\end{tabbing}
						\end{proof}
					\item \question{$\mathcal{P}\bigl(\{0, 1\}\bigr) = \{\emptyset\} \cup \bigl\{\{0\}\bigr\} \cup \bigl\{\{1\}\bigr\} \cup \bigl\{\{0, 1\}\bigr\}$}
						\begin{proof}
							Let $X \in \mathcal{P}\bigl(\{0, 1\}\bigr)$.
							\begin{tabbing}
								$\qquad$ \= $X \in \mathcal{P}\bigl(\{0, 1\}\bigr)$\\
								iff \> $X \in \bigl\{\emptyset, \{0\}, \{1\}, \{0, 1\}\bigr\}$\\
								iff \> $X = \emptyset$ or $X = \{0\}$ or $x = \{1\}$ or $X = \{0, 1\}$\\
								iff \> $X \in \{\emptyset\}$ or $X \in \bigl\{\{0\}\bigr\}$ or $X \in \bigl\{\{1\}\bigr\}$ or $X \in \bigl\{\{0, 1\}\bigr\}$\\
								iff \> $X \in \{\emptyset\} \cup \bigl\{\{0\}\bigr\} \cup \bigl\{\{1\}\bigr\} \cup \bigl\{\{0, 1\}\bigr\}$
							\end{tabbing}
						\end{proof}
					\item \question{$\{x \in \mathbb{R} \mid x^{2} + 6x + 5 < 0\} = \{x \in \mathbb{R} \mid -5 < x < -1\}$}
						\begin{proof}
							Let $y \in \{x \in \mathbb{R} \mid x^{2} + 6x + 5 < 0\}$
							\begin{tabbing}
								$\qquad$ \= $y \in \{x \in \mathbb{R} \mid x^{2} + 6x + 5 < 0\}$\\
								iff \> $y \in \mathbb{R}$ and $y^{2} + 6y + 5< 0$\\
								iff \> $y \in \mathbb{R}$ and $(y + 5)(y + 1) < 0$\\
								iff \> $y \in \mathbb{R}$ and either \= $(y + 5) < 0$ and $(y + 1) > 0$ (minus times a plus is a minus) or\\
								\> \> $(y + 5) > 0$ and $(y + 1) < 0$ (plus times a minus is a minus)\\
								iff \> $y \in \mathbb{R}$ and either $(y < -5$ and $y > -1)$ or $(y > -5$ and $y < -1)$\\
								\> \> (no real numbers meet the first statement)\\
								iff \> $y \in \mathbb{R}$ and $(y > -5$ and $y < -1)$\\
								iff \> $y \in \mathbb{R}$ and $-5 < y < -1$\\
								iff \> $y \in \{x \in \mathbb{R} \mid -5 < x < -1\}$
							\end{tabbing}
						\end{proof}
						\pagebreak
					\item \question{$\{x \in \mathbb{Z} \mid x^{2} - 5x + 4 < 0\} = \{x \in \mathbb{Z}^{+} \mid x$ is a prime factor of $6\}$}
						\begin{proof}
							Let $w \in \{x \in \mathbb{Z} \mid x^{2} - 5x + 4 < 0\}$
							\begin{tabbing}
								$\qquad$ \= $w \in \{x \in \mathbb{Z} \mid x^{2} - 5x + 4 < 0\}$\\
								iff \> $w \in \mathbb{Z}$ and $w^{2} - 5w + 4 < 0$\\
								iff \> $w \in \mathbb{Z}$ and $(w - 4)(w - 1) < 0$\\
								iff \> $w \in \mathbb{Z}$ and either \= $(w - 4) < 0$ and $(w - 1) > 0$ (minus times a plus is a minus) or\\
								\> \> $(w - 4) > 0$ and $(w - 1) < 0$ (plus times a minus is a minus)\\
								iff \> $w \in \mathbb{Z}$ and either $(w < 4$ and $w > 1)$ or $(w > 4$ and $w < 1)$\\
								\> \> (no integers meet the second statement)\\
								iff \> $w \in \mathbb{Z}$ and $(w < 4$ and $w > 1)$\\
								iff \> $w \in \mathbb{Z}^{+}$ and \=$(1 < w < 4)$\\
								\> \> ($\mathbb{Z}^{+}$ as all the numbers are positive)\\
								iff \> $w \in \mathbb{Z}^{+}$ and $w \in \{2, 3\}$\\
								iff \> $w \in \{x \in \mathbb{Z}^{+} \mid x \in \{2, 3\}\}$\\
								iff \> $w \in \{x \in \mathbb{Z}^{+} \mid x$ is a prime factor of $6\}$
							\end{tabbing}
						\end{proof}
					\item \question{$\{x \in \mathbb{R} \mid x^{2} + x - 2 > 0\} = \{x \in \mathbb{R} \mid x < -2$ or $x > 1\}$}
						\begin{proof}
							Let $z \in \{x \in \mathbb{R} \mid x^{2} + x - 2 > 0\}$
							\begin{tabbing}
								$\qquad$ \= $z \in \{x \in \mathbb{R} \mid x^{2} + x - 2 > 0\}$\\
								iff \> $z \in \mathbb{R}$ and $z^{2} + z - 2 > 0$\\
								iff \> $z \in \mathbb{R}$ and $(z + 2)(z - 1) > 0$\\
								iff \> $z \in \mathbb{R}$ and either \=$(z + 2) < 0$ and $(z - 1) < 0$ (minus times a minus is a plus) or\\
								\> \> $(z + 2)> 0$ and $(z - 1) > 0$ (plus times a plus is a plus)\\
								iff \> $z \in \mathbb{R}$ and either $(z < -2$ and $z < 1)$ or $(z > -2$ and $z > 1)$\\
								iff \> $z \in \mathbb{R}$ and either $(z < -2)$ or $(z > 1)$\\
								iff \> $z \in \{x \in \mathbb{R} \mid x < -2$ or $x > 1\}$
							\end{tabbing}
						\end{proof}
				\end{enumerate}
			\end{exercise}
			\pagebreak
			\begin{exercise}{Self Assessment Exercise \thechapter.12}
				\begin{enumerate}
					\item \question{Determine whether, for $V$, $W$, $Z \subseteq U$, if $V \subseteq W$, then $V \cup Z \subseteq W \cup Z$ and $V \cap Z \subseteq W \cap Z$. Provide either a proof or a counterexample.}\\
						Both statements are true.
						\begin{indentparagraph}
							\begin{proof}
								Suppose $V \subseteq W$
								\begin{tabbing}
									Let $\quad$ \= $x \in V \cup Z$\\
									Then \> $x \in V$ or $x \in Z$\\
									\> $\qquad$ (If $x \in V$, $x \in W$, as $V \subseteq W$)\\
									i.e. \> $x \in W$ or $x \in Z$\\
									i.e. \> $x \in W \cup Z$\\
									$\therefore$ \> $V \cup Z \subseteq W \cup Z$
								\end{tabbing}
							\end{proof}
							\begin{proof}
								Suppose $V \subseteq W$ 
								\begin{tabbing}
									Let $\quad$ \= $x \in V \cap Z$\\
									Then \> $x \in V$ and $x \in Z$\\
									\> $\qquad$ (If $x \in V$, $x \in W$, as $V \subseteq W$)\\
									i.e. \> $x \in W$ and $x \in Z$\\
									i.e. \> $x \in W \cap Z$\\
									$\therefore$ \> $V \cap Z \subseteq W \cap Z$
								\end{tabbing}
							\end{proof}
						\end{indentparagraph}
					\item \question{Is it the case that, for all subsets $X$, $Y$, $W \subseteq U$, if $X = Y$ and $Y = W$, then $X = W$, and if $X \subset Y$ and $Y \subset W$, then $X \subset W$?}\\
						Both statements are true.
						\begin{indentparagraph}
							\begin{proof}
								Suppose $X = Y$ and $Y = W$.
								\begin{tabbing}
									Let $\quad$ \= $x \in X$\\
									Then \> $x \in Y$, as $X = Y$\\
									Then \> $x \in W$, as $Y = W$\\
									$\therefore$ \> $X = W$
								\end{tabbing}
							\end{proof}
							\begin{proof}
								Suppose $X \subset Y$ and $Y \subset W$.
								\begin{tabbing}
									Let $\quad$ \= $x \in X$\\
									Then \> $x \in Y$, as $X \subset Y$\\
									Then \> $x \in W$, as $Y \subset W$\\
									$\therefore$ \> $X \subseteq W$\\
									\> $Y$ has at least one element not in $X$, as $X \subset Y$\\
									\> $W$ has at least one element not in $Y$, as $Y \subset W$\\
									So \> $W$ has at least two elements not in $X$\\
									i.e. $X \neq W$\\
									so $X \subset W$
								\end{tabbing}
							\end{proof}
						\end{indentparagraph}
					\pagebreak
					\item \question{Is it the case that, for all subsets $X$ of $U$, $X \cup \emptyset = X$? Justify your answer.}\\
						Yes.
						\begin{proof}
							$ $
							\begin{tabbing}
								Let \quad \=$x \in X$\\
								Then \> $x \in X$ or $x \in \emptyset$\\
								i.e. \> $x \in X \cup \emptyset$\\
								$\therefore$ \> $X \subseteq X \cup \emptyset$
							\end{tabbing}
							\begin{tabbing}
								Let \quad \=$x \in X \cup \emptyset$\\
								Then \> $x \in X$ or $x \in \emptyset$\\
								i.e. \> $x \in X$\\
								\> ($x$ cannot be in the empty set)\\
								$\therefore$ \> $X \cup \emptyset \subseteq X$
							\end{tabbing}
							As $(X \subseteq X \cup \emptyset)$ and $(X \cup \emptyset \subseteq X)$, $X \cup \emptyset = X$
						\end{proof}
					\item \question{Is it true that for all subsets $V$ and $W$ of $U$, $V \cap W = \emptyset$ iff $V = \emptyset$ or $W = \emptyset$?}\\
						No.
						\begin{proof}
							$ $
							\begin{enumerate}[label=(\roman*)]
								\item \question{If $V \cap W = \emptyset$ then $V = \emptyset$ or $W = \emptyset$}\\
									This claim is false.
									\begin{subproof}[Counterexample]
										Let $V = \{3, 4\}$ and $W = \{5, 6\}$.
										\begin{align*}
											V \cap W &= \{3, 4\} \cap \{5, 6\}\\
											&= \emptyset
										\end{align*}
										$V \cap W = \emptyset$ but $V \neq \emptyset$ and $W \neq \emptyset$.
									\end{subproof}
								\item \question{If $V = \emptyset$ or $W = \emptyset$, then $V \cap W = \emptyset$}\\
									This claim is true.
									\begin{subproof}[Subproof]
										Let $V = \emptyset$ and $W$ be some non-empty set.
										\begin{align*}
											V \cap W &= \emptyset \cap W\\
											&= \emptyset
										\end{align*}
										$\therefore$ if $V = \emptyset$, $V \cap W = \emptyset$\\
										Let $W = \emptyset$ and $V$ be some non-empty set.
										\begin{align*}
											V \cap W &= V \cap \emptyset\\
											&= \emptyset
										\end{align*}
										$\therefore$ if $W = \emptyset$, $V \cap W = \emptyset$\\
										$\therefore V \cap W = \emptyset$ if either $V = \emptyset$ or $W = \emptyset$
									\end{subproof}
							\end{enumerate}
							As the first claim is false, it is not the case that $V \cap W = \emptyset$ iff $V = \emptyset$ or $W = \emptyset$.
						\end{proof}
					\pagebreak
					\item \question{Is it the case that, for every subset $X$ of $U$ there exists a subset $Y$ of $U$ such that $X \cup Y = \emptyset$? Justify your answer.}\\
						No.
						\begin{proof}[Counterexample]
							Let $X = \{1,\}$ and $U$ = \{1, 2\}.\\
							The possible subsets of $U$ are $\emptyset$ or \{1\} or \{2\} or \{1, 2\}.
							\begin{align*}
								X \cup \emptyset &= \{1\} \cup \emptyset\\
								&= \{1\}\\
								X \cup \{1\} &= \{1\} \cup \{1\}\\
								&= \{1\}\\
								X \cup \{2\} &= \{1\} \cup \{2\}\\
								&= \{1, 2\}\\
								X \cup \{1, 2\} &= \{1\} \cup \{1, 2\}\\
								&= \{1, 2\}
							\end{align*}
							From the above, there is no set $Y$ such that $X \cup Y = \emptyset$.
						\end{proof}
					\item \question{Is it the case that, for every subset $X$ of $U$, there is some subset $Y$ such that $X \cap Y = U$? Justify your answer.}\\
						No.
						\begin{proof}[Counterexample]
							Let $X = \{1,\}$ and $U$ = \{1, 2\}.\\
							The possible subsets of $U$ are $\emptyset$ or \{1\} or \{2\} or \{1, 2\}.
							\begin{align*}
								X \cap \emptyset &= \{1\} \cap \emptyset\\
								&= \emptyset\\
								X \cap \{1\} &= \{1\} \cap \{1\}\\
								&= \{1\}\\
								X \cap \{2\} &= \{1\} \cap \{2\}\\
								&= \emptyset\\
								X \cap \{1, 2\} &= \{1\} \cap \{1, 2\}\\
								&= \{1\}
							\end{align*}
							From the above, there is no set $Y$ such that $X \cap Y = U$.
						\end{proof}
					\item \question{Using ``if and only if'' statements, prove the following:}
						\begin{enumerate}[label=(\alph*)]
							\item \question{$X + Y = Y + X$ for all $X$, $Y \subseteq U$.}
								\begin{proof}
									Let $x \in X + Y$.
									\begin{tabbing}
										\qquad \= $x \in X + Y$\\
										iff \> ($x \in X$ or $x \in Y$) and $x \notin X \cap Y$\\
										iff \> ($x \in Y$ or $x \in X$) and $x \notin X \cap Y$\\
										iff \> $x \in Y + X$\\
										$\therefore$ \> $X + Y = Y + X$
									\end{tabbing}
								\end{proof}
							\pagebreak
							\item \question{$X \cap (Y + Z) = (X \cap Y) + (X \cap Z)$ for all $X$, $Y$, $Z \subseteq U$.}
								\begin{proof}
									Let $x \in X \cap (Y + Z)$.
									\begin{tabbing}
										\qquad \= $x \in X \cap (Y + Z)$\\
										iff \> $x \in X$ and $x \in (Y + Z)$\\
										iff \> $x \in X$ and ($x \in Y$ or $x \in Z$ and $x \notin Y \cap Z$)\\
										iff \> $(x \in X$ and $x \in Y$) or $(x \in X$ and $x \in Z)$ and $x \notin (Y \cap Z)$\\
										iff \> $x \in (X \cap Y)$ or $x \in (X \cap Z)$ and $x \notin (Y \cap Z)$\\
										iff \> $x \in (X \cap Y) + (Y \cap Z)$\\
										$\therefore$ \> $X \cap (Y + Z) = (X \cap Y) + (Y \cap Z)$
									\end{tabbing}
								\end{proof}
						\end{enumerate}
				\end{enumerate}
			\end{exercise}
	\rulechapterend
\end{document}
