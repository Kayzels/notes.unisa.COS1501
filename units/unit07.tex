\providecommand{\main}{..}
\documentclass[\main/notes.tex]{subfiles}

\begin{document}
	\ifSubfilesClassLoaded{\setcounter{chapter}{6}}{}
	\chapter{More About Functions}
		\section{The Range of a Function}
			\begin{definition}{Range of a Function}
				Given a function $f: A \rightarrow B$, the \concept{range} or \concept{image set} of $f$ is the subset
				\begin{align*}
					\{f(x) \mid x \in A\}
				\end{align*} 
				of $B$, written $\mathrm{ran}(f)$ or $f[A]$.\\
				In other words, it is a subset of $B$ where and element $b$ of $B$ can be reached by calling the function with a specific element $a$ of $A$.
			\end{definition}
			\begin{example}[width=0.9\textwidth]
				Let $A = \{a, b, c\}$ and $B = \{1, 2, 3\}$. Let a function $f: A \rightarrow B$ be defined as 
				\begin{alignat*}{3}
					f(a) &= 1 \qquad & f(b) &= 2 \qquad & f(c) &= 1
				\end{alignat*}
				Then $\mathrm{ran}(f) = \{1, 2\}$. 
			\end{example}
			\subsection{Determining the Range of a Function}
				In order to find the range, you follow these steps:
				\begin{enumerate}
					\item Write the definition of the function $\bigl(f(x)\bigr)$, and the domain.
					\item Substitute the definition with the value of $f(x)$
					\item Calculate the first coordinate in terms of the second.
					\item Substitute the second coordinate and that formula for it into the definition.
					\item Simplify.
				\end{enumerate}
				\pagebreak
				Easier to show with an example:
				\begin{example}[width=0.8\textwidth]
					Let $g: \mathbb{Z} \rightarrow \mathbb{Z}$ be defined by $y = 2x$.
					\begin{align*}
						\mathrm{ran}(g) &= \{g(x) \mid x \in \mathbb{Z}\}\tag*{$(1)$}\\
						&= \{2x \mid x \in \mathbb{Z}\}\tag*{$(2)$}\\
					\end{align*}
					Calculate $x$ in terms of $y$ (Step $3$):
					\begin{alignat*}{2}
						& \qquad &y &= 2x\\
						& \Rightarrow \quad &\frac{y}{2} &= x\\
						& \Rightarrow \quad &x &= \frac{y}{2}\\
					\end{alignat*}
					\begin{align*}
						\mathrm{ran}(g) &= \{y \mid \frac{y}{2} \in \mathbb{Z}\}\tag*{$(4)$}\\
						&= \{y \mid \frac{y}{2} \text{ is an integer}\}\tag*{$(5)$}\\
						&= \{y \mid y \text{ is an even integer}\}
					\end{align*}
				\end{example}
		\section[Surjectivity]{Surjectivity (MAPPING)}
			\begin{definition}{Surjectivity}
				Given a function $f: A \rightarrow B$, the function $f$ would be \concept{surjective} iff the \emph{range} of $f$ is equal to the codomain of $f$.\\
				As $B$ is the codomain of $f$ above, that would mean that $\mathrm{ran}(f)$ (also written $f[A]$) is equal to $B$.
			\end{definition}
			\begin{example}
				Let $A = \{1, 2, 3\}$. Let $B = \{4, 5, 6\}$.
				\begin{description}
					\item[Surjective Function] For a surjective function, every element of $A$ needs to be present, and every element of $B$. So an example of a function $h: A \rightarrow B$ would be:
						\begin{align*}
							h = \Bigl\{(1, 6), (2, 4), (3, 5)\Bigr\}
						\end{align*}
						$\mathrm{ran}(h) = \{4, 5, 6\} = B$.
					\item[Non-Surjective Function] For a function, every element of $A$ needs to be present. For it to not be surjective, that means that at least one element of $B$ is not in the range of the function. An example function $h: A \rightarrow B$ would be:
						\begin{align*}
							h = \Bigl\{(1, 4), (2, 4), (3, 5)\Bigr\}
						\end{align*}
						$\mathrm{ran}(h) = \{4, 5\} \neq B$.
				\end{description}
			\end{example}
			\begin{exercise}{Self Assessment Exercise \thechapter.4}
				\begin{questions}
					\item In each of the following cases, write down the possible surjective functions from $X$ to $Y$.
						\begin{questions}
							\item $X = \{a, b\}$ and $Y = \{c\}$.\\
								\begin{answer}
									For a surjective function, make sure each element of $X$ appears as a first coordinate, and every element of $Y$ is used.
									\begin{align*}
										f_{1} &= \bigl\{(a, c), (b, c)\bigr\}
									\end{align*}
								\end{answer}
							\item $X = \{a, b\}$ and $Y = \{c, d\}$.
								\begin{answer}
									\begin{align*}
										f_{1} &= \bigl\{(a, c), (b, d)\bigr\}\\
										f_{2} &= \bigl\{(a, d), (b, c)\bigr\}
									\end{align*}
								\end{answer}
							\item $X = \{a, b\}$ and $Y = \{c, d, e\}$.\\
								\begin{answer}
									There are no possible surjective functions, as there are more $y$ elements than $x$ elements. Either an $x$ element appears twice, in which case it is not a function, or a $y$ element doesn't appear, in which case it is not surjective.
								\end{answer}
						\end{questions}
					\item Let $f: \mathbb{Z} \rightarrow \mathbb{Z}$ be defined by $f(x) = x + 1$.
						\begin{questions}
							\item Determine $f[\mathbb{Z}]$ (or $\mathrm{ran}(f)$).
								\begin{answer}
									\begin{align*}
										f[\mathbb{Z}] &= \{f(x) \mid x \in \mathbb{Z}\}\\
										&= \{x + 1 \mid x \in \mathbb{Z}\} \tag*{$(y = x + 1 \Rightarrow x = y - 1)$}\\
										&= \{y \mid y - 1 \in \mathbb{Z}\}\\
										&= \{y \mid y - 1 \text{ is an integer}\}\\
										&= \mathbb{Z}
									\end{align*}
								\end{answer}
							\item Is $f$ surjective? If $f$ is not surjective, provide a counterexample to show why it is not surjective.\\
								{\answer $f$ is surjective, as $f[\mathbb{Z}] = \mathbb{Z}$.}
						\end{questions}
					\item Let $f: \mathbb{Z} \rightarrow \mathbb{Z}$ be defined by $f(x) = 4x + 8$.
						\begin{questions}
							\item Determine $f[\mathbb{Z}]$ (or $\mathrm{ran}(f)$).
								\begin{answer}
									\begin{align*}
										f[\mathbb{Z}] &= \{f(x) \mid x \in \mathbb{Z}\}\\
										&= \{4x + 8 \mid x \in \mathbb{Z}\} \tag*{$(y = 4x + 8 \Rightarrow x = \frac{y}{4} - 2)$}\\
										&= \{y \mid \frac{y}{4} - 2 \in \mathbb{Z}\}\\
										&= \{y \mid \frac{y}{4} \text{ is an integer}\}\\
										&= \{y \mid y \text{ is an integer divisible by 4}\}
									\end{align*}
								\end{answer}
								\item Is $f$ surjective? If $f$ is not surjective, provide a counterexample to show why it is not surjective.\\
									\begin{answer}
										$f$ is not surjective, as the range of $f$ is not equal to the codomain.
										\begin{proof}[Counterexample]
											$3 \in \mathbb{Z}$, which is the codomain, but there is no $x \in \mathbb{Z}$ such that $4x + 8 = 3$.
										\end{proof}
									\end{answer}
						\end{questions}
				\end{questions}
			\end{exercise}
		\pagebreak
		\section[Injectivity]{Injectivity (ONE TO ONE)}
			\begin{definition}{Injectivity}
				A function $f: A \rightarrow B$ is \concept{injective} iff $f$ has the property that whenever $f(a_{1}) = f(a_{2})$, then $a_{1} = a_{2}$.\\
				In other words, every unique first coordinate is related to a unique second coordinate.
				\begin{description}
					\item[Another definition (contrapositive)] A function $f: A \rightarrow B$ is \concept{injective} iff $f$ has the property that whenever $a_{1} \neq a_{2}$, $f(a_{1}) \neq f(a_{2})$.
				\end{description}
			\end{definition}
			\begin{example}
				Let $A = \{1, 2, 3\}$. Let $B = \{4, 5, 6, 7\}$.
				\begin{description}
					\item[Injective Function] For an injective function, every element of $A$ should be related to a different element of $B$. An example function $g: A \rightarrow B$ would be:
						\begin{align*}
							g = \Bigl\{(1, 5), (2, 7), (3, 6)\Bigr\}
						\end{align*}
					\item[Non-Injective Function] For a function to not be injective, two or more elements of $A$ should be related to the same element of $B$. An example function $g: A \rightarrow B$ would be:
						\begin{align*}
							g = \Bigl\{(1, 4), (2, 5), (3, 4)\Bigr\}
						\end{align*}
				\end{description}
			\end{example}
			\subsection{Determining Whether an Abstract Function is Injective}
				For functions defined on all elements of an infinite set such as $\mathbb{Z}$, use logic to prove the function is injective:
				\nopagebreak
				\begin{enumerate}[nosep]
					\item Assume that the function being applied to two different elements results in the same value.
					\item Apply the function to the values.
					\item Simplify using algebra.
				\end{enumerate}
				\nopagebreak
				\begin{example}
					\begin{description}
						\item[Prove Injectivity] Let $f: \mathbb{Z} \rightarrow \mathbb{Z}$ be defined by $y = 4x$.
							\begin{alignat*}{2}
								& \text{Assume } \quad &f(u) &= f(v)\tag*{$(1)$}\\
								& \text{Then } &4u &= 4v\tag*{$(2)$}\\
								& \text{i.e. } &u &= v\tag*{$(3)$}
							\end{alignat*}
						\item[Non-Injective Function]  Let $f: \mathbb{Z} \rightarrow \mathbb{Z}$ be defined by $y = x^{2}$.
							\begin{alignat*}{2}
								& \text{Assume } \quad &f(u) &= f(v)\\
								& \text{Then } &u^{2} &= v^{2}\\
								& & \pm{u} &= \pm{v}\\
								& & u &\neq v
							\end{alignat*}
							$u$ is not necessarily equal to $v$. $u$ could be $1$, and $v$ could be $-1$, and $f(u)$ would be equal to $f(v) = 1$. Therefore, $f$ is not injective.
					\end{description}
				\end{example}
				\begin{exercise}{Self Assessment Exercise \thechapter.5}
					\begin{questions}
						\item In each of the following cases, write down the injective functions from $X$ to $Y$.
							\begin{questions}
								\item $X = \{2, 4\}$ and $Y = \{1\}$\\
									\begin{answer}
										There is no possible injective function, as $Y$ has only one member, but $X$ has two members. Either one of the members of $X$ is excluded, in which case it is not a function, or the two members point to the same member of $Y$, in which case it is not injective.
									\end{answer}
								\item $X = \{2, 4\}$ and $Y = \{1, 3\}$
									\begin{answer}
										\begin{align*}
											f_{1} &= \bigl\{(2, 1), (4, 3)\bigr\}\\
											f_{2} &= \bigl\{(2, 3), (4, 1)\bigr\}
										\end{align*}
									\end{answer}
								\item $X = \{2, 4\}$ and $Y = \{1, 3, 5\}$
									\begin{answer}
										\begin{align*}
											f_{1} &= \bigl\{(2, 1), (4, 3)\bigr\}\\
											f_{2} &= \bigl\{(2, 3), (4, 1)\bigr\}\\
											f_{3} &= \bigl\{(2, 1), (4, 5)\bigr\}\\
											f_{4} &= \bigl\{(2, 3), (4, 5)\bigr\}\\
											f_{5} &= \bigl\{(2, 5), (4, 1)\bigr\}\\
											f_{6} &= \bigl\{(2, 5), (4, 3)\bigr\}
										\end{align*}
									\end{answer}
							\end{questions}
						\item Consider $h: \mathbb{Z} \rightarrow \mathbb{Z}$ defined by $h(x) = 2x - 5$. Determine whether $h$ is injective.\\
							\begin{answer}
								$h$ is injective.
								\begin{proof} $
									\begin{alignedat}[t]{2}
										& \text{Assume } \quad &h(u) &= h(v)\\
										& \text{Then } &2u - 5 &= 2v - 5\\
										& & 2u &= 2v\\
										& & u &= v
									\end{alignedat} $\\
									$\therefore h$ is injective, because when $h(u) = h(v)$, $u = v$.
								\end{proof}
							\end{answer}
					\end{questions}
				\end{exercise}
			\pagebreak
		\section{Composition of Functions}
				\begin{theorem}{The composition of two functions is also a function}
					For any two functions $f: A \rightarrow B$ and $g: B \rightarrow C$, the composition of the two functions\\
						$g \circ f: A \rightarrow C$ is also a function.
						\begin{proof}
							Let $f: A \rightarrow B$ and $g: B \rightarrow C$ be two functions.\\
							As $f$ is a function, for every $a \in A$, there is exactly one $b \in B$ such that $(a, b) \in f$.\\
							As $g$ is a function, for every $b \in B$, there is exactly one $c \in C$ such that $(b, c) \in g$.\\
							As there is exactly one pair from $a$ to $b$, and from $b$ to $c$, there is exactly one pair in the composite function from $a$ to $c$.
						\end{proof}
				\end{theorem}
				\begin{definition}{Composite Function}
					Given the functions $f: A \rightarrow B$ and $g: B \rightarrow C$, the \concept{composite function} $g \circ f: A \rightarrow C$ is defined by
					\centerline{$ \begin{aligned}[t]
						g \circ f: A \rightarrow C &= g \circ f(x)\\
						&= g\bigl(f(x)\bigr)
					\end{aligned} $}
				\end{definition}
			\begin{example}[width=0.6\textwidth]
				Let $f: \mathbb{Z} \rightarrow \mathbb{Z}$ be defined by $f(x) = 4x + 2$.\\
				Let $g: \mathbb{Z} \rightarrow \mathbb{Z}^{+}$ be defined by $f(x) = x^{2} + 1$.\\
				Then $g \circ f: \mathbb{Z} \rightarrow \mathbb{Z}^{+}$.
				\begin{align*}
					(g \circ f)(x) &= g\bigl(f(x)\bigr)\\
					&= g(4x + 2)\\
					&= (4x + 2)^{2} + 1\\
					&= \left(16x^{2} + 16x + 4\right) + 1\\
					&= 16x^{2} + 16x + 5
				\end{align*}
				$(g \circ f)(x)$ is called the \concept{image of $x$ under $g \circ f$}.
			\end{example}
			\begin{theorem}{The composition of two surjective functions is surjective}
				For any two surjective functions $f: A \rightarrow B$ and $g: B \rightarrow C$, the composition of the two functions\\
					$g \circ f: A \rightarrow C$ is also a surjective function.
					\begin{proof}
						Let $f: A \rightarrow B$ and $g: B \rightarrow C$ be two surjective functions.\\
						As $g$ is surjective, every $c \in C$ appears as a second coordinate in $g$.\\
						As $f$ is surjective, every $b \in B$ appears as a second coordinate in $f$.\\
						As $f$ is a function, every $a$ appears as a first coordinate in $f$.\\
						As $g$ is a function, every $b$ appears as a first coordinate in $g$.\\
						Therefore, every $a$ maps to every $b$ which maps to every $c$.\\
						So every $a$ maps to every $c$.\\
						So the composite function is surjective.
					\end{proof}
			\end{theorem}
			\begin{theorem}{The composition of two injective functions is injective}
				For any two injective functions $f: A \rightarrow B$ and $g: B \rightarrow C$, the composition of the two functions\\
					$g \circ f: A \rightarrow C$ is also an injective function.
					\begin{proof}
						Let $f: A \rightarrow B$ and $g: B \rightarrow C$ be two injective functions.\\
						As $f$ is injective, every $a \in A$ maps to a unique $b \in B$.\\
						As $g$ is injective, every $b \in B$ maps to a unique $c \in C$.\\
						As every $a$ maps to a unique $b$, and every $b$ maps to a unique $c$, in the composite function, every $a$ maps to a unique $c$.\\
						So the composite function is injective.
					\end{proof}
			\end{theorem}
			\begin{exercise}{Self Assessment Exercise \thechapter.9}
				\begin{questions}
					\item Determine $f \circ f$, $g \circ g$, $g \circ f$ and $f \circ g$ in each of the following cases.
						\begin{questions}
							\item $f: \mathbb{Z} \rightarrow \mathbb{Z}$ is defined by $f(x) = x + 1$ and $g: \mathbb{Z} \rightarrow \mathbb{Z}$ is defined by $g(x) = x - 1$.\\
								\begin{answer}
									All these composite functions are defined on $\mathbb{Z} \rightarrow \mathbb{Z}$.
									\begin{alignat*}{2}
										(f \circ f)(x) &= f\bigl(f(x)\bigr) \qquad & (g \circ g)(x) &= g\bigl(g(x)\bigr)\\
										&= f(x + 1) \qquad & &= g(x - 1)\\
										&= (x + 1) + 1 \qquad & &= (x - 1) - 1\\
										&= x + 2 \qquad & &= x - 2
									\end{alignat*}
									\begin{alignat*}{2}
										(g \circ f)(x) &= g\bigl(f(x)\bigr) \qquad & (f \circ g)(x) &= f\bigl(g(x)\bigr)\\
										&= g(x + 1) \qquad & &= f(x - 1)\\
										&= (x + 1) - 1 \qquad & &= (x - 1) + 1\\
										&= x \qquad & &= x
									\end{alignat*}
								\end{answer}
							\item $f: \mathbb{R} \rightarrow \mathbb{R}$ is defined by $f(x) = 3x - 2$ and $g: \mathbb{R} \rightarrow \mathbb{R}$ is defined by $g(x) = x^{2} + x$.\\
								\begin{answer}
									All these composite functions are defined on $\mathbb{R} \rightarrow \mathbb{R}$.
									\begin{alignat*}{2}
										(f \circ f)(x) &= f\bigl(f(x)\bigr) \qquad & (g \circ g)(x) &= g\bigl(g(x)\bigr)\\
										&= f(3x - 2) \qquad & &= g\left(x^{2} + x\right)\\
										&= 3(3x - 2) - 2 \qquad & &= \left(x^{2} + x\right)^{2} + \left(x^{2} + x\right)\\
										&= 9x - 6 - 2 \qquad & &= x^{4} + 2x^{3} + x^{2} + x^{2} + x\\
										&= 9x - 8 \qquad & &= x^{4} + 2x^{3} + 2x^{2} + x
									\end{alignat*}
									\begin{alignat*}{2}
										(g \circ f)(x) &= g\bigl(f(x)\bigr) \qquad & (f \circ g)(x) &= f\bigl(g(x)\bigr)\\
										&= g(3x - 2) \qquad & &= f\left(x^{2} + x\right)\\
										&= (3x - 2)^{2} + (3x - 2) \qquad & &= 3\left(x^{2} + x\right) - 2\\
										&= 9x^{2} - 12x + 4 + 3x - 2 \qquad & &= 3x^{2} + 3x - 2\\
										&= 9x^{2} - 9x + 2
									\end{alignat*}
								\end{answer}
							\pagebreak
							\item $f: \mathbb{Z}^{\geq} \rightarrow \mathbb{Z}^{\geq}$ is defined by $f(x) = 113$ and $g: \mathbb{Z}^{\geq} \rightarrow \mathbb{Z}^{\geq}$ is defined by $g(x) = x + 1$.\\
								\begin{answer}
									All these composite functions are defined on $\mathbb{Z}^{\geq} \rightarrow \mathbb{Z}^{\geq}$.
									\begin{alignat*}{2}
										(f \circ f)(x) &= f\bigl(f(x)\bigr) \qquad & (g \circ g)(x) &= g\bigl(g(x)\bigr)\\
										&= f(113) \qquad & &= g(x + 1)\\
										&= 113 \qquad & &= (x + 1) + 1\\
										& \qquad & &= x + 2
									\end{alignat*}
									\begin{alignat*}{2}
										(g \circ f)(x) &= g\bigl(f(x)\bigr) \qquad & (f \circ g)(x) &= f\bigl(g(x)\bigr)\\
										&= g(113) \qquad & &= f(x + 1)\\
										&= 113 + 1 \qquad & &= 113\\
										&= 114
									\end{alignat*}
								\end{answer}
						\end{questions}
				\end{questions}
			\end{exercise}
		\section{Bijective and Invertible Functions}
			\subsection{Bijective Function}
				\begin{definition}{Bijective Function}
					A function $f: A \rightarrow B$ is \concept{bijective} iff $f$ is both surjective and injective.
				\end{definition}
				\begin{example}[width=0.8\textwidth]
					Let $f: \mathbb{Z} \rightarrow \mathbb{Z}$ be defined by $y = x + 2$.
					\begin{description}
						\item[Surjectivity] Determine the range of $f$.
							\begin{align*}
								\mathrm{ran}(f) &= \{f(x) \mid x \in \mathbb{Z}\}\\
								&= \{x + 2 \mid x \in \mathbb{Z}\}\\
								&= \{y \mid y - 2 \in \mathbb{Z}\}\tag*{$x = y - 2$}\\
								&= \{y \mid y \in \mathbb{Z}\}\\
								&= \mathbb{Z}
							\end{align*}
							Therefore, $f$ is surjective.
						\item[Injectivity]
							\begin{alignat*}{2}
								&\text{Assume } \quad &f(u) &= f(v)\\
								&\text{Then } &u + 2 &= v + 2\\
								&\text{i.e.} &u &= v
							\end{alignat*}
							Therefore $f$ is injective.
						\item[Bijectivity] As $f$ is both surjective and injective, $f$ is bijective.
					\end{description}
				\end{example}
			\pagebreak
			\subsection{Invertible Functions}
				\begin{definition}{Invertible Function}
					A function $f: A \rightarrow B$ is \concept{invertible} iff the inverse relation $f^{-1}$ is a function from $B$ to $A$.\\
					This occurs iff the function $f$ is bijective.
				\end{definition}
				\begin{theorem}{A function $f$ is invertible iff $f$ is bijective}
					\begin{proof}
						$ $
						\begin{indentparagraph}
							\begin{subproof}[Subproof]
								Suppose that $f: A \rightarrow B$ is an invertible function.\\
								Then $f^{-1} = \bigl\{(y, x) \mid (x, y) \in f\bigr\}$ is a function from $B$ to $A$.

								So the \emph{domain} of $f^{-1}$ is $B$. But the domain of $f^{-1}$ is also the set of $y$'s such that $(x, y) \in f$ for some $x$ i.e. the domain of $f^{-1}$ is the \emph{range} of $f$. So the range of $f$ is $B$.\\
								So $f: A \rightarrow B$ is surjective.

								As $f^{-1}$ is a function, an element $y \in B$ appears only once as the first coordinate in an ordered pair in $f^{-1}$. That is, if $(y, x_{1})$ and $(y, x_{2})$ are both in $f^{-1}$, then $x_{1} = x_{2}$.\\
								So $f: A \rightarrow B$ is injective.

								If $f$ is an invertible function, then $f$ is surjective and injective, so $f$ is bijective.
							\end{subproof}
							\begin{subproof}[Subproof]
								Suppose that $f: A \rightarrow B$ is bijective.

								As $f$ is surjective, every element of $B$ appears as the second coordinate in an ordered pair of $f$. Therefore, every $b \in B$ appears as the first coordinate in an ordered pair of $f^{-1}$.\\
								Therefore, the domain of $f^{-1}$ is $B$.

								As $f$ is injective, every element of $B$ appears \emph{only once} as the second coordinate in an ordered pair of $f$. Therefore, every $b \in B$ appears only once in an ordered pair of $f^{-1}$.\\
								Therefore, $f^{-1}$ is functional.

								If $f$ is bijective, then $f^{-1}$ is functional. $\mathrm{dom}(f^{-1})$ equals the codomain of $f$, so $f^{-1}$ is a function. Therefore, $f$ is invertible.
							\end{subproof}
						\end{indentparagraph}
						If a function is invertible, it is bijective. If a function is bijective, it is invertible.
					\end{proof}
				\end{theorem}
				\begin{example}
					Let $f: \mathbb{Z} \rightarrow \mathbb{Z}$ be defined by $y = x + 2$.\\
					As was shown in the previous example, this function is bijective. As it is bijective, it is invertible. The inverse function of $f$, $f^{-1}$ is a function from $\mathbb{Z}$ to $\mathbb{Z}$.
					\begin{align*}
						(y, x) \in f^{-1} &\text{ iff } (x, y) \in f\\
						& \text{ iff } y = x + 2\\
						(x, y) \in f^{-1} &\text{ iff } x = y + 2 \tag*{(swap variables)}\\
						&\text{ iff } x - 2 = y \tag*{(solve for $y$)}\\
						&\text{ iff } y = x - 2\\
						(y, x) \in f^{-1} & \text{ iff } x = y - 2 \tag*{(swap variables back)}
					\end{align*}
					$f^{-1}: \mathbb{Z} \rightarrow \mathbb{Z}$ is defined by $f^{-1}(y) = y - 2$
				\end{example}
		\section{Identity Function}
			\begin{definition}{Identity Function}
				For any set $A$, the function $i_{A}: A \rightarrow A$ is the function such that $i_{A}(x) = x$ for all $x \in A$. This function is called the \concept{identity function}.
			\end{definition}
			\begin{example}[width=0.8\textwidth]
				Let $B = \{2, 4, 6, 8\}$. The identity function $i_{B}: B \rightarrow B$ would be:
					\begin{align*}
						i_{B} = \bigl\{(2, 2), (4, 4), (6, 6), (8, 8)\bigr\}
					\end{align*}
			\end{example}
			\begin{exercise}{Self Assessment Exercise \thechapter.11}
				\begin{questions}
					\item In each of the following cases, write down the bijective functions from $X$ to $Y$ (if possible).
						\begin{questions}
							\item $X = \bigl\{\emptyset, \{113\}\bigr\}$ and $Y = \bigl\{\{1\}\bigr\}$.\\
								\begin{answer}
									There are no possible bijective functions, as there are more elements in $X$ than in $Y$. That means that there cannot be an injective function, so there cannot be a bijective function.
								\end{answer}
							\item $X = \bigl\{\emptyset, \{113\}\bigr\}$ and $Y = \bigl\{\{1\}, \{2\}\bigr\}$.
								\begin{answer}
									\begin{align*}
										f_{1} &= \Bigl\{\bigl(\emptyset, \{1\}\bigr), \bigl(\{113\}, \{2\}\bigr)\Bigr\}\\
										f_{2} &= \Bigl\{\bigl(\emptyset, \{2\}\bigr), \bigl(\{113\}, \{1\}\bigr)\Bigr\}
									\end{align*}
								\end{answer}
							\item $X = \bigl\{\emptyset, \{113\}\bigr\}$ and $Y = \bigl\{\{1\}, \{2\}, \{7\}\bigr\}$.\\
								\begin{answer}
									There are no possible bijective functions, as there are more elements in $Y$ than in $X$. That means that there cannot be a surjective function, so there cannot be a bijective function.
								\end{answer}
						\end{questions}
					\item Check the following functions for injectivity, surjectivity and bijectivity, and give the inverse relation of each:
						\begin{questions}
							\item $f: \mathbb{Z} \rightarrow \mathbb{Z}$ is defined by $f(x) = x + 1$.
								\begin{answer}
									\begin{description}
										\item[Injectivity] This function is injective.
											\begin{proof}
												$ 
												\begin{alignedat}[t]{2}
													& \text{Assume} \quad &f(u) &= f(v)\\
													& \text{Then} & u + 1 &= v + 1\\
													& \text{i.e.} & u &= v
												\end{alignedat}
												$\\
												Therefore $f$ is injective.
											\end{proof}
										\pagebreak
										\item[Surjectivity] This function is surjective.
											\begin{proof}
												$
												\begin{aligned}[t]
													f[\mathbb{Z}] &= \{f(x) \mid x \in \mathbb{Z}\}\\
													&= \{x + 1 \mid x \in \mathbb{Z}\}\\
													&= \{y \mid y - 1 \in \mathbb{Z}\}\\
													&= \{y \mid y \in \mathbb{Z}\}\\
													&= \mathbb{Z}
												\end{aligned} $\\
												Therefore $f$ is surjective.
											\end{proof}
										\item[Bijectivity] As $f$ is injective and surjective, $f$ is bijective.
										\item[Inverse Function] $
												\begin{aligned}[t]
													(y, x) \in f^{-1} &\text{ iff } (x, y) \in f\\
													&\text{ iff } y = x + 1\\
													(x, y) \in f^{-1} &\text{ iff } x = y + 1\\
													&\text{ iff } x - 1 = y\\
													&\text{ iff } y = x - 1\\
													(y, x) \in f^{-1} &\text{ iff } x = y - 1
												\end{aligned} $\\
											$f^{-1}(y) = y - 1$
									\end{description}
								\end{answer}
							\item $f: \mathbb{Z} \rightarrow \mathbb{Z}$ is defined by $f(x) = x^{2}$.
								\begin{answer}
									\begin{description}
										\item[Injectivity] This function is \emph{not} injective.
											\begin{proof}
												$ 
													\begin{aligned}[t]
														& \text{Assume} \quad &f(u) &= f(v)\\
														& \text{Then} & u^{2} &= v^{2}\\
														& \text{i.e.} & \pm u &= \pm v
													\end{aligned}
												$\\
												Therefore, $f$ is not injective.
											\end{proof}
										\item[Surjectivity] This function is \emph{not} surjective.
											\begin{proof}
												$ 
													\begin{aligned}[t]
														f[\mathbb{Z}] &= \{f(x) \mid x \in \mathbb{Z}\}\\
														&= \{x^{2} \mid x \in \mathbb{Z}\}\\
														&= \{y \mid \pm \sqrt{y} \in \mathbb{Z}\}\\
														& \neq \mathbb{Z}
													\end{aligned}
												$
												\begin{subproof}[Counterexample]
													Suppose $y = -1$, as $-1 \in \mathbb{Z}$. There is no $x \in \mathbb{Z}$ such that $x^{2} = -1$, so the range of $f$ is not equal to the codomain.
												\end{subproof}
												Therefore, $f$ is not surjective.
											\end{proof}
										\item[Bijectivity] As $f$ is neither injective nor surjective, $f$ is not bijective.
										\item[Inverse Function] As $f$ is not bijective, $f^{-1}$ is not defined. 
									\end{description}
								\end{answer}
							\pagebreak
							\item $f: \mathbb{Z} \rightarrow \mathbb{Z}$ is defined by $f(x) = 3 - x$.
								\begin{answer}
									\begin{description}
										\item[Injectivity] This function is injective.
											\begin{proof}
												$ 
													\begin{aligned}[t]
														& \text{Assume} \quad &f(u) &= f(v)\\
														& \text{Then} &3 - u &= 3 - v\\
														& \text{i.e.} & u &= v
													\end{aligned}
												$ \\
												Therefore $f$ is injective.
											\end{proof}
										\item[Surjectivity] This function is surjective.
											\begin{proof}
												$ 
													\begin{aligned}[t]
														f[\mathbb{Z}] &= \{f(x) \mid x \in \mathbb{Z}\}\\
														&= \{3 - x \mid x \in \mathbb{Z}\}\\
														&= \{y \mid 3 - y \in \mathbb{Z}\}\\
														&= \mathbb{Z}
													\end{aligned}
												$ \\
												Therefore $f$ is surjective.
											\end{proof}
										\item[Bijectivity] As this function is injective and surjective, this function is bijective.
										\item[Inverse Function] $ 
											\begin{aligned}[t]
												(y, x) \in f^{-1} &\text{ iff } (x, y) \in f\\
												& \text{ iff } y = 3 - x\\
												(x, y) \in f^{-1} & \text{ iff } x = 3 - y\\
												& \text{ iff } x + y = 3\\
												& \text{ iff } y = 3 - x\\
												(y, x) \in f^{-1} & \text{ iff } x = 3 - y
											\end{aligned}
										$ \\
										$f^{-1}(y) = 3 - y$
									\end{description}
								\end{answer}
							\item $f: \mathbb{Z} \rightarrow \mathbb{Z}$ is defined by $f(x) = 4x + 5$.
								\begin{answer}
									\begin{description}
										\item[Injectivity] This function is injective.
											\begin{proof}
												$ 
													\begin{aligned}[t]
														& \text{Assume} \quad &f(u) &= f(v)\\
														& \text{Then} &4u + 5 &= 4v + 5\\
														& & 4u &= 4v\\
														& \text{i.e.} & u &= v
													\end{aligned}
												$ \\
												Therefore $f$ is injective.
											\end{proof}
										\item[Surjectivity] This function is \emph{not} surjective.
											\begin{proof}
												$ 
													\begin{aligned}[t]
														f[\mathbb{Z}] &= \{f(x) \mid x \in \mathbb{Z}\}\\
														&= \{4x + 5 \mid x \in \mathbb{Z}\}\\
														&= \{y \mid \frac{y - 5}{4} \in \mathbb{Z}\}\\
														& \neq \mathbb{Z}
													\end{aligned}
												$ \\
												Therefore $f$ is not surjective.
											\end{proof}
										\item[Bijectivity] As $f$ is not surjective, $f$ is not bijective.
										\item[Inverse Function] As $f$ is not bijective, $f^{-1}$ is not defined. 
									\end{description}
								\end{answer}
						\end{questions}
					\pagebreak
					\item Consider an identity function $i_{C}: C \rightarrow C$.
						\begin{questions}
							\item Prove that $i_{C}: C \rightarrow C$ is bijective.
								\begin{answer}
									\begin{proof}
										$ $
										\begin{subproof}[Injectivity]
											$ 
												\begin{aligned}[t]
													& \text{Assume} \quad &i_{C}(u) &= i_{C}(v)\\
													& \text{Then} & u &= v
												\end{aligned}
											$ \\
											Therefore $i_{C}$ is injective.
										\end{subproof}
										\begin{subproof}[Surjectivity]
											$ 
												\begin{aligned}[t]
													i_{C}[C] &= \{i_{c} \mid c \in C\}\\
													&= \{c \mid c \in C\}\\
													&= C
												\end{aligned}
											$ \\
											Therefore $i_{C}$ is surjective.
										\end{subproof}
										As $i_{C}$ is both injective and surjective, $i_{C}$ is bijective.
									\end{proof}
								\end{answer}
							\item Prove that $i_{C}$ is an equivalence relation on $C$.
								\begin{answer}
									\begin{proof}
										\begin{subproof}[Reflexivity]
											For every $c \in C$, is $(c, c) \in i_{C}$? Yes.\\
											For any $c \in C$, $c = c$.\\
											So $(c, c) \in i_{C}$.\\
											Therefore, $i_{C}$ is reflexive.
										\end{subproof}
										\begin{subproof}[Symmetry]
											For every $c, d \in C$, if $(c, d) \in i_{C}$, is $(d, c) \in i_{C}$? Yes.\\
											Suppose $(c, d) \in C$. Then $c = d$. So $d = c$.\\
											Therefore, $(d, c) \in i_{C}$.\\
											Therefore, $i_{C}$ is symmetric.
										\end{subproof}
										\begin{subproof}[Transitivity]
											If $(c, d) \in i_{C}$ and $(d, e) \in i_{C}$, is $(c, e) \in i_{C}$? Yes.\\
											Assume $(c, d) \in i_{C}$ and $(d, e) \in i_{C}$.\\
											Then $c = d$ and $d = e$.\\
											So $c = d = e$.\\
											So $c = e$.\\
											Therefore, $(c, e) \in i_{C}$.
										\end{subproof}
										As $i_{C}$ is reflexive, symmetric and transitive, $i_{C}$ is an equivalence relation.
									\end{proof}
								\end{answer}
						\end{questions}
				\end{questions}
			\end{exercise}
	\rulechapterend
\end{document}
