\documentclass[../notes.tex]{subfiles}

\begin{document}
	\ifSubfilesClassLoaded{\setcounter{chapter}{6}}{}
	\chapter{More About Functions}
		\section{The Range of a Function}
			\begin{definition}{Range of a Function}
				Given a function $f: A \rightarrow B$, the \concept{range} or \concept{image set} of $f$ is the subset
				\begin{align*}
					\{f(x) \mid x \in A\}
				\end{align*} 
				of $B$, written $\mathrm{ran}(f)$ or $f[A]$.\\
				In other words, it is a subset of $B$ where and element $b$ of $B$ can be reached by calling the function with a specific element $a$ of $A$.
			\end{definition}
			\begin{example}
				Let $A = \{a, b, c\}$ and $B = \{1, 2, 3\}$. Let a function $f: A \rightarrow B$ be defined as 
				\begin{alignat*}{3}
					f(a) &= 1 \qquad & f(b) &= 2 \qquad & f(c) &= 1
				\end{alignat*}
				Then $\mathrm{ran}(f) = \{1, 2\}$. 
			\end{example}
			\subsection{Determining the Range of a Function}
				In order to find the range, you follow these steps:
				\begin{enumerate}
					\item Write the defintion of the function $\bigl(f(x)\bigr)$, and the domain.
					\item Substitute the definition with the value of $f(x)$
					\item Calculate the first coordinate in terms of the second.
					\item Substitute the second coordinate and that formula for it into the definition.
					\item Simplify.
				\end{enumerate}
				\pagebreak
				Easier to show with an example:
				\begin{example}
					Let $g: \mathbb{Z} \rightarrow \mathbb{Z}$ be defined by $y = 2x$.
					\begin{align*}
						\mathrm{ran}(g) &= \{g(x) \mid x \in \mathbb{Z}\}\tag*{$(1)$}\\
						&= \{2x \mid x \in \mathbb{Z}\}\tag*{$(2)$}\\
					\end{align*}
					Calculate $x$ in terms of $y$ (Step $3$):
					\begin{alignat*}{2}
						& \qquad &y &= 2x\\
						& \Rightarrow \quad &\frac{y}{2} &= x\\
						& \Rightarrow \quad &x &= \frac{y}{2}\\
					\end{alignat*}
					\begin{align*}
						\mathrm{ran}(g) &= \{y \mid \frac{y}{2} \in \mathbb{Z}\}\tag*{$(4)$}\\
						&= \{y \mid \frac{y}{2} \text{ is an integer}\}\tag*{$(5)$}\\
						&= \{y \mid y \text{ is an even integer}\}
					\end{align*}
				\end{example}
		\section[Surjectivity]{Surjectivity (MAPPING)}
			\begin{definition}{Surjectivity}
				Given a function $f: A \rightarrow B$, the function $f$ would be \concept{surjective} iff the \emph{range} of $f$ is equal to the codomain of $f$.\\
				As $B$ is the codomain of $f$ above, that would mean that $\mathrm{ran}(f)$ (also written $f[A]$) is equal to $B$.
			\end{definition}
			\begin{example}
				Let $A = \{1, 2, 3\}$. Let $B = \{4, 5, 6\}$.
				\begin{description}
					\item[Surjective Function] For a surjective function, every element of $A$ needs to be present, and every element of $B$. So an example of a function $h: A \rightarrow B$ would be:
						\begin{align*}
							h = \Bigl\{(1, 6), (2, 4), (3, 5)\Bigr\}
						\end{align*}
						$\mathrm{ran}(h) = \{4, 5, 6\} = B$.
					\item[Non-Surjective Function] For a function, every element of $A$ needs to be present. For it to not be surjective, that means that at least one element of $B$ is not in the range of the function. An example function $h: A \rightarrow B$ would be:
						\begin{align*}
							h = \Bigl\{(1, 4), (2, 4), (3, 5)\Bigr\}
						\end{align*}
						$\mathrm{ran}(h) = \{4, 5\} \neq B$.
				\end{description}
			\end{example}
		\section[Injectivity]{Injectivity (ONE TO ONE)}
			\begin{definition}{Injectivity}
				A function $f: A \rightarrow B$ is \concept{injective} iff $f$ has the property that whenever $f(a_{1}) = f(a_{2})$, then $a_{1} = a_{2}$.\\
				In other words, every unique first coordinate is related to a unique second coordinate.
				\begin{description}
					\item[Another definition (contrapositive)] A function $f: A \rightarrow B$ is \concept{injective} iff $f$ has the property that whenever $a_{1} \neq a_{2}$, $f(a_{1}) \neq f(a_{2})$.
				\end{description}
			\end{definition}
			\begin{example}
				Let $A = \{1, 2, 3\}$. Let $B = \{4, 5, 6, 7\}$.
				\begin{description}
					\item[Injective Function] For an injective function, every element of $A$ should be related to a different element of $B$. An example function $g: A \rightarrow B$ would be:
						\begin{align*}
							g = \Bigl\{(1, 5), (2, 7), (3, 6)\Bigr\}
						\end{align*}
					\item[Non-Injective Function] For a function to not be injective, two or more elements of $A$ should be related to the same element of $B$. An example function $g: A \rightarrow B$ would be:
						\begin{align*}
							g = \Bigl\{(1, 4), (2, 5), (3, 4)\Bigr\}
						\end{align*}
				\end{description}
			\end{example}
			\subsection{Determining Whether an Abstract Function is Injective}
				For functions defined on all elements of an infinite set such as $\mathbb{Z}$, use logic to prove the function is injective:
				\nopagebreak
				\begin{enumerate}[nosep]
					\item Assume that the function being applied to two different elements results in the same value.
					\item Apply the function to the values.
					\item Simplify using algebra.
				\end{enumerate}
				\nopagebreak
				\begin{example}
					\begin{description}
						\item[Prove Injectivity] Let $f: \mathbb{Z} \rightarrow \mathbb{Z}$ be defined by $y = 4x$.
							\begin{alignat*}{2}
								& \text{Assume } \quad &f(u) &= f(v)\tag*{$(1)$}\\
								& \text{Then } &4u &= 4v\tag*{$(2)$}\\
								& \text{i.e. } &u &= v\tag*{$(3)$}
							\end{alignat*}
						\item[Non Injective Function]  Let $f: \mathbb{Z} \rightarrow \mathbb{Z}$ be defined by $y = x^{2}$.
							\begin{alignat*}{2}
								& \text{Assume } \quad &f(u) &= f(v)\\
								& \text{Then } &u^{2} &= v^{2}\\
								& & \pm{u} &= \pm{v}\\
								& & u &\neq v
							\end{alignat*}
							$u$ is not necessarily equal to $v$. $u$ could be $1$, and $v$ could be $-1$, and $f(u)$ would be equal to $f(v) = 1$. Therefore, $f$ is not injective.
					\end{description}
				\end{example}
		\section{Composition of Functions}
				\begin{definition}{Composite Function}
					The composition of two functions is also a function.\\
					Given the functions $f: A \rightarrow B$ and $g: B \rightarrow C$, the \concept{composite function} $g \circ f: A \rightarrow C$ is defined by
					\begin{align*}
						g \circ f: A \rightarrow C &= g \circ f(x)\\
						&= g\bigl(f(x)\bigr)
					\end{align*}
				\end{definition}
			\begin{example}
				Let $f: \mathbb{Z} \rightarrow \mathbb{Z}$ be defined by $f(x) = 4x + 2$.\\
				Let $g: \mathbb{Z} \rightarrow \mathbb{Z}^{+}$ be defined by $f(x) = x^{2} + 1$.\\
				Then $g \circ f: \mathbb{Z} \rightarrow \mathbb{Z}^{+}$.
				\begin{align*}
					(g \circ f)(x) &= g\bigl(f(x)\bigr)\\
					&= g(4x + 2)\\
					&= (4x + 2)^{2} + 1\\
					&= \left(16x^{2} + 16x + 4\right) + 1\\
					&= 16x^{2} + 16x + 5
				\end{align*}
				$(g \circ f)(x)$ is called the \concept{image of $x$ under $g \circ f$}.
			\end{example}
			The composition of two surjective functions is surjective.\\
			The composition of two injective functions is injective.
		\pagebreak
		\section{Bijective and Invertible Functions}
			\subsection[Inverse Relation]{Inverse Relation (RECAP)}
				\begin{definition}{Inverse Relation}
					For any relation $R$, the \concept{inverse relation}, written $R^{-1}$, is the set $\bigl\{(y, x) \mid (x, y) \in R\bigr\}$.
				\end{definition}
			\subsection{Bijective Function}
				\begin{definition}
					A function $f: A \rightarrow B$ is \concept{bijective} iff $f$ is both surjective and injective.
				\end{definition}
				\begin{example}
					Let $f: \mathbb{Z} \rightarrow \mathbb{Z}$ be defined by $y = x + 2$.
					\begin{description}
						\item[Surjectivity] Determine the range of $f$.
							\begin{align*}
								\mathrm{ran}(f) &= \{f(x) \mid x \in \mathbb{Z}\}\\
								&= \{x + 2 \mid x \in \mathbb{Z}\}\\
								&= \{y \mid y - 2 \in \mathbb{Z}\}\tag*{$x = y - 2$}\\
								&= \{y \mid y \in \mathbb{Z}\}\\
								&= \mathbb{Z}
							\end{align*}
							Therefore, $f$ is surjective.
						\item[Injectivity]
							\begin{alignat*}{2}
								&\text{Assume } \quad &f(u) &= f(v)\\
								&\text{Then } &u + 2 &= v + 2\\
								&\text{i.e.} &u &= v
							\end{alignat*}
							Therefore $f$ is injective.
						\item[Bijectivity] As $f$ is both surjective and injective, $f$ is bijective.
					\end{description}
				\end{example}
			\pagebreak
			\subsection{Invertible Functions}
				\begin{definition}
					A function $f: A \rightarrow B$ is \concept{invertible} iff the inverse relation $f^{-1}$ is a function from $B$ to $A$.\\
					This occurs iff the function $f$ is bijective.
				\end{definition}
				\begin{example}
					Let $f: \mathbb{Z} \rightarrow \mathbb{Z}$ be defined by $y = x + 2$.\\
					As was shown in the previous example, this function is bijective. As it is bijective, it is invertible. The inverse function of $f$, $f^{-1}$ is a function from $\mathbb{Z}$ to $\mathbb{Z}$.
					\begin{align*}
						(y, x) \in f^{-1} &\text{ iff } (x, y) \in f\\
						& \text{ iff} y = x + 2\\
						& \text{ iff} x = y - 2
					\end{align*}
					$f^{-1}: \mathbb{Z} \rightarrow \mathbb{Z}$ is defined by $f^{-1}(y) = y - 2$
				\end{example}
		\section{Identity Function}
			\begin{definition}{identity Function}
				For any set $A$, the function $i_{A}: A \rightarrow A$ is the function such that $i_{A}(x) = x$ for all $x \in A$. This function is called the \concept{identity function}.
			\end{definition}
			\begin{example}
				Let $B = \{2, 4, 6, 8\}$. The identity function $i_{B}: B \rightarrow B$ would be:
					\begin{align*}
						i_{B} = \bigl\{(2, 2), (4, 4), (6, 6), (8, 8)\bigr\}
					\end{align*}
			\end{example}
	\rulechapterend
\end{document}
