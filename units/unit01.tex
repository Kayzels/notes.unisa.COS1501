\documentclass[../notes.tex]{subfiles}

\begin{document}
	\chapter{Number Systems}
		\section{Number Properties}
			\subsection{Commutativity}
				For all integers $m$ and $n$, \emph{addition} and \emph{multiplication} are \concept{commutative}.
				\nopagebreak
				\begin{align*}
					m + n &= n + m \tag*{addition}\\
					mn &= nm \tag*{multiplication}
				\end{align*}
			\subsection{Associativity}
				For all integers $m$, $n$ and $k$, \emph{addition} and \emph{multiplication} are \concept{associaive}.
				\nopagebreak
				\begin{align*}
					m+(n+k) &= (m+n)+k \tag*{addition}\\
					(m)(nk) &= (mn)k \tag*{multiplication}
				\end{align*}
			\subsection{Distributivity}
				For all integers $m$, $n$ and $k$, \emph{multiplication} is \concept{distributive} over \emph{addition}.
				\nopagebreak
				\begin{align*}
					m(n + k) &= mn + mk\\
					(n + k)m &= m(n + k)\\
					&= mn + mk\\
					&= nm + km
				\end{align*}
			\subsection{Multiplicative Identity}
				There exists an integer ($1$) that has the property that for every integer $m$, $m\cdot 1 = m$.
			\subsection{Additive Identity}
				There exists an integer ($0$) that has the property that for every integer $m$, $m + 0 = m$.
			\subsection{Linearity}
				For all integers $m$ and $n$, exactly one of the following is true:
				\nopagebreak
				\begin{align*}
					m &< n\\
					m &= n\\
					m &> n
				\end{align*}
			\subsection{Monotocity}
				For all integers $m$, $n$ and $k$,
				\nopagebreak
				\begin{indentparagraph}
					If $m = n$, then $m + k = n + k$ and $mk = nk$.\\
					If $m < n$, then $m + k < n + k$.
					\begin{indentparagraph}
						If $k > 0$, then $mk < nk$.\\
						If $k < 0$, then $mk < nk$.
					\end{indentparagraph}
				\end{indentparagraph}
			\subsection{Transitivity of \texorpdfstring{$=$}{=}, \texorpdfstring{$<$}{<} and \texorpdfstring{$>$}{>}}
				For all integers $m$, $n$ and $k$,
				\nopagebreak
				\begin{indentparagraph}
					If $m = n$ and $n = k$, then $m = k$.\\
					If $m < n$ and $n < k$, then $m < k$.\\
					If $m > n$ and $n > k$, then $m > k$.
				\end{indentparagraph}
			\subsection{Absence of Zero Divisors}
				For all integers $m$ and $n$,
				\nopagebreak
				\begin{indentparagraph}
					$mn = 0$ if and only if $m = 0$ or $n = 0$.
				\end{indentparagraph}
			\subsection{Additive Inverses}
				For every integer $m$ there exists an integer $n$ such that
				\nopagebreak
				\begin{align*}
					m + n = 0
				\end{align*}
			\pagebreak
			\begin{exercise}{Self Assessment Exercise (Activity \thechapter.11)}
				\begin{enumerate}
					\item \question{Factorise the following expressions}:
						\begin{enumerate}[label=(\alph*)]
							\item \moveup
								\begin{align*}
									x^{2} + 6x + 9 &= (x + 3)^{2}
								\end{align*}
							\item \moveup
								\begin{align*}
									x^{2} - x - 2 &= (x - 2)(x + 1)
								\end{align*}
							\item \moveup
								\begin{align*}
									x^{2} - 5x + 6 &= (x - 3)(x - 2)
								\end{align*}
							\item \moveup
								\begin{align*}
									x^{2} + 4x - 12 &= (x + 6)(x - 2)
								\end{align*}
						\end{enumerate}
					\item \question{Solve $x^{2} - 4x + 4 = 0$ by factorising}:
						\begin{alignat*}{3}
							& \qquad & x^{2} - 4x + 4 &= 0\\
							& \Rightarrow \quad & (x - 2)(x - 2) &= 0\\
							& \Rightarrow \quad &x &= 2 &
						\end{alignat*}
					\item \question{Complete the square to solve $x^{2} - 4x = 12$}
						\begin{alignat*}{2}
							& \qquad &x^{2} - 4x &= 12\\
							& \Rightarrow \quad &x^{2} - 4x + 4 &= 12 + 4\\
							& \Rightarrow \quad &(x - 2)^{2} &= 16\\
							& \Rightarrow \quad & x - 2 &= \pm 4
						\end{alignat*}
						\begin{alignat*}{3}
							& \qquad &x - 2 &= 4 \qquad &x - 2 &= -4\\
							& \Rightarrow \quad &x &= 6 \quad \text{ or } \quad &x &= -2 
						\end{alignat*}
					\item \question{Is $21$ a prime number?}\\
						No, as $3$ and $7$ are both factors of $21$.
					\item \question{What is the value of $5!$ ($5$ factorial)?}\\
						$5! = 5 \times 4 \times 3 \times 2 \times 1 = 120$
				\end{enumerate}
			\end{exercise}
		\rulechapterend
\end{document}
