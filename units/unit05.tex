\documentclass[../notes.tex]{subfiles}

\begin{document}
	\ifSubfilesClassLoaded{\setcounter{chapter}{4}}{}
	\chapter{Relations}
		\section{Ordered Pairs}
			In sets the order of the elements is insignificant. If the order of the elements is significant, it is written with an \textbf{ordered pair}, which is written in round brackets ().
			\begin{example}
				An ordered pair is written $(a, b)$ where $a$ and $b$ are elements of the pair. $(a, b) \neq (b, a)$.
			\end{example}
		\section{Cartesian Product}
			For any sets $A$ and $B$, the \textbf{Cartesian product} of $A$ and $B$ is written $A \times B$, and is equal to the set
			\begin{align*}
				\bigl\{(x, y) \mid x \in A \text{ and } y \in B\bigr\}
			\end{align*}
			In other words, the Cartesian product $A \times B$ denotes a set of ordered pairs such that all the first coordinates of the pairs are elements of set $A$, and all the second coordinates of the pairs are elements of set $B$.
			\begin{example}
				$A = \{2, 3, 4\}, B = \{5, 6\}$
				\begin{align*}
					A \times B &= \bigl\{(2, 5), (2, 6), (3, 5), (3, 6), (4, 5), (4, 6)\bigr\}\\
					B \times A &= \bigl\{(5, 2), (5, 3), (5, 4), (6, 2), (6, 3), (6, 4)\bigr\}\\
					B \times B &= \bigl\{(5, 5), (5, 6), (6, 5), (6, 6)\bigr\}\\
					A \times A &= \bigl\{(2, 2), (2, 3), (2, 4), (3, 2), (3, 3), (3, 4), (4, 2), (4, 3), (4, 4)\bigr\}
				\end{align*}
			\end{example}
		\pagebreak
		\section{Relation}
			A subset of a Cartesian product from $C$ to $D$ is called a \textbf{relation} from $C$ to $D$.
			\begin{example}
				$A = \{2, 3, 4\}$ and  $B = \{6, 7\}$. The following are some relations from $A$ to $B$
				\begin{align*}
					\emptyset \tag*{(This is a subset, even though it has no elements)}\\
					\bigl\{(3, 7)\bigr\}\\
					\bigl\{(2, 6), (2, 7)\bigr\}\\
					\bigl\{(2, 6), (3, 6), (4, 6)\bigr\}\\
					A \times B
				\end{align*}
				\pagebreak
			\end{example}
				\begin{exercise}{Self-Assessment Exercise \thechapter.4}
					\begin{alignat*}{3}
						A &= \{1, 2, 3, 4\} \qquad & B &= \{2, 5\} \qquad & C &= \{3, 4, 7\}
					\end{alignat*}
					\textbf{List the following Cartesian products in list notation:}
					\begin{enumerate}[label=(\alph*)]
						\item $A \times B = \bigl\{(1, 2), (1, 5), (2, 2), (2, 5), (3, 2), (3, 5), (4, 2), (4, 5)\bigr\}$
						\item $B \times A = \bigl\{(2, 1), (2, 2), (2, 3), (2, 4), (5, 1), (5, 2), (5, 3), (5, 4)\bigr\}$
						\item $(A \cup B) \times C$
							\begin{align*}
								A \cup B &= \{1, 2, 3, 4, 5\}\\
								(A \cup B) \times C &= \bigl\{(1, 3), (1, 4), (1, 7), (2, 3), (2, 4), (2, 7), (3, 3), (3, 4), (3, 7),\\
								& \qquad (4, 3), (4, 4), (4, 7), (5, 3), (5, 4), (5, 7)\bigr\}
							\end{align*}
						\item $(A + B) \times B$
							\begin{align*}
								A + B &= \{1, 3, 4, 5\}\\
								(A + B) \times B &= \bigl\{(1, 2), (1, 5), (3, 2), (3, 5), (4, 2), (4, 5), (5, 2), (5, 5)\bigr\}
							\end{align*}
					\end{enumerate}
				\end{exercise}
				\pagebreak
			\subsection{Domain, Range and Codomain}
				Suppose $T$ is a relation from $X$ to $Y$.\\
				Then $Y$ is the \textbf{codomain} of $T$.\\
				\vspace{2mm}\\
				The \textbf{domain} of $T$, written $\mathrm{dom}(T)$ is:
					\begin{align*}
						\mathrm{dom}(T) = \{x \mid \text{for some } y \in Y, (x, y) \in T\}
					\end{align*}
				That is, all the elements that actually appear as first elements in the relation $T$.\\
				\vspace{2mm}\\
				The \textbf{range} of $T$, written $\mathrm{ran}(T)$ is:
				\begin{align*}
					\mathrm{ran}(T) = \{y \mid \text{for some } x \in X, (x, y) \in T\}
				\end{align*}
			That is, all the elements that actually appear as second elements in the relation $T$.
			\begin{sidenote}{Domain and Range are not equal to $X$ and $Y$}
				$\mathrm{dom}(T) \subseteq X$. The domain of the relation is a \textit{subset} of $X$, but not necessarily equal to $X$.\\
				$\mathrm{ran}(T) \subseteq Y$. The range of the relation is a \textit{subset} of $Y$, but not necessarily equal to $Y$.
			\end{sidenote}
			\begin{example}
				Let $S = \bigl\{(a, 1), (b, 1), (a, 2)\bigr\}$ be a relation from $\{a, b, c\}$ to $\{1, 2, 3\}$.\\
				Then $\mathrm{dom}(S) = \{a, b\} \subseteq \{a, b, c\}$.\\
				And $\mathrm{ran}(S) = \{1, 2\} \subseteq \{1, 2, 3\}$.\\
				The codomain of $S$ is the set $\{1, 2, 3\}$. 
			\end{example}
			\subsection{Binary Relation}
				If $R$ is any subset of a Cartesian product $X \times Y$, then $R$ is called a \textbf{binary relation} from $X$ to $Y$, or between $X$ and $Y$.\\
				A subset $R$ of $X \times Y$ is called the \textbf{rule} for the relation.\\
				If $R \subseteq X \times X$, $R$ is a binary relation on $X$.
			\pagebreak
		\section{Properties of Relations}
			\subsection{Reflexivity}
				A relation $R$ on $A$ ($R \subseteq A \times A$) is called \textbf{reflexive} on $A$ iff for every $x \in A$, we have $(x, x) \in R$.\\
				In other words, every element needs to be related to itself (although it can also be related to other elements).
				\begin{example}
					Let $A = \{2, 3, 5\}$. For a relation $S$ to be reflexive on $A$, $\bigl\{(2, 2), (3, 3), (5, 5)\bigr\}$ needs to be a subset of $S$.
						\begin{align*}
							\bigl\{(2, 2), (3, 3), (5, 5)\bigr\} \subseteq S.
						\end{align*}
					Therefore, the relation $\bigl\{(2, 2), (3, 3), (5, 5) (2, 3)\bigr\}$ would be a reflexive relation on $A$.
				\end{example}
			\subsection{Irreflexitivity}
				A relation $R$ on $A$ ($R \subseteq A \times A$) is called \textbf{irreflexive} iff there is \textit{no} $x$ such that $(x, x) \in R$. In other words, for any $x \in A, (x, x) \notin R$
				\begin{example}
					Let $A = \{2, 3, 5\}$.\\
					\vspace{2mm}\\
					$R = \bigl\{(3, 2), (2, 5), (3, 5)\bigr\}$. $R$ is \textbf{irreflexive}, as there is no element that relates to itself. i.e. None of the elements of $\bigl\{(2, 2), (3, 3), (5, 5)\bigr\}$ are elements of $R$.\\
					\vspace{2mm}\\
					$S = \bigl\{(2, 2), (2, 5), (3, 5)\bigr\}$. $S$ is not reflexive, as the elements $\bigl\{(3, 3), (5, 5)\bigr\}$ are not present. $S$ is also not irreflexive, as the element $(2, 2)$ is an element of $S$.
				\end{example}
			\subsection{Symmetry}
				A relation $R$ on $A$ ($R \subseteq A \times A$) is called \textbf{symmetric} iff $R$ has the property that, for all $x, y \in R$, if $(x, y) \in R$, then $(y, x) \in R$.
				\begin{example}
					Let $B = \{1, 2, 3\}$\\
					$R_{1} = \bigl\{(1, 2), (2, 1), (1, 3), (3, 1)\bigr\}$ is symmetric and irreflexive.\\
					$R_{2} = \bigl\{(1, 1), (2, 2), (3, 3), (2, 3)\bigr\}$ is reflexive, but not symmetric.\\
					$R_{3} = \bigl\{(1, 1), (2, 2), (3, 3), (1, 2), (2, 1)\bigr\}$ is symmetric and reflexive.\\
					$R_{4} = \bigl\{(1, 1), (2, 3)\bigr\}$ is not reflexive, irreflexive or symmetric.
				\end{example}
				\pagebreak
			\subsection{Antisymmetry}
				A relation $R$ on $A$ ($R \subseteq A \times A$) is called \textbf{antisymmetric} iff $R$ has the property that, for all $x, y \in R$, if $x \neq y$ and $(x, y) \in R$, then $(y, x) \notin R$.\\
				\textbf{Another definition:} A relation $R$ on $A$ ($R \subseteq A \times A$) is called \textbf{antisymmetric} iff $R$ has the property that, for all $x, y \in R$, if $(x, y) \in R$ and $(y, x) \in R$, then $x = y$.
				\begin{example}
					Let $A = \{a, b, c\}$\\
					$P = \bigl\{(a, b), (b, b), (b, c)\bigr\}$ on $A$.\\
					$a \neq b, (a, b) \in P$, but $(b, a) \notin P$.\\
					$b \neq c, (b, c) \in P$, but $(c, b) \notin P$.\\
					$\therefore P$ is antisymmetric on $A$.
				\end{example}
				\begin{sidenote}{Antisymmetric and Not Symmetric Are Not The Same}
					A relation can be both not antisymmetric and symmetric at the same time. Consider the relation:\\
					$R = \{(1, 2), (2, 1), (2, 3)\}$ on $A = \{1, 2, 3\}$.\\
					This relation is not symmetric, as $(2, 3) \in R$, but $(3, 2) \notin R$.\\
					This relation is also not antisymmetric, since $(1, 2)$ and $(2, 1)$ are elements of $R$, but $1 \neq 2$.
				\end{sidenote}
			\subsection{Transitivity}
				A relation $R$ on $A$ ($R \subseteq A \times A$) is called \textbf{transitive} iff $R$ has the property that, for all $x, y, z \in R$, whenever $(x, y) \in R$ and $(y, z) \in R$, then $(x, z) \in R$.
				\begin{example}
					If $(1, 2) \in R$ and $(2, 3) \in R$, then $(1, 3)$ must be in $R$.
				\end{example}
				\begin{example}
					Let $R = \bigl\{(1, 1), (2, 2), (1, 2), (2, 1)\bigr\}$ be a relation on $A = \{1, 2, 3\}$. This relation is transitive:\\
					$(2, 1)$ and $(1, 2)$ mean $(2, 2)$ should be present.\\
					Can be done with all possible combinations.
				\end{example}
			\subsection{Trichotomy}
				A relation $R$ on $A$ satisfies \textbf{trichotomy} iff, for every $x$ and $y$ chosen from $A$ such that $x \neq y$, $x$ and $y$ are comparable.\\
				In other words, for every $x \neq y$, every element is related to every other element. So $x R y$ or $y R x$.
				\begin{example}
					Let $S = \bigl\{(3, 2), (2, 1), (3, 1)\bigr\}$ be a relation on $A = \{1, 2, 3\}$.\\
					$S$ satisfies the requirements for trichotomy, since:\\
					\-\hspace{1cm} $1$ is related to $2$ in $(2, 1)$ and related to $3$ in $(3, 1)$.\\
					\-\hspace{1cm} $2$ is related to $1$ in $(2, 1)$ and related to $3$ in $(3, 2)$.\\
					\-\hspace{1cm} $3$ is related to $1$ in $(3, 1)$ and related to $2$ in $(3, 2)$.
				\end{example}
			\subsection{Inverse Relation}
				Given a relation $R$ with domain $A$ and range $B$, the relation $R^{-1}$ with domain $B$ and range $A$ is called the \textbf{inverse of $R$}, and is defined such that:
				\begin{align*}
					(x, y) \in R \text{ iff } (y, x) \in R^{-1}
				\end{align*}
				\begin{example}
					Let $X = {a, b, c}$ and $R = \bigl\{(a, b), (b, c), (a, c)\bigr\}$\\
					Then $R^{-1} = \bigl\{(b, a), (c, b), (c, a)\bigr\}$
				\end{example}
			\subsection{Relation Composition}
				Given relations $R$ from $A$ to $B$ and $S$ from $B$ to $C$, the composition of $R$ followed by $S$, written $S \circ R$ or $R;S$ is the relation from $A$ to $C$ defined by:\\
				$S \circ R = R;S = \bigl\{(a, c) \mid$ there is some $b \in B$ such that $(a, b) \in R$ and $(b, c) \in S\bigr\}$
				\begin{example}
					Let $R = \bigl\{(1, a), (2, b)\bigr\}$ be a relation from $\{1, 2\}$ to $\{a, b\}$\\
					Let $S = \bigl\{(a, s), (b, s), (b, t)\bigr\}$ be a relation from $\{a, b\}$ to $\{s, t\}$.\\
					Then $S \circ R = R;S$.
					\begin{adjustwidth}{1cm}{}
						$(1, a) \rightarrow (a, s) \rightarrow (1, s)$\\
						$(2, b) \rightarrow (b, s) \rightarrow (2, s)$\\
						$(2, b) \rightarrow (b, t) \rightarrow (2, t)$
					\end{adjustwidth}
					$S \circ R = R;S = \bigl\{(1, s), (2, s), (2, t)\bigr\}$
				\end{example}
				\pagebreak
				\begin{exercise}{Self Assessment Activity \thechapter.8}
					\begin{enumerate}
						\item \textbf{Let $P$ and $R$ be relations on $A = \bigl\{1, 2, 3, \{1\}, \{2\}\bigr\}$, where\\
						$P = \Bigl\{\bigl(1, \{1\}\bigr), \bigl(1, 2\bigr)\Bigr\}$ and $R = \Bigl\{\bigl(1, \{1\}\bigr), \bigl(1, 3\bigr), \bigl(2, \{1\}\bigr), \bigl(2, \{2\}\bigr), \bigl(\{1\}, 3\bigr), \bigl(\{2\}, \{1\}\bigr) \Bigr\}$}
							\begin{enumerate}[label=(\alph*)]
								\item \textbf{Is $R$ irreflexive?}\\
									Yes. There are no elements that are related to themselves.
								\item \textbf{Is $R$ reflexive?}\\
									No. There are no elements that are related to themselves.
								\item \textbf{Is $R$ symmetric?}\\
									No. $\bigl(1, \{1\}\bigr) \in R$, but $\bigl(\{1\}, 1\bigr) \notin R$.
								\item \textbf{Is $R$ antisymmetric?}\\
									Yes.\\
									$\bigl(1, \{1\}\bigr) \in R$, and $\bigl(\{1\}, 1\bigr) \notin R$.\\
									$\bigl(1, 3\bigr) \in R$, and $\bigl(3, 1\bigr) \notin R$.\\
									$\bigl(2, \{1\}\bigr) \in R$, and $\bigl(\{1\}, 2\bigr) \notin R$.\\
									$\bigl(2, \{2\}\bigr) \in R$, and $\bigl(\{2\}, 2\bigr) \notin R$.\\
									$\bigl(\{1\}, 3\bigr) \in R$, and $\bigl(3, \{1\}\bigr) \notin R$.\\
									$\bigl(\{2\}, \{1\}\bigr) \in R$, and $\bigl(\{1\}, \{2\}\bigr) \notin R$.
								\item \textbf{Is $R$ transitive?}\\
									No. $\bigl(2, \{1\}\bigr) \in R$, and $\bigl(\{1\}, 3\bigr) \in R$, but $\bigl(2, 3\bigr) \notin R$
								\item \textbf{Does $R$ satisfy the requirement for trichotomy?}\\
									No. There is no pair where $1$ is related to $2$.
								\item \textbf{Determine the relation $R \circ R$.}\\
									$R \circ R = R; R$.
									\begin{adjustwidth}{1cm}{}
										$\bigl(1, \{1\}\bigr) \rightarrow \bigl(\{1\}, 3\bigr) \rightarrow \bigl(1, 3\bigr)$\\
										$\bigl(1, 3\bigr) \not \rightarrow$\\
										$\bigl(2, \{1\}\bigr) \rightarrow \bigl(\{1\}, 3\bigr) \rightarrow \bigl(2, 3\bigr)$\\
										$\bigl(2, \{2\}\bigr) \rightarrow \bigl(\{2\}, \{1\}\bigr) \rightarrow \bigl(2, \{1\}\bigr)$\\
										$\bigl(\{1\}, 3\bigr) \not \rightarrow$\\
										$\bigl(\{2\}, \{1\}\bigr) \rightarrow \bigl(\{1\}, 3\bigr) \rightarrow \bigl(\{2\}, 3\bigr)$
									\end{adjustwidth}
									$R \circ R = R; R = \Bigl\{\bigl(1, 3\bigr), \bigl(2, 3\bigr), \bigl(2, \{1\}\bigr), \bigl(\{2\}, 3\bigr)\Bigr\}$
								\item \textbf{Determine the relation $R \circ P$.}
									$R \circ P = P;R$.
									\begin{adjustwidth}{1cm}{}
										$\bigl(1, \{1\}\bigr) \rightarrow \bigl(\{1\}, 3\bigr) \rightarrow \bigl(1, 3\bigr)$\\
										$\bigl(1, 2\bigr) \rightarrow \bigl(2, \{1\}\bigr) \rightarrow \bigl(1, \{1\}\bigr)$\\
										$\bigl(1, 2\bigr) \rightarrow \bigl(2, \{2\}\bigr) \rightarrow \bigl(1, \{2\}\bigr)$
									\end{adjustwidth}
									$R \circ P = R;R = \Bigl\{\bigl(1, 3\bigr), \bigl(1, \{1\}\bigr), \bigl(1, \{2\}\bigr)\Bigr\}$
								\item \textbf{Give the subset $T$ of $R$ where $(a, B) \in T$ iff $a \in B$.}\\
									$T = \Bigl\{\bigl(1, \{1\}\bigr), \bigl(2, \{2\}\bigr)\Bigr\}$
							\end{enumerate}
						\pagebreak
						\item \textbf{Let $A = \{a, b\}$. For each of the specifications given below, find suitable examples of relations on $\mathcal{P}(A)$}
							\begin{align*}
								\mathcal{P}(A) &= \bigl\{\emptyset, \{a\}, \{b\}, \{a, b\}\bigr\}\\
								\mathcal{P}(A) \times \mathcal{P}(A) &= \Bigl\{
									\bigl(\emptyset, \emptyset\bigr), \bigl(\emptyset, \{a\}\bigr), \bigl(\emptyset, \{b\}\bigr), \bigl(\emptyset, \{a, b\}\bigr)\\
									& \quad \bigl(\{a\}, \emptyset\bigr), \bigl(\{a\}, \{a\}\bigr), \bigl(\{a\}, \{b\}\bigr), \bigl(\{a\}, \{a, b\}\bigr)\\
									& \quad \bigl(\{b\}, \emptyset\bigr), \bigl(\{b\}, \{a\}\bigr), \bigl(\{b\}, \{b\}\bigr), \bigl(\{b\}, \{a, b\}\bigr)\\
									& \quad \bigl(\{a, b\}, \emptyset\bigr), \bigl(\{a, b\}, \{a\}\bigr), \bigl(\{a, b\}, \{b\}\bigr), \bigl(\{a, b\}, \{a. b\}\bigr)
								\Bigr\}
							\end{align*}
							\begin{enumerate}[label=(\alph*)]
								\item \textbf{$R$ is reflexive, symmetric and transitive on $\mathcal{P}(A)$}
									\begin{description}
										\item[Reflexivity] To be reflexive, these pairs need to appear:
											\begin{align*}
												\bigl(\emptyset, \emptyset\bigr), \bigl(\{a\}, \{a\}\bigr), \bigl(\{b\}, \{b\}\bigr), \bigl(\{a, b\}, \{a, b\}\bigr)
											\end{align*}
										\item[Symmetry] Whichever pair is added, the pair that makes it symmetric needs to be added too.
											\begin{adjustwidth}{1cm}{}
												If $\bigl(\emptyset, \{a\}\bigr)$ is added, then $\bigl(\{a\}, \emptyset\bigr)$ needs to be added
											\end{adjustwidth}
										\item[Examples] Two relations that meet these requirements are:
											\begin{align*}
												R_{1} &= \Bigl\{\bigl(\emptyset, \emptyset\bigr), \bigl(\{a\}, \{a\}\bigr), \bigl(\{b\}, \{b\}\bigr), \bigl(\{a, b\}, \{a, b\}\bigr)\Bigr\}\\
												R_{2} &= \Bigl\{\bigl(\emptyset, \emptyset\bigr), \bigl(\{a\}, \{a\}\bigr), \bigl(\{b\}, \{b\}\bigr), \bigl(\{a, b\}, \{a, b\}\bigr), \bigl(\emptyset, \{a\}\bigr), \bigl(\{a\}, \emptyset\bigr)\Bigr\}
											\end{align*}
									\end{description}
								\item \textbf{$R$ is reflexive and symmetric, but not transitive on $\mathcal{P}(A)$}
									\begin{description}
										\item[Reflexivity] To be reflexive, these pairs need to appear:
											\begin{align*}
												\bigl(\emptyset, \emptyset\bigr), \bigl(\{a\}, \{a\}\bigr), \bigl(\{b\}, \{b\}\bigr), \bigl(\{a, b\}, \{a, b\}\bigr)
											\end{align*}
										\item[Symmetry] Whichever pair is added, the pair that makes it symmetric needs to be added too.
											\begin{adjustwidth}{1cm}{}
												If $\bigl(\emptyset, \{a\}\bigr)$ is added, then $\bigl(\{a\}, \emptyset\bigr)$ needs to be added
											\end{adjustwidth}
										\item[Transitivity] In order for the relation to not be transitive, two elements need to be added (for symmetry) where the first element is the second element of another pair, and the second element is the first element of a different pair.
											\begin{adjustwidth}{1cm}{}
												If $\bigl(\{a\}, \{a, b\}\bigr)$ and $\bigl(\{a, b\}, \{a\}\bigr)$ are added.\\
												$\bigl(\emptyset, \{a\}\bigr) \rightarrow \bigl(\{a\}, \{a, b\}\bigr) \not \rightarrow \bigl(\emptyset, \{a, b\}\bigr)$
											\end{adjustwidth}
										\item[Example]
											\begin{align*}
												R_{3} &= \Bigl\{\bigl(\emptyset, \emptyset\bigr), \bigl(\{a\}, \{a\}\bigr), \bigl(\{b\}, \{b\}\bigr), \bigl(\{a, b\}, \{a, b\}\bigr),\\
												& \qquad \bigl(\emptyset, \{a\}\bigr), \bigl(\{a\}, \emptyset\bigr), \bigl(\{a\}, \{a, b\}\bigr), \bigl(\{a, b\}, \{a\}\bigr)\Bigr\}
											\end{align*}
									\end{description}
								\pagebreak
								\item \textbf{$R$ is reflexive and transitive, but not symmetric, and not antisymmetric on $\mathcal{P}(A)$}
									\begin{description}
										\item[Reflexivity] To be reflexive, these pairs need to appear:
											\begin{align*}
												\bigl(\emptyset, \emptyset\bigr), \bigl(\{a\}, \{a\}\bigr), \bigl(\{b\}, \{b\}\bigr), \bigl(\{a, b\}, \{a, b\}\bigr)
											\end{align*}
										\item[Symmetry] For the relation to not be symmetric, at least one pair cannot be flipped.
											\begin{adjustwidth}{1cm}{}
												If $\bigl(\emptyset, \{a\}\bigr)$ is added, then $\bigl(\{a\}, \emptyset\bigr)$ is not added.\\
												Adding this single element would still mean $R$ is transitive.
											\end{adjustwidth}
										\item[Antisymmetry] For the relation to not be antisymmetric, at least one pair can be flipped.
											If $\bigl(\emptyset, \{b\}\bigr)$ is added, then $\bigl(\{b\}, \emptyset\bigr)$ is added.
										\item[Example]
											\begin{align*}
												R_{4} &= \Bigl\{
													\bigl(\emptyset, \emptyset\bigr), \bigl(\{a\}, \{a\}\bigr), \bigl(\{b\}, \{b\}\bigr), \bigl(\{a, b\}, \{a, b\}\bigr),\\
													& \qquad \bigl(\emptyset, \{a\}\bigr), \bigl(\emptyset, \{b\}\bigr), \bigl(\{b\}, \emptyset\bigr)
												\Bigr\}
											\end{align*}
									\end{description}
								\item \textbf{$R$ is simultaneuosly symmetric and antisymmetric on $\mathcal{P}(A)$}
									\begin{description}
										\item[Antisymmetry] If there are no elements that are not equal to each other, then $R$ is vacuously antisymmetric.
										\item[Symmetry] If every element is equal to each other, then every element is symmetric with itself.
										\item[Example]
											\begin{align*}
												R_{5} &= \Bigl\{\bigl(\emptyset, \emptyset\bigr), \bigl(\{a\}, \{a\}\bigr), \bigl(\{b\}, \{b\}\bigr), \bigl(\{a, b\}, \{a, b\}\bigr)\Bigr\}
											\end{align*} 
									\end{description}
								\item \textbf{$R$ is irreflexive, antisymmetric and transitive on $\mathcal{P}(A)$}
									\begin{description}
										\item[Irreflexitivity] None of these pairs appear in $R$:
											\begin{align*}
												\bigl(\emptyset, \emptyset\bigr), \bigl(\{a\}, \{a\}\bigr), \bigl(\{b\}, \{b\}\bigr), \bigl(\{a, b\}, \{a, b\}\bigr)
											\end{align*}
										\item[Antisymmetry] No pairs $(x, y)$ and $(y, x)$ appear in $R$.
										\item[Transitivity] Can go from one pair to the next.
										\item[Example]
											\begin{align*}
												R_{6} &= \Bigl\{\bigl(\emptyset, \{a\}\bigr), \bigl(\{a\}, \{a, b\}\bigr), \bigl(\emptyset, \{a, b\}\bigr)\Bigr\}
											\end{align*}
									\end{description}
							\end{enumerate}
						\item \textbf{Prove that if $R$ is a relation on $X$, then $R$ is transitive iff $R \circ R \subseteq R$}.
							\begin{proof}
								$ $
								\begin{enumerate}[label=(\roman*)]
									\item \textbf{If $R$ is transitive, then $R \circ R \subseteq R$}
										\begin{subproof}[Subproof]
											$ $
											\begin{tabbing}
												Assume \qquad \= $R$ is transitive.\\
												Suppose \> $(x, z) \in R \circ R$.\\
												Then \> there is some $y \in X$ such that $(x, y) \in R$ and $(y, z) \in R$\\
												\> (By definition of composition)\\
												And \> $(x, z) \in R$\\
												\> (Because $R$ is transitive)\\
												$\therefore$ \> if $R$ is transitive, then $R \circ R \subseteq R$ \` \qedhere
											\end{tabbing}
										\end{subproof}
									\item \textbf{If $R \circ R \subseteq R$, then $R$ is transitive}
										\begin{subproof}[Subproof]
											$ $
											\begin{tabbing}
												Assume \qquad \= $R \circ R \subseteq R$.\\
												Suppose \> $(x, y) \in R$ and $(y, z) \in R$.\\
												Then \> $(x, z) \in R \circ R$.\\
												\> (By definition of composition)\\
												And \> $(x, z) \in R$\\
												\> (Because $R \circ R \subseteq R$)\\
												$\therefore$ \> if $(x, y) \in R$ and $(y, z) \in R$, then $(x, z) \in R$\\
												$\therefore$ \> if $R \circ R \subseteq R$, then $R$ is transitive. \` \qedhere
											\end{tabbing}
										\end{subproof}
								\end{enumerate}
								$\therefore R$ is transitive iff $R \circ R \subseteq R$.
							\end{proof}
					\end{enumerate}
				\end{exercise}
	\rulechapterend
\end{document}