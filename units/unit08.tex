\providecommand{\main}{..}
\documentclass[\main/notes.tex]{subfiles}

\begin{document}
	\ifSubfilesClassLoaded{\setcounter{chapter}{7}}{}
	\addtocontents{toc}{\protect\newpage}
	\chapter{Operations}
		\section{Binary Operation}
			\begin{definition}[width=0.9\textwidth]{Binary Operation}
				If $f: X \times X \rightarrow X$, then $f$ is called a \concept{binary operation} on $X$.\\
				In other words, a binary operation takes in a pair, and returns a single value.
			\end{definition}
			\begin{sidenote}{Operations Notation}
				An operation is just a function, which means it can be written in normal function \emph{prefix} notation: $f(x, y)$. However, it is more conventional to write it in \emph{infix} notation: $x f y$.
				\begin{example}
					Addition of numbers is a binary operation. If $(x, y) = (3, 4)$, then it could be written $+(3, 4)$, but it is more conventional to write $3 + 4$.
				\end{example}
			\end{sidenote}
			By convention, the elements of a binary operation are all the same set.
			\subsection[Finite and Infinite Sets]{Finite and Infinite Sets (Informal Definition)}
				\begin{definition}[width=0.8\textwidth]{Finite Set}
					A set whose cardinality is a non-negative number. Meaning one can count the number of elements in the set.\par
					\begin{example}[hbox]
						$A = \{1, 2, 3, 4\}$, where $\left\lvert A\right\rvert = 4$
					\end{example}
				\end{definition}
				\begin{definition}[width=0.8\textwidth]{Infinite Set}
					A set that is not finite. Meaning one \emph{cannot} count the number of elements in the set.\par
					\begin{example}[hbox]
						The set of real numbers $\mathbb{R}$ is an infinite set.
					\end{example}
				\end{definition}
			\pagebreak
			\subsection{Tables For Binary Operations}
				A way to describe a binary operation is to use a table, where the rows are based on the \emph{first} element, and the columns on the \emph{second}. The operator (the symbol used to describe the operation) is written in the top left corner.
				\begin{example}
					Let $A = \{a, b, c, d\}$.\\
					A binary operation called $+$ (NB: This is \emph{not} addition) could be written as follows:
					\begin{center}
						\begin{tblr}{colspec={|c|cccc|}, row{1}={font=\bfseries}, column{1}={font=\bfseries}}
							\toprule
							$+$ & a & b & c & d\\
							\midrule
							a & a & b & c & d \\
							b & b & c & d & a \\
							c & c & d & a & b \\
							d & d & a & b & c \\
							\bottomrule
						\end{tblr}
					\end{center}
					This would be read (row, column). +(b, d) = a.
					\begin{center}
						\begin{tblr}{colspec={|c|cccc|}, row{1}={font=\bfseries}, column{1}={font=\bfseries}, row{3}={Lavender}, column{5}={Lavender}, cell{3}{5}={Orchid}}
							\toprule
							$+$ & a & b & c & d\\
							\midrule
							a & a & b & c & d \\
							b & b & c & d & a \\
							c & c & d & a & b \\
							d & d & a & b & c \\
							\bottomrule
						\end{tblr}
					\end{center}
					\begin{sidenote}{Extra Notes for this operation}
						Applying concepts from later to the operation:
						\begin{description}[nosep]
							\item[Identity] This operation has an identity element, which is a.
							\item[Commutativity] This operation is commutative.
							\item[Associativity] This operation is associative.
						\end{description}
					\end{sidenote}
					Another binary operation, called $\bullet$ could be written as follows:
					\begin{center}
						\begin{tblr}{colspec={|c|c c c c|}, row{1}={font=\bfseries}, column{1}={font=\bfseries}}
							\toprule
							$\bullet$ & a & b & c & d\\
							\midrule
							a & a & b & c & d \\
							b & b & a & d & c \\
							c & c & d & a & b \\
							d & d & c & b & a \\
							\bottomrule
						\end{tblr}
					\end{center}
					\begin{sidenote}{Extra Notes for this operation}
						Applying concepts from later to the operation:
						\begin{description}[nosep]
							\item[Identity] This operation has an identity element, which is a.
							\item[Commutativity] This operation is commutative.
							\item[Associativity] This operation is associative.
						\end{description}
					\end{sidenote}
				\end{example}
			\pagebreak
		\section{Properties of Binary Operations}
			For examples below, the following binary operation has been used:
			\begin{indentparagraph}
				$*: \{1, 2\} \times \{1, 2\} \rightarrow \{1, 2\}$ is defined by:
				\begin{align*}
					\Bigl\{\bigl((1, 1), 1\bigr), \bigl((1, 2), 2\bigr), \bigl((2, 1), 2\bigr), \bigl((2, 2), 1\bigr)\Bigr\}
				\end{align*}
				In table form this would be:
				\begin{center}
					\begin{tblr}{colspec={|c|c c|}, row{1}={font=\bfseries}, column{1}={font=\bfseries}}
						\toprule
						$*$ & $1$ & $2$\\
						\midrule
						$1$ & 1 & 2 \\
						$2$ & 2 & 1 \\
						\bottomrule
					\end{tblr}
				\end{center}
			\end{indentparagraph}
			\subsection{Commutative Binary Operation}
				\begin{definition}{Commutativity}
					A binary operation $\Diamond: X \times X \rightarrow X$ is \concept{commutative} iff $x \Diamond y = y \Diamond x$ for all $x, y \in X$.
				\end{definition}
				The easiest way to check this is it will be commutative if it is symmetrical across the diagonal from the top left to the bottom right.
				\begin{example}[width=0.35\textwidth] $
					\begin{aligned}[t]
						1 * 1 = 1 * 1 = 1\\
						1 * 2 = 2 * 1 = 2\\
						2 * 2 = 2 * 2 = 1
					\end{aligned} $\\
					Therefore $*$ is commutative.
				\end{example}
			\subsection{Associative Binary Operation}
				\begin{definition}{Associativity}
					A binary operation $\Diamond: X \times X \rightarrow X$ is \concept{associative} iff $(x \Diamond y) \Diamond z = x \Diamond (y \Diamond z)$ for all $x, y, z \in X$.
				\end{definition}
				Unfortunately, for this one, you have to check each instance.
				\begin{example}[width=0.65\textwidth] $
					\begin{aligned}[t]
						(1 * 1) * 1 = 1 * 1 = 1 &\text{ and } 1 * (1 * 1) = 1 * 1 = 1\\
						(1 * 1) * 2 = 1 * 2 = 2 &\text{ and } 1 * (1 * 2) = 1 * 2 = 2\\
						(1 * 2) * 1 = 2 * 1 = 2 &\text{ and } 1 * (2 * 1) = 2 * 1 = 2\\
						(1 * 2) * 2 = 2 * 2 = 1 &\text{ and } 1 * (2 * 2) = 1 * 1 = 1\\
						(2 * 1) * 1 = 2 * 1 = 2 &\text{ and } 2 * (1 * 1) = 2 * 1 = 2\\
						(2 * 1) * 2 = 2 * 2 = 1 &\text{ and } 2 * (1 * 2) = 2 * 2 = 1\\
						(2 * 2) * 1 = 1 * 1 = 1 &\text{ and } 2 * (2 * 1) = 2 * 2 = 1\\
						(2 * 2) * 2 = 1 * 2 = 2 &\text{ and } 2 * (2 * 2) = 2 * 1 = 2
					\end{aligned} $\\
					Therefore $*$ is associative.
				\end{example}
			\pagebreak
			\subsection{Identity Element of a Binary Operation}
				\begin{definition}{Identity Element}
					An element $e$ of $X$ is an \concept{identity element} in respect of the binary operation $\Diamond: X \times X \rightarrow X$ iff $e \Diamond x = x \Diamond e = x$ for all $x \in X$.
				\end{definition}
				The easiest way to check this is if there is a row and column in the table that is identical to the header. (NB: It needs to be \emph{both} row and column, which contain the same element.)
				\begin{example}[hbox] $
					\begin{aligned}[t]
						\mathbf{1} * 1 = 1 &\text{ and } 1 * \mathbf{1} = 1\\
						1 * \mathbf{2} = 2 &\text{ and } 2 * \mathbf{1} = 2
					\end{aligned} $
				\end{example}
				\begin{exercise}{Self Assessment Exercise \thechapter.3}
					\begin{enumerate}
						\item \question{Let $X$ be $\{2, 7\}$.}
							\begin{enumerate}[label=(\alph*)]
								\item \question{Provide 3 binary operations on $X$, both in list notation and in tabular form.}
									\begin{align*}
										\triangle = \Bigl\{\bigl((2, 2), 2\bigr), \bigl((2, 7), 2\bigr), \bigl((7, 2), 2\bigr), \bigl((7, 7), 7\bigr)\Bigr\}
									\end{align*}
									\begin{center}
										\begin{tblr}{colspec={|c | c c|}, row{1}={font=\bfseries}, column{1}={font=\bfseries}}
											\toprule
											$\triangle$ & $2$ & $7$\\
											\midrule
											$2$ & 2 & 2\\
											$7$ & 2 & 7\\
											\bottomrule
										\end{tblr}
									\end{center}
									\begin{align*}
										\triangledown = \Bigl\{\bigl((2, 2), 2\bigr), \bigl((2, 7), 7\bigr), \bigl((7, 2), 7\bigr), \bigl((7, 7), 7\bigr)\Bigr\}
									\end{align*}
									\begin{center}
										\begin{tblr}{colspec={|c | c c|}, row{1}={font=\bfseries}, column{1}={font=\bfseries}}
											\toprule
											$\triangledown$ & $2$ & $7$\\
											\midrule
											$2$ & 2 & 7\\
											$7$ & 7 & 7\\
											\bottomrule
										\end{tblr}
									\end{center}
									\begin{align*}
										\square = \Bigl\{\bigl((2, 2), 2\bigr), \bigl((2, 7), 2\bigr), \bigl((7, 2), 7\bigr), \bigl((7, 7), 7\bigr)\Bigr\}
									\end{align*}
									\begin{center}
										\begin{tblr}{colspec={|c | c c|}, row{1}={font=\bfseries}, column{1}={font=\bfseries}}
											\toprule
											$\square$ & $2$ & $7$\\
											\midrule
											$2$ & 2 & 2\\
											$7$ & 7 & 7\\
											\bottomrule
										\end{tblr}
									\end{center}
								\pagebreak
								\item \question{Check the three operations for commutativity and associativity.}
									\begin{description}
										\item[Commutativity] $\triangle$ is commutative, as it is symmetric about the diagonal.\\
											$\triangledown$ is commutative, as it is symmetric about the diagonal.\\
											$\square$ is \emph{not} commutative.
										\item[Associativity] $\triangle$ is associative.
											\begin{alignat*}{3}
												x = 2, y = 2, z = 2 \qquad & (2 \triangle 2) \triangle 2 &= 2 \qquad & 2 \triangle (2 \triangle 2) &= 2\\
												x = 2, y = 2, z = 7 \qquad & (2 \triangle 2) \triangle 7 &= 2 \qquad & 2 \triangle (2 \triangle 7) &= 2\\
												x = 2, y = 7, z = 2 \qquad & (2 \triangle 7) \triangle 2 &= 2 \qquad & 2 \triangle (7 \triangle 2) &= 2\\
												x = 2, y = 7, z = 7 \qquad & (2 \triangle 7) \triangle 7 &= 2 \qquad & 2 \triangle (7 \triangle 7) &= 2\\
												x = 7, y = 2, z = 2 \qquad & (7 \triangle 2) \triangle 2 &= 2 \qquad & 7 \triangle (2 \triangle 2) &= 2\\
												x = 7, y = 2, z = 7 \qquad & (7 \triangle 2) \triangle 7 &= 2 \qquad & 7 \triangle (2 \triangle 7) &= 2\\
												x = 7, y = 7, z = 2 \qquad & (7 \triangle 7) \triangle 2 &= 2 \qquad & 7 \triangle (7 \triangle 2) &= 2\\
												x = 7, y = 7, z = 7 \qquad & (7 \triangle 7) \triangle 7 &= 7 \qquad & 7 \triangle (7 \triangle 7) &= 7
											\end{alignat*}
											Doing the same for $\triangledown$ and $\square$. Both are associative as well.
									\end{description}
								\item \question{Provide $2$ binary operations on $X = \{a, b, c\}$ and check them for commutativity and associativity.}
									\begin{center}
										\begin{tblr}{colspec={| c | c c c |}, row{1}={font=\bfseries}, column{1}={font=\bfseries}}
											\toprule
											$\bigstar$ & a & b & c\\
											\midrule
											a & a & a & a\\ 
											b & b & b & b\\ 
											c & c & c & c\\
											\bottomrule
										\end{tblr}
										\begin{tblr}{colspec={| c | c c c |}, row{1}={font=\bfseries}, column{1}={font=\bfseries}}
											\toprule
											$\heartsuit$ & a & b & c\\
											\midrule
											a & a & a & a\\ 
											b & a & a & a\\ 
											c & a & a & a\\
											\bottomrule
										\end{tblr}
									\end{center}
									\begin{description}
										\item[Commutativity] $\bigstar$ is not commutative.\\
											$\heartsuit$ is not commutative.
										\item[Associativity] $\bigstar$ is associative.\\
											$\heartsuit$ is associative.
									\end{description}
							\end{enumerate}
						\item \question{Consider the $\bullet$ operation defined in the example above on $A = \{a, b, c, d\}$}
							\begin{enumerate}
								\item \question{Examine $y \bullet x$ and $x \bullet y$ for each $x, y \in A$. Is $\bullet$ commutative?}\\
									Yes, as it is symmetric about the diagonal.
								\item \question{Does $A$ have an identity element for $\bullet$?}\\
									Yes, as the row and column for $a$ matches the headers. So $a$ is an identity element.
							\end{enumerate}
					\end{enumerate}
				\end{exercise}
		\pagebreak
		\section{Operations on Vectors}
			\subsection{Vector}
				\begin{definition}{Vector}
					In this course, a \concept{vector} is considered to be an ordered \emph{n-tuple} of numbers.\\
					A \concept{vector} is represented by an n-tuple $u$ in the following way:
					\begin{align*}
						u = (u_{1}, u_{2}, u_{3}, \ldots, u_{n}) \text{ for some } n \geq 2
					\end{align*}
				\end{definition}
			\subsection{Vector Sum}
				\begin{definition}{Vector Sum}
					If $u$ and $v$ are vectors with the \emph{same number of coordinates}, then their \concept{sum}, written $u + v$ is the vector obtained by adding the corresponding coordinates of $u$ and $v$.
					\begin{align*}
						u + v &= (u_{1}, u_{2}, \ldots, u_{n}) + (v_{1}, v_{2}, \ldots, v_{n})\\
						&= (u_{1} + v_{2}, u_{2} + v_{2}, \ldots, u_{n} + v_{n})
					\end{align*}
				\end{definition}
				\begin{example}[width=0.5\textwidth]
					Let $u = (1, 2, 3)$ and $v = (4, 5, 6)$.\\
					Then 
					\begin{align*}
						u + v &= (1, 2, 3) + (4, 5, 6)\\
						&= (1 + 4, 2 + 5, 3 + 6)\\
						&= (5, 7, 9)
					\end{align*} 
				\end{example}
				\begin{sidenote}{Vector addition is not defined for vectors of different sizes}
					If two vectors have a different number of coordinates, you cannot add those two vectors together.
				\end{sidenote}
			\pagebreak
			\subsection{Scalar-Vector Product}
				\begin{definition}{Scalar-Vector Product}
					If $u$ is a vector and $r$ is some scalar number, then the \concept{product} of the number $r$ and the vector $u$ is the vector $r \cdot u$ obtained by multiplying each coordinate of $u$ by $r$.
					\begin{align*}
						r\cdot u &= r(u_{1}, u_{2}, \ldots, u_{n})\\
						&= (ru_{1}, ru_{2}, \ldots, ru_{n})
					\end{align*}
				\end{definition}
				\begin{example}[width=0.5\textwidth]
					Let $u = (7, 8, 9)$ and $r = 2$.\\
					Then
					\begin{align*}
						r \cdot u &= 2(7, 8, 9)\\
						&= (14, 16, 18)
					\end{align*}
				\end{example}
				\begin{exercise}{Self Assessment Exercise \thechapter.6}
					\question{If $u = (3, 1)$, $v = (-4, -4)$ and $w = (0, -1)$, determine}
					\begin{enumerate}[label=(\alph*)]
						\item \question{$2u + v$} \hfill
							$ \begin{aligned}[t]
								2u + v &= 2(3, 1) + (-4, -4)\\
								&= (6, 2) + (-4, -4)\\
								&= (2, -2)
							\end{aligned} $ \hfill \phantom{}
						\item \question{$u - 3v$} \hfill
							$ \begin{aligned}[t]
								u - 3v &= (3, 1) - 3(-4, -4)\\
								&= (3, 1) + (12, 12)\\
								&= (15, 13)
							\end{aligned} $ \hfill \phantom{}
						\item \question{$-3(v + w)$} \hfill
							$ \begin{aligned}[t]
								-3(v + w) &= -3\bigl((-4, -4) + (0, -1)\bigr)\\
								&= -3(-4, -5)\\
								&= (12, 15)
							\end{aligned} $ \hfill \phantom{}
					\end{enumerate}
				\end{exercise}
			\pagebreak
			\subsection{Dot Product}
				\begin{definition}[width=0.87\textwidth]{Dot Product}
					The \concept{dot product} (also called the \concept{inner product}) of vectors $u = (u_{1}, u_{2}, \ldots, u_{n})$ and \\$v = (v_{1}, v_{2}, \ldots, v_{n})$ is written $u \cdot v$ and defined by:
					\begin{align*}
						u \cdot v = u_{1}v_{1} + u_{2}v_{2} + \ldots + u_{n}v_{n}
					\end{align*}
				\end{definition}
				\begin{sidenote}{The result of the dot product is a number}
					Unlike the other operations, which result in vectors, the dot product produces a single number.
				\end{sidenote}
				\begin{example}[width=0.6\textwidth]
					Let $u = (2, 4, 6)$ and $v = (1, 3, 5)$. Then
					\begin{align*}
						u \cdot v &= (2, 4, 6)(1, 3, 5)\\
						&= (2 \cdot 1) + (4 \cdot 3) + (6 \cdot 5)\\
						&= 2 + 12 + 30\\
						&= 44
					\end{align*}
				\end{example}
				\begin{sidenote}{The dot product is not defined for vectors of different sizes}
					As with addition, if two vectors have a different number of coordinates, you cannot calculate the dot product.
				\end{sidenote}
			\begin{exercise}{Self Assessment Exercise \thechapter.7}
				\question{If $u = (1, 2, 5)$ and $v = (2, 3, 5)$, determine}
				\begin{enumerate}[label=(\alph*)]
					\item \question{$u \cdot v$} \hfill
						$ \begin{aligned}[t]
							u \cdot v &= (1, 2, 5) \cdot (2, 3, 5)\\
							&= (1 \cdot 2) + (2 \cdot 3) + (5 \cdot 5)\\
							&= 2 + 6 + 25\\
							&= 33
						\end{aligned} $\hfill \phantom{}
					\item \question{$v(2u)$} \hfill $
						\begin{aligned}[t]
							v(2u) &= (2, 3, 5)\bigl(2(1, 2, 5)\bigr)\\
							&= (2, 3, 5) \cdot (2, 4, 10)\\
							&= (2 \cdot 2) + (3 \cdot 4) + (5 \cdot 10)\\
							&= 4 + 12 + 50\\
							&= 66
						\end{aligned} $ \hfill \phantom{}
				\end{enumerate}
			\end{exercise}
		\pagebreak
		\section{Operations on Matrices}
			\subsection{Matrix}
				\begin{definition}{Matrix}
					A \concept{matrix} is an array of numbers organised into rows and columns, and enclosed within brackets.\\
					The number of rows is written with the letter $m$ and the number of columns with the letter $n$. So a matrix is said to have the size $m \times n$.
				\end{definition}
				\begin{example}[hbox]
					$\begin{bmatrix}
						3 & 2\\
						1 & 5
					\end{bmatrix}$ is a $2 \times 2$ matrix, and
					$\begin{bmatrix}
						-1 & 3 & 0 & 5\\
						0 & 2 & 0 & 6\\
						1 & - 1& 0 & 13
					\end{bmatrix}$ is a $3 \times 4$ matrix.
				\end{example}
				Matrices (pronounced \emph{may-trisseez}) have the form 
				\begin{align*}
					\begin{bmatrix}
						a_{11} & a_{12} & \cdots & a_{1n}\\
						a_{21} & a_{22} & \cdots & a_{2n}\\
						\vdots & \vdots & & \vdots\\
						a_{m1} & a_{m2} & \cdots & a_{mn}
					\end{bmatrix}
				\end{align*}
			\subsection{Matrix Addition}
				\begin{definition}{Matrix Addition}
					Let $A$ and $B$ be two matrices of the same size. Then the matrix \concept{$A + B$} is:
					\begin{align*}
						A + B &= \begin{bmatrix}
							a_{11} & a_{12} & \cdots & a_{1n}\\
							a_{21} & a_{22} & \cdots & a_{2n}\\
							\vdots & \vdots & & \vdots\\
							a_{m1} & a_{m2} & \cdots & a_{mn}
						\end{bmatrix} + \begin{bmatrix}
							b_{11} & b_{12} & \cdots & b_{1n}\\
							b_{21} & b_{22} & \cdots & b_{2n}\\
							\vdots & \vdots & & \vdots\\
							b_{m1} & b_{m2} & \cdots & b_{mn}
						\end{bmatrix}\\
						&= \begin{bmatrix}
							a_{11} + b_{11} & a_{12} + b_{12} & \cdots & a_{1n} + b_{1n}\\
							a_{21} + b_{21} & a_{22} + b_{22} & \cdots & a_{2n} + b_{2n}\\
							\vdots & \vdots & & \vdots\\
							a_{m1} + b_{m1} & a_{m2} + b_{m2} & \cdots & a_{mn} + b_{mn}
						\end{bmatrix}
					\end{align*}
				\end{definition}
				\begin{example}[width=0.5\textwidth]
					Let $A = \begin{bmatrix}
						1 & 2\\
						3 & 4
					\end{bmatrix}$ and 
					$B = \begin{bmatrix}
						5 & 6\\
						7 & 8
					\end{bmatrix}$
					\begin{align*}
						A + B &= \begin{bmatrix}
							1 & 2\\
							3 & 4
						\end{bmatrix} + \begin{bmatrix}
							5 & 6\\
							7 & 8
						\end{bmatrix}\\
						&= \begin{bmatrix}
							1 + 5 & 2 + 6\\
							3 + 7 & 4 + 8
						\end{bmatrix}\\
						&= \begin{bmatrix}
							6 & 8\\
							10 & 12
						\end{bmatrix}
					\end{align*}
				\end{example}
				\begin{exercise}{Self Assessment Exercise \thechapter.8}
					\question{For each pair $A$ and $B$ determine $A + B$ (if possible):}
					\begin{multicols}{2}
						\begin{enumerate}[label=(\alph*)]
							\item \question{$A = \begin{bmatrix}-1 & 0\\ 0 & 1\end{bmatrix}$ and $B = \begin{bmatrix}5 & 5 \\ 4 & -1\end{bmatrix}$}\\
								$ \begin{aligned}[t]
									A + B &= \begin{bmatrix}
										-1 & 0 \\ 0 & 1
									\end{bmatrix} + \begin{bmatrix}
										5 & 5 \\ 4 & -1
									\end{bmatrix}\\
									&= \begin{bmatrix}
										4 & 5\\ 4 & 0
									\end{bmatrix}
								\end{aligned} $
							\item \question{$A = \begin{bmatrix}-1 & 0\\ 0 & 1\end{bmatrix}$ and $B = \begin{bmatrix}3 & 2 & 0 \\ 1 & 0 & 0\end{bmatrix}$}\\
								This operation is not defined, as the matrices are different sizes.
							\item \question{$A = \begin{bmatrix}2 & 0 & 3\\ 0 & 7 & 1\end{bmatrix}$ and $B = \begin{bmatrix}1 & 1 & -2 \\ 2 & 0 & 6\end{bmatrix}$}\\
								$ \begin{aligned}[t]
									A + B &= \begin{bmatrix}
										2 & 0 & 3 \\ 0 & 7 & 1
									\end{bmatrix} + \begin{bmatrix}
										1 & 1 & -2 \\ 2 & 0 & 6
									\end{bmatrix}\\
									&= \begin{bmatrix}
										3 & 1 & 1 \\ 2 & 7 & 7
									\end{bmatrix}
								\end{aligned} $
							\item \question{$A = \begin{bmatrix}3 & 1\\ -2 & -5\end{bmatrix}$ and $B = \begin{bmatrix}1 & 2 & 0 \\ 2 & 5 & 1\end{bmatrix}$}\\
								This operation is not defined, as the matrices are different sizes.
						\end{enumerate}
					\end{multicols}
				\end{exercise}
			\subsection{Scalar-Matrix Multiplication}
				\begin{definition}[width=0.7\textwidth]{Scalar-Matrix Multiplication}
					Let $A$ be a matrix, and $r$ be some scalar number.\\
					Then the product \concept{$rA$} is defined as:
					\begin{align*}
						rA = r \begin{bmatrix}
							a_{11} & a_{12} & \cdots & a_{1n}\\
							a_{21} & a_{22} & \cdots & a_{2n}\\
							\vdots & \vdots & & \vdots\\
							a_{m1} & a_{m2} & \cdots & a_{mn}
						\end{bmatrix}
						= \begin{bmatrix}
							ra_{11} & ra_{12} & \cdots & ra_{1n}\\
							ra_{21} & ra_{22} & \cdots & ra_{2n}\\
							\vdots & \vdots & & \vdots\\
							ra_{m1} & ra_{m2} & \cdots & ra_{mn}
						\end{bmatrix}
					\end{align*}
				\end{definition}
				\begin{example}[width=0.42\textwidth]
					Let $r = 3$ and $A = \begin{bmatrix}
						1 & 2\\
						3 & 4
					\end{bmatrix}$
					\begin{align*}
						rA &= 3\begin{bmatrix}
							1 & 2\\
							3 & 4
						\end{bmatrix}
						= \begin{bmatrix}
							3 & 6\\
							9 & 12
						\end{bmatrix}
					\end{align*}
				\end{example}
				\begin{exercise}{Self Assessment Exercise \thechapter.9}
					\question{Perform the indicated operation:}
					\begin{align*}
						2\begin{bmatrix}
							-1 \\ 2 \\ 3
						\end{bmatrix} - 3 \begin{bmatrix}
							2 \\ 1 \\ 0
						\end{bmatrix} + 4 \begin{bmatrix}
							2 \\ 1 \\ 5
						\end{bmatrix} &= \begin{bmatrix}
							-2 \\ 4 \\ 6
						\end{bmatrix} + \begin{bmatrix}
							-6 \\ -3 \\ 0
						\end{bmatrix} + \begin{bmatrix}
							8 \\ 4 \\ 20
						\end{bmatrix} = \begin{bmatrix}
							0 \\ 5 \\ 26
						\end{bmatrix}
					\end{align*}
				\end{exercise}
			\pagebreak
			\subsection{Matrix Multiplication}
				\begin{definition}{Matrix Multiplication}
					Let $A$ and $B$ both be matrices.
					In order for the product \concept{$AB$} to be defined,
					\begin{itemize}
						\item The number of columns of $A$ needs to be equal to the number of rows of $B$, i.e. $A_{n} = B_{m}$.
					\end{itemize}
					If the product is defined, then it will result in a matrix that is the size $A_{m} \times B_{n}$.
					\begin{center}
						\tikz[baseline=(am.base)]\node[inner xsep=0pt] (am) {$A_{m}$}; $\times$ \tikz[baseline=(an.base)]\node[inner xsep=0pt] (an) {$A_{n}$}; $\quad \cdot \quad$ \tikz[baseline=(bm.base)]\node[inner xsep=0pt] (bm) {$B_{m}$}; $\times$ \tikz[baseline=(bn.base)]\node[inner xsep=0pt] (bn) {$B_{n}$}; $= A_{m} \times B_{n}$
					\end{center}
					\begin{tikzpicture}[overlay]
						\draw[<->] (an.south) -- ++(0,-1.5ex) -| (bm.south);
						\draw[<->] (am.south) -- ++(0,-2.5ex) -| (bn.south);
					\end{tikzpicture}\\
					When multiplying matrics, it is row of first multiplied by column of second.
				\end{definition}
				\begin{example}[width=0.76\textwidth] $
					\begin{aligned}[t]
						\begin{bmatrix}
							-1 & 3\\
							4 & 2\\
							5 & -7
						\end{bmatrix}\begin{bmatrix}
							6 & 9\\
							-8 & 1
						\end{bmatrix} &= \begin{bmatrix}
							(-1 \cdot 6) + (3 \cdot -8) & (-1 \cdot 9) + (3 \cdot 1)\\
							(4 \cdot 6) + (2 \cdot -8) & (4 \cdot 9) + (2 \cdot 1) \\
							(5 \cdot 6) + (-7 \cdot -8) & (5 \cdot 9) + (-7 \cdot 1)
						\end{bmatrix}\\
						&= \begin{bmatrix}
							-6 - 24 & -9 + 3\\
							24 - 16 & 36 + 2\\
							30 + 56 & 45 - 7
						\end{bmatrix}\\
						&= \begin{bmatrix}
							-30 & -6\\
							8 & 38\\
							86 & 38
						\end{bmatrix}
					\end{aligned} $
				\end{example}
			\subsection{Identity Matrix}
				\begin{definition}{Identity Matrix}
					If $A$ is matrix, then an \concept{identity matrix} $I$ with respect to $A$ is a matrix such that $IA = AI = A$.\\
				For the identity matrix to be defined,
				\begin{itemize}
					\item $A$ must be a square matrix, because matrix multiplication is \emph{not} commutative.
					\item $I$ must therefore also be a square matrix with the same number of rows and columns as $A$.
				\end{itemize}
				Then the identity matrix would have $1$'s for the main diagonal, and $0$'s elsewhere.
				\end{definition}
				\begin{example}[width=0.7\textwidth]
					For a $2 \times 2$ matrix, the identity matrix would be:
						\begin{align*}
							I = \begin{bmatrix}
								1 & 0\\
								0 & 1
							\end{bmatrix}
						\end{align*}
				\end{example}
				\pagebreak
				\begin{exercise}{Self Assessment Exercise \thechapter.10}
					\begin{enumerate}
						\item \question{Perform the indicated matrix operations (if possible)}
							\begin{enumerate}[label=(\alph*)]
								\item 
									$ \begin{aligned}[t]
										\bm{{\begin{bmatrix}
											31 & -3 & 2\\
											2 & 5 & 1\\
											3 & 0 & 0
										\end{bmatrix} \begin{bmatrix}
											0 \\ 1 \\ 5
										\end{bmatrix}}} &= \begin{bmatrix}
											(31 \cdot 0) + (-3 \cdot 1) + (2 \cdot 5)\\
											(2 \cdot 0) + (5 \cdot 1) + (1 \cdot 5)\\
											(3 \cdot 0) + (0 \cdot 1) + (0 \cdot 5)
										\end{bmatrix}
										= \begin{bmatrix}
											0 - 3 + 10\\
											0 + 5 + 5\\
											0 + 0 + 0
										\end{bmatrix}
										= \begin{bmatrix}
											7\\
											10\\
											0
										\end{bmatrix}
									\end{aligned} $
								\item \question{$\begin{bmatrix}9 & 3\\ 1 & 5\\ 3 & 0\end{bmatrix}\begin{bmatrix}1 & 0\\ 2 & 4\\ 5 & 1\end{bmatrix}$}\\
									This operation is not defined.
								\item 
								$ \begin{aligned}[t]
									&\bm{{\begin{bmatrix}
										1 & -3 & 2\\
										0 & 6 & 4\\
										3 & 0 & 3
									\end{bmatrix}\begin{bmatrix}
										0 & -1 & 3\\
										1 & \frac{1}{3} & 1\\
										\frac{1}{2} & 5 & 0
									\end{bmatrix}}}\\ 
									= \, &\begin{bmatrix}
										(1 \cdot 0) + (-3 \cdot 1) + \left(2 \cdot \frac{1}{2}\right) & (1 \cdot - 1) + \left(-3 \cdot \frac{1}{3}\right) + (2 \cdot 5) & (1 \cdot 3) + (-3 \cdot 1) + (2 \cdot 0)\\
										(0 \cdot 0) + (6 \cdot 1) + \left(4 \cdot \frac{1}{2}\right) & (0 \cdot -1) + \left(6 \cdot \frac{1}{3}\right) + (4 \cdot 5) & (0 \cdot 3) + (6 \cdot 1) + (4 \cdot 0)\\
										(3 \cdot 0) + (0 \cdot 1) + \left(3 \cdot \frac{1}{2}\right) & (3 \cdot -1) + \left(0 \cdot \frac{1}{3}\right) + (3 \cdot 5) & (3 \cdot 3) + (0 \cdot 1) + (3 \cdot 0)
									\end{bmatrix}\\
									= \, &\begin{bmatrix}
										(0 - 3 + 1) & (- 1 - 1 + 10) & (3 - 3 + 0)\\
										(0 + 6 + 2) & (0 + 2 + 20) & (0 + 6 + 0)\\
										\left(0 + 0 + \frac{3}{2}\right) & (-3 + 0 + 15) & (9 + 0 + 0)
									\end{bmatrix}\\
									= \, &\begin{bmatrix}
										-2 & 8 & 0\\
										8 & 22 & 6\\
										\frac{3}{2} & 12 & 9
									\end{bmatrix}
								\end{aligned} $
							\end{enumerate}
						\item \question{Provide examples of matrices $X$ and $Y$ such that $XY$ is a $3 \times 3$ matrix, but $YX$ is a $2 \times 2$~matrix.}\\
							Any matrices $X$ and $Y$ such that $X$ is a $2 \times 3$ matrix, and $Y$ is a $3 \times 2$ matrix.\\
							Two examples: \quad
							$ \begin{aligned}[t]
								A_{1} = \begin{bmatrix}
									1 & 2 & 3\\
									4 & 5 & 6
								\end{bmatrix} \text{ and }
								B_{1} = \begin{bmatrix}
									1 & 2\\
									3 & 4\\
									5 & 6
								\end{bmatrix}\\
								A_{2} = \begin{bmatrix}
									2 & 4 & 6\\
									8 & 6 & 4
								\end{bmatrix} \text{ and }
								B_{2} = \begin{bmatrix}
									3 & 6\\
									9 & 6\\
									3 & 0
								\end{bmatrix}
							\end{aligned} $
						\item \question{Provide examples of matrices $X$ and $Y$ such that both $X$ and $Y$ have at least some non-zero entries, but $XY$ is the $2 \times 2$~zero~matrix.}\\
							Any matrix $A$ that has a zero column, where matrix $B$ has a zero row that are at different indexes.\\
							Example: \quad $
							\begin{aligned}[t]
								A = \begin{bmatrix}
									0 & 4\\
									0 & 5
								\end{bmatrix} \text{ and }
								B = \begin{bmatrix}
									6 & 8\\
									0 & 0
								\end{bmatrix}
							\end{aligned}$
						\pagebreak
						\item \question{Prove that addition is a commutative operation on the set of $2 \times 2$~matrices, and that there is a $2 \times 2$~matrix that acts as an identity element in respect of addition.}
							\begin{proof}
								Let $A$ and $B$ be two $2 \times 2$~matrices, where $A = \begin{bmatrix}a & b\\ c & d\end{bmatrix}$ and $B = \begin{bmatrix}e & f\\g & h\end{bmatrix}$.
								\begin{subproof}[Commutativity]
									Then: $ \begin{aligned}[t]
										A + B &= \begin{bmatrix}a & b\\ c & d\end{bmatrix} + \begin{bmatrix}e & f\\g & h\end{bmatrix}\\
										&= \begin{bmatrix}
											a + e & b + f\\
											c + g & d + h
										\end{bmatrix}\\
										B + A &= \begin{bmatrix}e & f\\g & h\end{bmatrix} + \begin{bmatrix}a & b\\ c & d\end{bmatrix}\\
										&= \begin{bmatrix}
											e + a & f + b\\
											g + c & h + d
										\end{bmatrix} = \begin{bmatrix}
											a + e & b + f\\
											c + g & d + h
										\end{bmatrix}
									\end{aligned}$\\
									As $A + B = B + A$, matrix addition is commutative.
								\end{subproof}
								\begin{subproof}[Identity]
									The identity element for matrix addition on $2 \times 2$~matrices is $\begin{bmatrix}0 & 0 \\ 0 & 0\end{bmatrix}$
								\end{subproof}
								So, matrix addition is commutative, and an identity element exists for matrix addition.
							\end{proof}
						\item \question{Prove that multiplication is \emph{not} a commutative operation on the set of $2 \times 2$ matrices, and that there is a $2 \times 2$~matrix that acts as an identity element in respect of multiplication.}
							\begin{proof}
								$ $
								\begin{subproof}[Commutativity Counterexample] Let 
									$A = \begin{bmatrix}1 & 2\\ 3 & 4
								\end{bmatrix}$ and $B = \begin{bmatrix}4 & 3 \\ 2 & 1\end{bmatrix}$
									\begin{align*}
										AB &= \begin{bmatrix}
											1 & 2 \\ 3 & 4
										\end{bmatrix}\begin{bmatrix}
											4 & 3 \\ 2 & 1
										\end{bmatrix} = \begin{bmatrix}
											(1 \cdot 4) + (2 \cdot 2) & (1 \cdot 3) + (2 \cdot 1)\\
											(3 \cdot 4) + (4 \cdot 2) & (3 \cdot 3) + (4 \cdot 1)
										\end{bmatrix} = \begin{bmatrix}
											4 + 4 & 3 + 2\\
											12 + 8 & 9 + 4
										\end{bmatrix} = \begin{bmatrix}
											8 & 5\\
											20 & 13
										\end{bmatrix}\\
										BA &= \begin{bmatrix}
											4 & 3 \\ 2 & 1
										\end{bmatrix} \begin{bmatrix}
											1 & 2 \\ 3 & 4
										\end{bmatrix} = \begin{bmatrix}
											(4 \cdot 1) + (3 \cdot 3) & (4 \cdot 2) + (3 \cdot 4)\\
											(2 \cdot 1) + (1 \cdot 3) & (2 \cdot 2) + (1 \cdot 4)
										\end{bmatrix} = \begin{bmatrix}
											4 + 9 & 8 + 12\\
											2 + 3 & 4 + 4
										\end{bmatrix} = \begin{bmatrix}
											13 & 20\\
											5 & 8
										\end{bmatrix}
									\end{align*}
									As $AB \neq BA$, matrix multiplication is not commutative.
								\end{subproof}
								\begin{subproof}[Identity]
									The identity element for matrix multiplication is the identity matrix. For a $2 \times 2$ matrix, that is $\begin{bmatrix}1 & 0 \\ 0 & 1\end{bmatrix}$
								\end{subproof}
								So, matrix multiplication is not commutative, and an identity element exists for matrix multiplication.
							\end{proof}
					\end{enumerate}
				\end{exercise}
	\rulechapterend
\end{document}
