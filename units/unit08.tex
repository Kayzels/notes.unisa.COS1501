\documentclass[../notes.tex]{subfiles}

\begin{document}
	\ifSubfilesClassLoaded{\setcounter{chapter}{7}}{}
	\chapter{Operations}
		\section{Binary Operation}
			\begin{definition}{Binary Operation}
				If $f: X \times X \rightarrow X$, then $f$ is called a \concept{binary operation} on $X$.\\
				In other words, a binary operation takes in a pair, and returns a single value.
			\end{definition}
			\begin{sidenote}{Operations Notation}
				An operation is just a function, which means it can be written in normal function \emph{prefix} notation: $f(x, y)$. However, it is more conventional to write it in \emph{infix} notation: $x f y$.
				\begin{example}
					Addition of numbers is a binary operation. If $(x, y) = (3, 4)$, then it could be written $+(3, 4)$, but it is more conventional to write $3 + 4$.
				\end{example}
			\end{sidenote}
			By convention, the elements of a binary operation are all the same set.
			\subsection[Finite and Infinite Sets]{Finite and Infinite Sets (Informal Definition)}
				\begin{definition}{Finite Set}
					A set whose cardinality is a non-negative number. Meaning one can count the number of elements in the set.\par
					\begin{example}
						$A = \{1, 2, 3, 4\}$, where $\left\lvert A\right\rvert = 4$
					\end{example}
				\end{definition}
				\begin{definition}{Infinite Set}
					A set that is not finite. Meaning one \emph{cannot} count the number of elements in the set.\par
					\begin{example}
						The set of real numbers $\mathbb{R}$ is an infinite set.
					\end{example}
				\end{definition}
			\pagebreak
			\subsection{Tables For Binary Operations}
				A way to describe a binary operation is to use a table, where the rows are based on the \emph{first} element, and the columns on the \emph{second}. The operator (the symbol used to describe the operation) is written in the top left corner.
				\begin{example}
					Let $A = \{a, b, c, d\}$.\\
					A binary operation called $+$ (NB: This is \emph{not} addition) could be written as follows:
					\begin{center}
						\begin{tabular}{|c|c c c c|}
							\hline
							\tablehead{$+$} & \tablehead{a} & \tablehead{b} & \tablehead{c} & \tablehead{d}\\
							\hline
							\tablehead{a} & a & b & c & d \\
							\tablehead{b} & b & c & d & a \\
							\tablehead{c} & c & d & a & b \\
							\tablehead{d} & d & a & b & c \\
							\hline
						\end{tabular}
					\end{center}
					This would be read (row, column). +(b, d) = a.
					\begin{center}
						\begin{tabular}{|c|c c c >{\columncolor{Lavender}}c|}
							\hline
							\tablehead{$+$} & \tablehead{a} & \tablehead{b} & \tablehead{c} & \tablehead{d}\\
							\hline
							\tablehead{a} & a & b & c & d \\
							\rowcolor{Lavender}
							\tablehead{b} & b & c & d & \cellcolor{Orchid}a \\
							\tablehead{c} & c & d & a & b \\
							\tablehead{d} & d & a & b & c \\
							\hline
						\end{tabular}
					\end{center}
					\begin{sidenote}{Extra Notes for this operation}
						Applying concepts from later to the operation:
						\begin{description}
							\item[Identity] This operation has an identity element, which is a.
							\item[Commutativity] This operation is commutative.
							\item[Associativity] This operation is associative.
						\end{description}
					\end{sidenote}
					Another binary operation, called $\bullet$ could be written as follows:
					\begin{center}
						\begin{tabular}{|c|c c c c|}
							\hline
							\tablehead{$\bullet$} & \tablehead{a} & \tablehead{b} & \tablehead{c} & \tablehead{d}\\
							\hline
							\tablehead{a} & a & b & c & d \\
							\tablehead{b} & b & a & d & c \\
							\tablehead{c} & c & d & a & b \\
							\tablehead{d} & d & c & b & a \\
							\hline
						\end{tabular}
					\end{center}
					\begin{sidenote}{Extra Notes for this operation}
						Applying concepts from later to the operation:
						\begin{description}
							\item[Identity] This operation has an identity element, which is a.
							\item[Commutativity] This operation is commutative.
							\item[Associativity] This operation is associative.
						\end{description}
					\end{sidenote}
				\end{example}
			\pagebreak
		\section{Properties of Binary Operations}
			For examples below, the following binary operation has been used:
			\begin{indentparagraph}
				$*: \{1, 2\} \times \{1, 2\} \rightarrow \{1, 2\}$ is defined by:
				\begin{align*}
					\Bigl\{\bigl((1, 1), 1\bigr), \bigl((1, 2), 2\bigr), \bigl((2, 1), 2\bigr), \bigl((2, 2), 1\bigr)\Bigr\}
				\end{align*}
				In table form this would be:
				\begin{center}
					\begin{tabular}{|c|c c|}
						\hline
						\tablehead{$*$} & \tablehead{$1$} & \tablehead{$2$}\\
						\hline
						\tablehead{$1$} & 1 & 2 \\
						\tablehead{$2$} & 2 & 1 \\
						\hline
					\end{tabular}
				\end{center}
			\end{indentparagraph}
			\subsection{Commutative Binary Operation}
				\begin{definition}{Commutativity}
					A binary operation $\Diamond: X \times X \rightarrow X$ is \concept{commutative} iff $x \Diamond y = y \Diamond x$ for all $x, y \in X$.
				\end{definition}
				The easiest way to check this is it will be commutative if it is symmetrical across the diagonal from the top left to the bottom right.
				\begin{example} \moveup
					\begin{align*}
						1 * 1 = 1 * 1 = 1\\
						1 * 2 = 2 * 1 = 2\\
						2 * 2 = 2 * 2 = 1
					\end{align*}
					Therefore $*$ is commutative.
				\end{example}
			\subsection{Associative Binary Operation}
				\begin{definition}{Associativity}
					A binary operation $\Diamond: X \times X \rightarrow X$ is \concept{associative} iff $(x \Diamond y) \Diamond z = x \Diamond (y \Diamond z)$ for all $x, y, z \in X$.
				\end{definition}
				Unfortunately, for this one, you have to check each instance.
				\begin{example} \moveup
					\begin{align*}
						(1 * 1) * 1 = 1 * 1 = 1 &\text{ and } 1 * (1 * 1) = 1 * 1 = 1\\
						(1 * 1) * 2 = 1 * 2 = 2 &\text{ and } 1 * (1 * 2) = 1 * 2 = 2\\
						(1 * 2) * 1 = 2 * 1 = 2 &\text{ and } 1 * (2 * 1) = 2 * 1 = 2\\
						(1 * 2) * 2 = 2 * 2 = 1 &\text{ and } 1 * (2 * 2) = 1 * 1 = 1\\
						(2 * 1) * 1 = 2 * 1 = 2 &\text{ and } 2 * (1 * 1) = 2 * 1 = 2\\
						(2 * 1) * 2 = 2 * 2 = 1 &\text{ and } 2 * (1 * 2) = 2 * 2 = 1\\
						(2 * 2) * 1 = 1 * 1 = 1 &\text{ and } 2 * (2 * 1) = 2 * 2 = 1\\
						(2 * 2) * 2 = 1 * 2 = 2 &\text{ and } 2 * (2 * 2) = 2 * 1 = 2
					\end{align*}
					Therefore $*$ is associative.
				\end{example}
			\pagebreak
			\subsection{Identity Element of a Binary Operation}
				\begin{definition}{Identity Element}
					An element $e$ of $X$ is an \concept{identity element} in respect of the binary operation $\Diamond: X \times X \rightarrow X$ iff $e \Diamond x = x \Diamond e = x$ for all $x \in X$.
				\end{definition}
				The easiest way to check this is if there is a row and column in the table that is identical to the header. (NB: It needs to be \emph{both} row and column, which contain the same element.)
				\begin{example} \moveup
					\begin{align*}
						\mathbf{1} * 1 = 1 &\text{ and } 1 * \mathbf{1} = 1\\
						1 * \mathbf{2} = 2 &\text{ and } 2 * \mathbf{1} = 2
					\end{align*}
				\end{example}
		\section{Operations on Vectors}
			\subsection{Vector}
				\begin{definition}{Vector}
					In this course, a \concept{vector} is considered to be an ordered \emph{n-tuple} of numbers.\\
					A \concept{vector} is represented by an n-tuple $u$ in the following way:
					\begin{align*}
						u = (u_{1}, u_{2}, u_{3}, \ldots, u_{n}) \text{ for some } n \geq 2
					\end{align*}
				\end{definition}
			\subsection{Vector Sum}
				\begin{definition}{Vector Sum}
					If $u$ and $v$ are vectors with the \emph{same number of coordinates}, then their \concept{sum}, written $u + v$ is the vector obtained by adding the corresponding coordinates of $u$ and $v$.
					\begin{align*}
						u + v &= (u_{1}, u_{2}, \ldots, u_{n}) + (v_{1}, v_{2}, \ldots, v_{n})\\
						&= (u_{1} + v_{2}, u_{2} + v_{2}, \ldots, u_{n} + v_{n})
					\end{align*}
				\end{definition}
				\begin{example}
					Let $u = (1, 2, 3)$ and $v = (4, 5, 6)$.\\
					Then
					\begin{align*}
						u + v &= (1, 2, 3) + (4, 5, 6)\\
						&= (1 + 4, 2 + 5, 3 + 6)\\
						&= (5, 7, 9)
					\end{align*}
				\end{example}
				\begin{sidenote}{Vector addition is not defined for vectors of different sizes}
					If two vectors have a different number of coordinates, you cannot add those two vectors together.
				\end{sidenote}
			\pagebreak
			\subsection{Scalar-Vector Product}
				\begin{definition}{Scalar-Vector Product}
					If $u$ is a vector and $r$ is some scalar number, then the \concept{product} of the number $r$ and the vector $u$ is the vector $r \cdot u$ obtained by multiplying each coordinate of $u$ by $r$.
					\begin{align*}
						r\cdot u &= r(u_{1}, u_{2}, \ldots, u_{n})\\
						&= (ru_{1}, ru_{2}, \ldots, ru_{n})
					\end{align*}
				\end{definition}
				\begin{example}
					Let $u = (7, 8, 9)$ and $r = 2$.\\
					Then
					\begin{align*}
						r \cdot u &= 2(7, 8, 9)\\
						&= (14, 16, 18)
					\end{align*}
				\end{example}
			\subsection{Dot Product}
				\begin{definition}{Dot Product}
					The \concept{dot product} (also called the \concept{inner product}) of vectors $u = (u_{1}, u_{2}, \ldots, u_{n})$ and \\$v = (v_{1}, v_{2}, \ldots, v_{n})$ is written $u \cdot v$ and defined by:
					\begin{align*}
						u \cdot v = u_{1}v_{1} + u_{2}v_{2} + \ldots + u_{n}v_{n}
					\end{align*}
				\end{definition}
				\begin{sidenote}{The result of the dot product is a number}
					Unlike the other operations, which result in vectors, the dot product produces a single number.
				\end{sidenote}
				\begin{example}
					Let $u = (2, 4, 6)$ and $v = (1, 3, 5)$. Then
					\begin{align*}
						u \cdot v &= (2, 4, 6)(1, 3, 5)\\
						&= (2 \cdot 1) + (4 \cdot 3) + (6 \cdot 5)\\
						&= 2 + 12 + 30\\
						&= 44
					\end{align*}
				\end{example}
				\begin{sidenote}{The dot product is not defined for vectors of different sizes}
					As with addition, if two vectors have a different number of coordinates, you cannot calculate the dot product.
				\end{sidenote}
		\pagebreak
		\section{Operations on Matrices}
			\subsection{Matrix}
				\begin{definition}{Matrix}
					A \concept{matrix} is an array of numbers organised into rows and columns, and encolsed within brackets.\\
					The number of rows is written with the letter $m$ and the number of columns with the letter $n$. So a matrix is said to have the size $m \times n$.
				\end{definition}
				\begin{example}
					$\begin{bmatrix}
						3 & 2\\
						1 & 5
					\end{bmatrix}$ is a $2 \times 2$ matrix, and
					$\begin{bmatrix}
						-1 & 3 & 0 & 5\\
						0 & 2 & 0 & 6\\
						1 & - 1& 0 & 13
					\end{bmatrix}$ is a $3 \times 4$ matrix.
				\end{example}
				Matrices (pronounced \emph{may-trisseez}) have the form 
				\begin{align*}
					\begin{bmatrix}
						a_{11} & a_{12} & \cdots & a_{1n}\\
						a_{21} & a_{22} & \cdots & a_{2n}\\
						\vdots & \vdots & & \vdots\\
						a_{m1} & a_{m2} & \cdots & a_{mn}
					\end{bmatrix}
				\end{align*}
			\subsection{Matrix Addition}
				\begin{definition}{Matrix Addition}
					Let $A$ and $B$ be two matrices of the same size. Then the matrix \concept{$A + B$} is:
					\begin{align*}
						A + B &= \begin{bmatrix}
							a_{11} & a_{12} & \cdots & a_{1n}\\
							a_{21} & a_{22} & \cdots & a_{2n}\\
							\vdots & \vdots & & \vdots\\
							a_{m1} & a_{m2} & \cdots & a_{mn}
						\end{bmatrix} + \begin{bmatrix}
							b_{11} & b_{12} & \cdots & b_{1n}\\
							b_{21} & b_{22} & \cdots & b_{2n}\\
							\vdots & \vdots & & \vdots\\
							b_{m1} & b_{m2} & \cdots & b_{mn}
						\end{bmatrix}\\
						&= \begin{bmatrix}
							a_{11} + b_{11} & a_{12} + b_{12} & \cdots & a_{1n} + b_{1n}\\
							a_{21} + b_{21} & a_{22} + b_{22} & \cdots & a_{2n} + b_{2n}\\
							\vdots & \vdots & & \vdots\\
							a_{m1} + b_{m1} & a_{m2} + b_{m2} & \cdots & a_{mn} + b_{mn}
						\end{bmatrix}
					\end{align*}
				\end{definition}
				\begin{example}
					Let $A = \begin{bmatrix}
						1 & 2\\
						3 & 4
					\end{bmatrix}$ and 
					$B = \begin{bmatrix}
						5 & 6\\
						7 & 8
					\end{bmatrix}$
					\begin{align*}
						A + B &= \begin{bmatrix}
							1 & 2\\
							3 & 4
						\end{bmatrix} + \begin{bmatrix}
							5 & 6\\
							7 & 8
						\end{bmatrix}\\
						&= \begin{bmatrix}
							1 + 5 & 2 + 6\\
							3 + 7 & 4 + 8
						\end{bmatrix}\\
						&= \begin{bmatrix}
							6 & 8\\
							10 & 12
						\end{bmatrix}
					\end{align*}
				\end{example}
			\subsection{Scalar-Matrix Multiplication}
				\begin{definition}{Scalar-Matrix Multiplication}
					Let $A$ be a matrix, and $r$ be some scalar number.\\
					Then the product \concept{$rA$} is defined as:
					\begin{align*}
						rA = r \begin{bmatrix}
							a_{11} & a_{12} & \cdots & a_{1n}\\
							a_{21} & a_{22} & \cdots & a_{2n}\\
							\vdots & \vdots & & \vdots\\
							a_{m1} & a_{m2} & \cdots & a_{mn}
						\end{bmatrix}
						= \begin{bmatrix}
							ra_{11} & ra_{12} & \cdots & ra_{1n}\\
							ra_{21} & ra_{22} & \cdots & ra_{2n}\\
							\vdots & \vdots & & \vdots\\
							ra_{m1} & ra_{m2} & \cdots & ra_{mn}
						\end{bmatrix}
					\end{align*}
				\end{definition}
				\begin{example}
					Let $r = 3$ and $A = \begin{bmatrix}
						1 & 2\\
						3 & 4
					\end{bmatrix}$
					\begin{align*}
						rA &= 3\begin{bmatrix}
							1 & 2\\
							3 & 4
						\end{bmatrix}\\
						&= \begin{bmatrix}
							3 & 6\\
							9 & 12
						\end{bmatrix}
					\end{align*}
				\end{example}
			\subsection{Matrix Multiplication}
				\begin{definition}{Matrix Multiplication}
					Let $A$ and $B$ both be matrices.
					In order for the product \concept{$AB$} to be defined,
					\begin{itemize}
						\item The number of columns of $A$ needs to be equal to the number of rows of $B$, i.e. $A_{n} = B_{m}$.
					\end{itemize}
					If the product is defined, then it will result in a matrix that is the size $A_{m} \times B_{n}$.
					\begin{center}
						\tikz[baseline=(am.base)]\node[inner xsep=0pt] (am) {$A_{m}$}; $\times$ \tikz[baseline=(an.base)]\node[inner xsep=0pt] (an) {$A_{n}$}; $\quad \cdot \quad$ \tikz[baseline=(bm.base)]\node[inner xsep=0pt] (bm) {$B_{m}$}; $\times$ \tikz[baseline=(bn.base)]\node[inner xsep=0pt] (bn) {$B_{n}$}; $= A_{m} \times B_{n}$
					\end{center}
					\begin{tikzpicture}[overlay]
						\draw[<->] (an.south) -- ++(0,-1.5ex) -| (bm.south);
						\draw[<->] (am.south) -- ++(0,-2.5ex) -| (bn.south);
					\end{tikzpicture}\\
					When multiplying matrics, it is row of first multiplied by column of second.
				\end{definition}
				\begin{example} \moveup
					\begin{align*}
						\begin{bmatrix}
							-1 & 3\\
							4 & 2\\
							5 & -7
						\end{bmatrix}\begin{bmatrix}
							6 & 9\\
							-8 & 1
						\end{bmatrix} &= \begin{bmatrix}
							(-1 \cdot 6) + (3 \cdot -8) & (-1 \cdot 9) + (3 \cdot 1)\\
							(4 \cdot 6) + (2 \cdot -8) & (4 \cdot 9) + (2 \cdot 1) \\
							(5 \cdot 6) + (-7 \cdot -8) & (5 \cdot 9) + (-7 \cdot 1)
						\end{bmatrix}\\
						&= \begin{bmatrix}
							-6 - 24 & -9 + 3\\
							24 - 16 & 36 + 2\\
							30 + 56 & 45 - 7
						\end{bmatrix}\\
						&= \begin{bmatrix}
							-30 & -6\\
							8 & 38\\
							86 & 38
						\end{bmatrix}
					\end{align*}
				\end{example}
			\pagebreak
			\subsection{Identity Matrix}
				\begin{definition}{Identity Matrix}
					If $A$ is matrix, then an \concept{identity matrix} $I$ with respect to $A$ is a matrix such that $IA = AI = A$.\\
				For the identity matrix to be defined,
				\begin{itemize}
					\item $A$ must be a square matrix, because matrix multiplication is \emph{not} commutative.
					\item $I$ must therefore also be a square matrix with the same number of rows and columns as $A$.
				\end{itemize}
				Then the identity matrix would have $1$'s for the main diagonal, and $0$'s elsewhere.
				\end{definition}
				\begin{example}
					For a $2 \times 2$ matrix, the identity matrix would be:
						\begin{align*}
							I = \begin{bmatrix}
								1 & 0\\
								0 & 1
							\end{bmatrix}
						\end{align*}
				\end{example}
	\rulechapterend
\end{document}
