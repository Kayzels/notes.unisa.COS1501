\providecommand{\main}{..}
\documentclass[\main/notes.tex]{subfiles}

\begin{document}
	\ifSubfilesClassLoaded{\setcounter{chapter}{5}}{}
	\chapter{Special Kinds of Relation}
		\section{Order Relations}
			\subsection{Weak Partial Order}
				\begin{definition}[width=0.68\textwidth]{Weak Partial Order}
					A relation $R$ on a set $A$ is called a \concept{weak partial order} iff $R$ is
					\begin{itemize}[nosep]
						\item reflexive on $A$
						\item antisymmetric, and
						\item transitive
					\end{itemize}
				\end{definition}
				\begin{example}
					Let $A = \bigl\{\{a\}, \{a, b\}\bigr\}$. A relation $S$ on $A$ is defined by $(B, C) \in S \text{ iff } B \subseteq C$. (Each first coordinate is a subset of the second coordinate.)
					\begin{align*}
						S = \Bigl\{\bigl(\{a\}, \{a\}\bigr), \bigl(\{a\}, \{a, b\}\bigr), \bigl(\{a, b\}, \{a, b\}\bigr)\Bigr\}
					\end{align*}
					To prove this is a weak partial order, prove reflexivity, antisymmetry and transitivity.
					\begin{description}
						\item[Reflexivity] Is it true that $(B, B) \in S$ for all $B \in A$? Yes.
							\begin{alignat*}{2}
								\bigl(\{a, a\}\bigr) &\in S \qquad & \bigl(\{a, b\}, \{a, b\}\bigr) &\in S
							\end{alignat*}
						\item[Antisymmetry] Is it true that for all $(B, C) \in A$, if $B \neq C$, and $(B, C) \in S$, then $(C, B) \notin S$? Yes.\\
							The elements where $B \neq C$ are $\{a\}$ and $\{a, b\}$.\\
							$\bigl(\{a\}, \{a, b\}\bigr) \in S$, and $\bigl(\{a, b\}, \{a\}\bigr) \notin S$
						\item[Transitivity] Is it true that for all $B, C, D \in A$, if $(B, C) \in S$, and $(C, D) \in S$, then $(B, D) \in S$? Yes.
							\begin{alignat*}{3}
								\bigl(\{a\}, \{a\}\bigr) &\in S \qquad &\rightarrow \bigl(\{a\}, \{a\}\bigr) &\in S \qquad &\rightarrow \bigl(\{a\}, \{a\}\bigr) &\in S\\
								\bigl(\{a\}, \{a\}\bigr) &\in S \qquad &\rightarrow \bigl(\{a,\}, \{a, b\}\bigr) &\in S \qquad &\rightarrow \bigl(\{a\}, \{a, b\}\bigr) &\in S
							\end{alignat*}
							The above can be done for all elements.
						\item[Weak Partial Order] As $S$ is reflexive, antisymmetric, and transitive, $S$ is a weak partial order.
					\end{description}
				\end{example}
				\begin{exercise}{Activity \thechapter.4}
					\begin{questions}
					 \item Determine whether the following relations are weak partial orders.
						\begin{questions}
							\item Let $A = \bigl\{a, b, \{a, b\}\bigr\}$. $S$ is the relation on $A$ defined by $(c, B) \in S$ iff $c \in B$.
								\begin{answer}
									\begin{align*}
										S = \bigl\{(a, \{a, b\}), (b, \{a, b\})\bigr\}
									\end{align*}
									\begin{description}
										\item[Reflexivity] Is it true that $(x, x) \in S$ for all $x \in A$?\\
											No. Using a counterexample: $(a, a) \notin S$.\\
											Can stop here and conclude that $S$ is not a weak partial order, but for completeness, checking the other two conditions as well.
										\item[Antisymmetry] Is it true for all $(x, y) \in A$, if $x \neq y$, and $(x, y) \in S$, then $(y, x) \notin S$?\\
											Yes. If $(x, y) \in S$, then $x \in y$. If $x \in y$, then $y$ cannot be an element of $x$, so $(y, x) \notin S$.
										\item[Transitivity] Is it true for all $x, y, z \in A$, if $(x, y) \in S$, and $(y, z) \in SS$, then $(x, z) \in S$?\\
											Yes. Vacuously true, as there is no element that has the same first coordinate as another element's second coordinate.
										\item[Weak Partial Order] As $S$ is not reflexive, $S$ is \emph{not} a weak partial order.
									\end{description}
								\end{answer}
							\item $R \subseteq \mathbb{Z} \times \mathbb{Z}$ such that $x \, R \, y$ iff $x + y$ is even.\\
								\begin{answer}
									If $x + y$ is even, then $x + y = 2k$ for some integer $k$.
									\begin{description}
										\item[Reflexivity] Is it true that $(x, x) \in R$ for all $x \in \mathbb{Z}$?\\
											Yes. $x + x = 2x$, which would be part of $R$ if $k = x$. As $2x$ is always even, $(x, x) \in R$.
										\item[Antisymmetry] Is it true that for all $(x, y) \in R$, if $x \neq y$ and $(x, y) \in R$, then $(y, x) \notin R$?\\
											No. Let $(x, y) \in R$. Then $x + y = 2k$. But $(y + x)$ also equals $2k$. So $(y, x) \in R$.\\
											Can stop here and conclude that $R$ is not a weak partial order. For completeness, checking transitivity as well.
										\item[Transitivity] Is it true that for all $x, y, z \in \mathbb{Z}$, if $(x, y) \in R$ and $(y, z) \in R$, then $(x, z) \in R$?\\
											Yes. Let $(x, y) \in R$, and $(y, z) \in R$. Then $x + y = 2k$, and $y + z = 2m$, where $k$ and $m$ are integers.
											\begin{align*}
												x + y &= 2k\\
												x &= 2k - y\\
												y + z &= 2m\\
												z &= 2m - y\\
												x + z &= (2k - y) + (2m - y)\\
												&= 2k + 2m - 2y\\
												&= 2(k + m - y)
											\end{align*}
											From the above, if $x + y$ is even, and $y + z$ is even, then $x + z$ is also even.
										\item[Weak Partial Order] As $R$ is not antisymmetric, $R$ is \emph{not} a weak partial order.
									\end{description}
								\end{answer}
								\pagebreak
							\item $R$ on $\mathbb{Z} \times \mathbb{Z}$ by $(a, b) \,R \, (c, d)$ if either $a < c$ or $(a = c$ and $b \leq d)$.
								\begin{answer}
									\begin{description}
										\item[Reflexivity] Is it true that for all $(a, b) \in \mathbb{Z} \times \mathbb{Z}$, $(a, b) \, R \, (a, b)$?\\
											Yes. It is never the case that $a < a$, but $a = a$ and $b \leq b$ is true.
										\item[Antisymmetry] Is it true that for all $(a, b)$ and $(c, d) \in \mathbb{Z} \times \mathbb{Z}$, if $(a, b) \neq (c,d)$ and $\bigl\{(a, b), (c, d)\bigr\} \in R$, then $\bigl\{(c, d), (a, b)\bigr\} \notin R$?\\
											Yes. If $(a, b) R (c, d)$, then $a < c$ or $(a = c$ and $b \leq d)$.
											If the first case is matched, then $a < c$, which means that $c > a$. Which means that $c \neq a$, so the second condition is not met.\\
											If the second case is matched, then $a = c$ and $b \leq d$. If $a = c$, then $c = a$, and if $b \leq d$, then $d \geq b$. However, if $b = d$, then $(a, b) = (c, d)$, which would mean it would be excluded. Therefore, $d < b$, which means that $\bigl\{(c, d), (a, b)\bigr\} \notin R$.
										\item[Transitivity] If $\bigl\{(a, b), (c, d)\bigr\} \in R$ and $\bigl\{(c, d), (e, f)\bigr\} \in R$, is $\bigl\{(a, b), (e, f)\bigr\} \in R$?\\
											Yes.
											\begin{indentparagraph}
												If $\bigl\{(a, b), (c, d)\bigr\} \in R$, then either $a < c$ or $(a = c$ and $b \leq d)$.\\
												If $\bigl\{(c, d), (e, f)\bigr\} \in R$, then either $c < e$ or $(c = e$ and $d \leq f)$.\\
												If $a < c$ and $c < e$, then $a < e$, which means $\bigl\{(a, b), (e, f)\bigr\} \in R$.\\
												If $a < c$ and $c = e$ and $d \leq f$, then $a < e$, which means $\bigl\{(a, b), (e, f)\bigr\} \in R$.\\
												If $a = c$ and $b \leq d$ and $c < e$, then $a < e$, which means $\bigl\{(a, b), (e, f)\bigr\} \in R$.\\
												If $a = c$ and $b \leq d$ and $c = e$ and $d \leq f$, then $c = e$, and $b \leq f$, which means $\bigl\{(a, b), (e, f)\bigr\} \in R$.
											\end{indentparagraph}
											\item[Weak Partial Order] As $R$ is reflexive, antisymmetric and transitive, $R$ \emph{is} a weak partial order.
									\end{description}
								\end{answer}
						\end{questions}
					\end{questions}
				\end{exercise}
			\pagebreak
			\subsection{Strict Partial Order}
				\begin{definition}[width=0.68\textwidth]{Strict Partial Order}
					A relation $R$ on a set $A$ is called a \concept{strict partial order} iff $R$ is
					\begin{itemize}[nosep]
						\item irreflexive on $A$
						\item antisymmetric, and
						\item transitive
					\end{itemize}
				\end{definition}
				\begin{example}
					Let $A = \{1, 2, 3\}$ and let $S$ on $A$ be the relation $S = \bigl\{(1, 2), (1, 3), (2, 3)\bigr\}$. (Every first coordinate is less than the second coordinate.)\\
					To prove this is a strict partial order, prove irreflexivity, antisymmetry and transitivity.
					\begin{description}
						\item[Irreflexivity] Is it true that $(x, x) \notin S$ for any $x \in A$?\\
							Yes, no element is related to itself, i.e. the pairs $(1, 1)$, $(2, 2)$ and $(3, 3)$ are not elements of $S$.
						\item[Antisymmetry] Is it true that for all $x, y \in A$, if $(x, y) \in S$, then $(y, x) \notin S$?\\
							Yes. $1 \neq 2$ and $(1, 2) \in S$ and $(2, 1) \notin S$.\\
							$1 \neq 3$ and $(1, 3) \in S$ and $(3, 1) \notin S$.\\
							$2 \neq 3$ and $(2, 3) \in S$ and $(3, 2) \notin S$.
						\item[Transitivity] Is it true that for all $x, y, z \in A$, if $(x, y) \in S$ and $(y, z) \in S$, then $(x, z) \in S$?\\
							Yes. $(1, 2) \in S$ and $(2, 3) \in S$ and $(1, 3) \in S$.
						\item[Strict Partial Order] As $S$ is irreflexive, antisymmetric, and transitive, $S$ is a strict partial order.
					\end{description}
				\end{example}
				\begin{exercise}{Activity \thechapter.5}
					\begin{questions}
						\item Determine whether the following relations are strict partial orders.
							\begin{questions}
								\item $A = \bigl\{a, \{a\}, \{b\}\bigr\}$ and the relation $S$ on $A$ is $S = \Bigl\{\bigl(a, \{a\}\bigr), \bigl(a, \{b\}\bigr)\Bigr\}$
									\begin{answer}
										\begin{description}
											\item[Irreflexivity] Is it true that $(x, x) \notin S$ for any $x \in A$?\\
												Yes, no element is related to itself, i.e. the pairs $\bigl(a, a\bigr)$, $\bigl(\{a\}, \{a\}\bigr)$ and $\bigl(\{b\}, \{b\}\bigr)$ are not elements of $S$.
											\item[Antisymmetry] Is it true that for all $x, y \in A$ and $x \neq y $, if $(x, y) \in S$, then $(y, x) \notin S$?\\
												Yes. $a \neq \{a\}$. $\bigl(a, \{a\}\bigr) \in S$, and $\bigl(\{a\}, a\bigr) \notin S$.\\   
												$a \neq \{b\}$. $\bigl(a, \{b\}\bigr) \in S$, and $\bigl(\{b\}, a\bigr) \notin S$.
											\item[Transitivity] Is it true that for all $x, y, z \in A$, if $(x, y) \in S$ and $(y, z) \in S$, then $(x, z) \in S$?\\
												Yes. There are no ordered pairs such that $(x, y) \in R$ and $(y, z) \in R$, so $R$ is \emph{vacuously} transitive.
											\item[Strict Partial Order] As $R$ is irreflexive, antisymmetric, and transitive, $R$ \emph{is} a strict partial order.
										\end{description}
									\end{answer}
								\pagebreak
								\item $R \subseteq (\mathbb{Z} \times \mathbb{Z}) \times (\mathbb{Z} \times \mathbb{Z})$ such that $(a, b) \, R \, (c, d)$ iff $a < c$.
									\begin{answer}
										\begin{description}
											\item[Irreflexivity] Is it true that $\bigl((a, b), (a, b)\bigr)\notin R$ for any $(a, b) \in \mathbb{Z} \times \mathbb{Z}$?\\
												Yes. For $\bigl((a, b), (a, b)\bigr) \in R$, it would need to satisfy the requirement $a < a$, which is never true.
											\item[Antisymmetry] Is it true that for all $(a, b), (c, d) \in \mathbb{Z} \times \mathbb{Z}$, where $(a, b) \neq (c, d)$, \\ if $\left((a, b), (c, d)\right) \in R$, then $\left((c, d), (a, b)\right) \notin R$.\\
												Yes. If $\left((a, b), (c, d)\right) \in R$, then $a < c$. As $a < c$, $\left((c, d), (a, b)\right) \notin R$.
											\item[Transitivity] Is it true for all $(a, b), (c, d), (e, f) \in \mathbb{Z} \times \mathbb{Z}$, \\ if $\left((a, b), (c, d)\right) \in R$ and $\left((c, d), (e, f)\right) \in R$, then $\left((a, b), (e, f)\right) \in R$?\\
												Yes. If $\left((a, b), (c, d)\right) \in R$, then $a < c$. If $\left((c, d), (e, f)\right) \in R$, then $c < e$. Therefore, $a < e$, so $\left((a, b), (e, f)\right) \in R$.
											\item[Strict Partial Order] As $R$ is irreflexive, antisymmetric, and transitive, $R$ \emph{is} a strict partial order.
										\end{description}
									\end{answer}
							\end{questions}
					\end{questions}
				\end{exercise}
			\pagebreak
			\subsection{A Total (or Linear) Order Relation}
				\begin{definition}{Total Order Relation}
					A relation $R$ on a set $A$ is called a \concept{total} or \concept{linear order} if $R$ is a partial order on $A$ that also satisfies \emph{trichotomy}.
				\end{definition}
				\begin{example}
					The example for Strict Partial Orders:
					\begin{indentparagraph}
						Let $A = \{1, 2, 3\}$ and let $S$ on $A$ be the relation $S = \bigl\{(1, 2), (1, 3), (2, 3)\bigr\}$. (Every first coordinate is less than the second coordinate.)
					\end{indentparagraph}
					also satisfies trichotomy.
					\begin{description}
						\item[Trichotomy] Is every element of $A$ related to every other element in the relation $S$?\\
							Yes. $1$ is related to $2$ in $(1, 2)$, and related to $3$ in $(1, 3)$.\\
							$2$ is related to $1$ in $(1, 2)$, and related to $3$ in $(2, 3)$.\\
							$3$ is related to $1$ in $(1, 3)$, and related to $2$ in $(2, 3)$.
						\item[Total Order Relation] As this relation is a partial order relation that satisfies trichotomy, it is a \concept{total order relation}. As the relation is a \emph{strict} partial order, this is a \concept{strict total order relation}.
					\end{description}
				\end{example}
				\begin{sidenote}{Proof Strategies}
					You cannot use examples to prove a general statement, i.e, something of the form:
						\begin{indentparagraph}
							For all $x$, or\\
							For all pairs $(x, y)$
						\end{indentparagraph}
					Instead, \emph{abstract reasoning} needs to be used to produce a \emph{general proof}.\\
					However, an example can be used to show that a statement is false, which is known as a \concept{counterexample}.
				\end{sidenote}
				\begin{exercise}{Self Assesmment \thechapter.7}
					\begin{questions}
						\item Let $X = \{a, b, c\}$. Write down all strict partial orders on $X$. Which of them are linear?
							\begin{answer}
								Strict partial orders are irreflexive, antisymmetric and transitive.
								\begin{description}
									\item[One element] There are $6$ relations on $X$ that are strict partial orders that contain only one element:
										\begin{center}
											$ \bigl\{(a, b)\bigr\}$, $\bigl\{(a, c)\bigr\}$, $\bigl\{(b, a)\bigr\}$, $\bigl\{(b, c)\bigr\}$, $\bigl\{(c, a)\bigr\}$, $\bigl\{(c, b)\bigr\}$
										\end{center}
									\item[Two elements] There are $6$ relations on $X$ that are strict partial orders that contain two elements:
										\begin{center}
											$ \bigl\{(a, b), (a, c)\bigr\}$, $ \bigl\{(a, b), (c, b)\bigr\}$, $ \bigl\{(a, c), (b, c)\bigr\}$, $ \bigl\{(b, a), (b, c)\bigr\}$, $ \bigl\{(b, a), (c, a)\bigr\}$, $\bigl\{(c, a), (c, b)\bigr\}$
										\end{center}
									\tcbbreak
									\item[Three elements] There are $6$ relations on $X$ that are strict partial orders that contain three elements:
										\begin{center}
											$\bigl\{(a, b), (b, c), (a, c)\bigr\}$, $\bigl\{(b, a), (a, c), (b, c)\bigr\}$, $\bigl\{(c, b), (b, a), (c, a)\bigr\}$, $\bigl\{(a, c), (c, b), (a, b)\bigr\}$, $\bigl\{(c, a), (a, b), (c, b)\bigr\}$, $\bigl\{(b, c), (c, a), (b, a)\bigr\}$
										\end{center}
									\item[More than three elements] There are no relations on $X$ that are strict partial orders And contain more than three elements.
									\item[Linear] For a relation to be linear, it needs to satisfy \emph{trichotomy}. As there are three elements in $X$ the relation should contain three or more elements.\\
										All the strict partial relations with three elements satisfy trichotomy, and so are linear. 
								\end{description}
							\end{answer}
						\item In each of the following cases, determine whether $R$ is some sort of order relation on the given set $X$. Justify your answer.
							\begin{questions}
								\item $X = \bigl\{\emptyset, \{0\}, \{2\}\bigr\}$ and $R = \Bigl\{\bigl(\emptyset, \{0\}\bigr), \bigl(\emptyset, \{2\}\bigr)\Bigr\}$\\
									\begin{answer}
										$R$ is a \emph{strict partial order}.	
										\begin{proof}
											$ $
											\begin{description}
												\item[Reflexivity] $R$ is not reflexive.
													\begin{subproof}[Counterexample]
														$(\emptyset, \emptyset) \notin R$
													\end{subproof}
												\item[Irreflexivity] $R$ is irreflexive.
													\begin{subproof}
														For all $x \in X$, $(x, x) \notin R$.
													\end{subproof}
												\item[Antisymmetry] $R$ is antisymmetric.
													\begin{subproof}
														$\bigl(\emptyset, \{0\}\bigr) \in R$ and $\bigl(\{0\}, \emptyset\bigr) \notin R$\\
														$\bigl(\emptyset, \{2\}\bigr) \in R$ and $\bigl(\{2\}, \emptyset\bigr) \notin R$\\
														For all elements $(x, y) \in R$, $(y, x) \notin R$.
													\end{subproof}
												\item[Transitivity] $R$ is transitive.
													\begin{subproof}
														There are no elements such that the second coordinate of a pair is the first coordinate of another pair.
													\end{subproof}
												\item[Trichotomy] $R$ does not satisfy trichotomy.
													\begin{subproof}[Counterexample]
														There are no pairs in the relation where $\{0\}$ and $\{2\}$ are related to each other.
													\end{subproof}
											\end{description}
											As $R$ is irreflexive, antisymmetric and transitive, but does not satisfy trichotomy, $R$ is a strict partial order.
										\end{proof}
									\end{answer}
								\tcbbreak
								\item $X = \Bigl\{\emptyset, \{\emptyset\}, \bigl\{\{\emptyset\}\bigr\}\Bigr\}$ and $R = \subseteq$. (That is, each first coordinate is a subset of the second coordinate)\\
									\begin{answer}
										$R = \biggl\{(\emptyset, \emptyset), \bigl(\{\emptyset\}, \{\emptyset\}\bigr), \Bigl(\bigl\{\{\emptyset \}\bigr\}, \bigl\{\{\emptyset\}\bigr\}\Bigr), \bigl(\emptyset, \{\emptyset\}\bigr), \Bigl(\emptyset, \bigl\{\{\emptyset\}\bigr\}\Bigr)\biggr\}$\\
										$R$ is a \emph{weak partial order}.
										\begin{proof}
											$ $
											\begin{description}
												\item[Reflexivity] $R$ is reflexive.
													\begin{subproof}
														For all $x$ in $X$, $(x, x) \in R$.
													\end{subproof}
												\item[Irreflexivity] $R$ is not irreflexive.
													\begin{subproof}[Counterexample]
														$(\emptyset, \emptyset) \in R$.
													\end{subproof}
												\item[Antisymmetry] $R$ is antisymmetric.
													\begin{subproof}
														For all elements $(x, y) \in R$, $(y, x) \notin R$.
													\end{subproof}
												\item[Transitivity] $R$ is transitive.
													\begin{subproof}
														Whenever $(x, y) \in R$ and $(y, z) \in R$, $(x, z) \in R$.
													\end{subproof}
												\item[Trichotomy] $R$ does not satisfy trichotomy.
													\begin{subproof}[Counterexample]
														There are no pairs in $R$ where $\{\emptyset\}$ is related to $\bigl\{\{\emptyset\}\bigr\}$.
													\end{subproof}
											\end{description}
											As $R$ is reflexive, antisymmetric and transitive, but does not satisfy trichotomy, $R$ is a weak partial order.
										\end{proof}
									\end{answer}
							\end{questions}
						\item $X = \mathbb{Z}$ and $R = \leq$\\
							\begin{answer}
								$R$ is a \emph{weak total order}.
								\begin{proof}
									$ $
									\begin{description}
										\item[Reflexivity] $R$ is reflexive.
											\begin{subproof}
												For all $x \in \mathbb{Z}$, $x = x$, so $x \leq x$, so $(x, x) \in R$.
											\end{subproof}
										\item[Irreflexivity] $R$ is not irreflexive.
											\begin{subproof}[Counterexample]
												$(1, 1) \in R$
											\end{subproof}
										\item[Antisymmetry] $R$ is antisymmetric.
											\begin{subproof}
												If $(x, y) \in R$ and $x \neq y$, then $x < y$.\\
												Therefore, $y \not < x$, so $(y, x) \notin R$.
											\end{subproof}
										\pagebreak
										\item[Transitivity] $R$ is transitive.
											\begin{subproof}
												If $(x, y) \in R$, then $x \leq y$, and
												if $(y, z) \in R$, then $y \leq z$.\\
												If $x < y$ and $y < z$, then $x < z$, so $(x, z) \in R$.\\
												If $x < y$ and $y = z$, then $x < z$, so $(x, z) \in R$.\\
												If $x = y$ and $y < z$, then $x < z$, so $(x, z) \in R$.\\
												If $x = y$ and $y = z$, then $x = z$, so $(x, z) \in R$.\\
												Therefore, if $(x, y) \in R$ and $(y, z) \in R$, then $(x, z) \in R$. 
											\end{subproof}
										\item[Trichotomy] $R$ satisfies trichotomy.
											\begin{subproof}
												For all $x, y \in \mathbb{Z}$, either $x = y$, or $x > y$ or $x < y$. If $x > y$, then $y < x$. So $x$ and $y$ are always related to each other in $R$.
											\end{subproof}
									\end{description}
									As $R$ is reflexive, antisymmetric and transitive, and satisfies trichotomy, $R$ is a weak total order.
								\end{proof}
							\end{answer}
						\item $X = \mathbb{Z}$ and $R = >$\\
							\begin{answer}
								$R$ is a \emph{strict total order}.
								\begin{proof}
									$ $
									\begin{description}
										\item[Reflexivity] $R$ is not reflexive.
											\begin{subproof}[Counterexample]
												$(1, 1) \notin R$.
											\end{subproof}
										\item[Irreflexivity] $R$ is irreflexive.
											\begin{subproof}
												For all $x \in \mathbb{Z}$, $(x, x) \notin R$. That is, $x \not > x$.
											\end{subproof}
										\item[Antisymmetry] $R$ is antisymmetric.
											\begin{subproof}
												If $(x, y) \in R$, then $x > y$, so $y \not > x$, so $(y, x) \notin R$.
											\end{subproof}
										\item[Transitivity] $R$ is transitive.
											\begin{subproof}
												If $(x, y) \in R$, then $x > y$. If $(y, z) \in R$, then $y > z$.\\
												Therefore, $x > y > z$, i.e. $x > z$, so $(x, z) \in R$.
											\end{subproof}
										\item[Trichotomy] $R$ satisfies trichotomy.
											\begin{subproof}
												For all elements $x, y \in \mathbb{Z}$, if $x \neq y$, then $x > y$ or $y > x$, so either $(x, y) \in R$, or $(y, x) \in R$
											\end{subproof}
									\end{description}
									As $R$ is irreflexive, antisymmetric and transitive, and satisfies trichotomy, $R$ is a strict total order.
								\end{proof}
							\end{answer}
						\pagebreak
						\item $X \in \mathbb{Z}^{+}$ and $x \, R \, y$ iff $x$ divides into $y$ with zero remainder. $y = kx$ for some $k \in \mathbb{Z}^{+}$. $x$ is a \emph{factor} of $y$ and $y$ is a \emph{multiple} of $x$.\\
							\begin{answer}
								Some example elements of $R$ are $(2, 8)$, $(7, 21)$, $(6, 36)$, $(1, 1)$.\\
								$R$ is a \emph{weak partial order}.
								\begin{proof}
									$ $
									\begin{description}
										\item[Reflexivity] $R$ is reflexive.
											\begin{subproof}
												For all $x \in \mathbb{Z}^{+}$, $(x, x) \in R$, as
												\begin{align*}
													y &= kx \tag*{$(k \in \mathbb{Z}^{+})$}\\
													&= (1)x\\
													&= x
												\end{align*}
											\end{subproof}
										\item[Irreflexivity] $R$ is not irreflexive.
											\begin{subproof}[Counterexample]
												$(1, 1) \in R$
											\end{subproof}
										\item[Antisymmetry] $R$ is antisymmetric.
											\begin{subproof}
												For all $x, y \in \mathbb{Z}^{+}$, where $x \neq y$, let $y = kx$.\\
												Let $x = my$. Then $y = k(my) = (km)y$. So $km = 1$. That means $x = y$, but that was assumed to be false.\\
												So, if $(x, y) \in R$, then $(y, x) \notin R$.
											\end{subproof}
										\item[Transitive] $R$ is transitive.
											\begin{subproof}
												Let $(x, y) \in R$ and $(y, z) \in R$. That means that $y = kx$, where $k \in \mathbb{Z}^{+}$, and $z = my$, where $m \in \mathbb{Z}^{+}$.\\
												As $z = my$, that means $z = m(kx)$, i.e. $z = (km)x$. $km$ is also an element of $\mathbb{Z}^{+}$, so $(x, z) \in R$.
											\end{subproof}
										\item[Trichotomy] $R$ does not satisfy trichotomy.
											\begin{subproof}[Counterexample]
												There are no elements of $R$ where $2$ is related to $3$.
											\end{subproof}
									\end{description}
									As $R$ is reflexive, antisymmetric and transitive, and does not satisfy trichotomy, $R$ is a weak partial order.
								\end{proof}
							\end{answer}
					\end{questions}
				\end{exercise}
		\pagebreak
		\section{Equivalence Relation}
			\begin{definition}[width=0.7\textwidth]{Equivalence Relation}
				A relation $R$ on a set $A$ is called an \concept{equivalence relation} if $R$ is:
				\begin{itemize}[nosep]
					\item reflexive on $A$
					\item symmetric, and
					\item transitive
				\end{itemize}
			\end{definition}
			\begin{example}
				Let $A$ be the set of real numbers. A relation $R$ on $A$ is defined as $(x, y) \in R$ iff $x = y$.
				\begin{description}
					\item[Reflexivity] Is it true that $(x, x) \in R$ for all $x \in A$?\\
						Yes. If $x = x$, then $(x, x) \in R$, and $x = x$ is always true.
					\item[Symmetry] Is it true that if $(x, y) \in R$, then $(y, x) \in R$?\\
						Yes. If $(x, y) \in R$, then $x = y$. But if $x = y$, then $y = x$, so $(y, x) \in R$.
					\item[Transitivity] Is it true that is $(x, y) \in R$, and $(y, z) \in R$, then $(x, z) \in R$?\\
						Yes. If $(x, y) \in R$, then $x = y$. And if $(y, z) \in R$, then $y = z$. So $x = y = z$, i.e. $x = z$, i.e. $(x, z) \in R$.
					\item[Equivalence Relation] As $R$ is reflexive, symmetric and transitive, $R$ is an equivalence relation.
				\end{description} 
			\end{example}
			Equivalence relations are used to group related data together based on a specific characteristic.
			\begin{example}
				Students get marked for an assignment using grades from A to E. All students who get an A would be in the same equivalence class, even if their individual marks are different.
			\end{example}
			\subsection{Equivalence Class}
				\begin{definition}[width=0.7\textwidth]{Equivalence Class}
					For each $x \in A$, the \concept{equivalence class} $[x] = \{y \mid y \in A$ and $ x \, R \, y\}$
				\end{definition}
				\nopagebreak
				\begin{example}
					Let $R$ be the relation on $\mathbb{Z}$ defined by $(x, y) \in R$ iff $y - x$ is even.\\
					That is, $R = \{y \mid y - x = 2k\}$ for some $k \in \mathbb{Z}$. So,
					\begin{align*}
						[x] &= \{y \mid y - x = 2k\}\\
						&= \{y \mid y = 2k + x\}
					\end{align*}
					Then substitute elements of $x$ until there are no more equivalence classes.
					\begin{align*}
						[0] &= \{y = 2k\}\\
						&= \{\ldots, -6, -4, -2, 0, 2, 4, 6, \ldots\} = [2] = [4] \ldots\\
						[1] &= \{y = 2k + 1\}\\
						&= \{\ldots, -5, -3, -1, 1, 3, 5, 7, \ldots\} = [3] = [5] \ldots
					\end{align*}
					$[0]$ is the set of even integers, and $[1]$ is the set of odd integers.\\
					These two equivalence classes would be the parts of the \concept{partition} $S$ of the set $\mathbb{Z}$ on the relation $R$: $S = \bigl\{[0], [1]\bigr\}$
				\end{example}
				\begin{exercise}{Self-Assessment Exercise \thechapter.10}
					\begin{questions}
						\item Let $X = \{a, b, c\}$. Write down all equivalence relations on $X$.\\
							\begin{answer}
								For an equivalence relation, the relation needs to be \emph{reflexive}, \emph{symmetric} and \emph{transitive}.
								\begin{description}
									\item[Reflexivity] For reflexivity, \{(a, a), (b, b), (c, c)\} need to be part of the relation.
									\item[Symmetry] If $(a, b)$ is added, then $(b, a)$ must be added. This still satisfies transitivity.\\
									If $(a, c)$ is added, then $(c, a)$ must be added.\\
									If $(b, c)$ is added, then $(c, b)$ must be added.
									\item[Transitivity] If $(a, b)$ is added, and $(b, c)$ is added, then $(a, c)$ must be added.
									\item[All equivalence relations]
										\begin{align*}
											R_{1} &= \bigl\{(a, a), (b, b), (c, c)\bigr\}\\
											R_{2} &= \bigl\{(a, a), (b, b), (c, c), (a, b), (b, a)\bigr\}\\
											R_{3} &= \bigl\{(a, a), (b, b), (c, c), (a, c), (c, a)\bigr\}\\
											R_{4} &= \bigl\{(a, a), (b, b), (c, c), (b, c), (c, b)\bigr\}\\
											R_{5} &= \bigl\{(a, a), (b, b), (c, c), (a, b), (b, a), (a, c), (c, a), (b, c), (c, b)\bigr\}
										\end{align*} 
								\end{description}
							\end{answer}
						\item Determine whether the following relations $R$ on $X$ are equivalence relations. If they are, describe the equivalence classes of $R$.
							\begin{questions}
								\item $X = \{a, b, c\}$ and $R = \bigl\{(c, c), (b, b), (a, a)\bigr\}$
									\begin{answer}
										\begin{description}
											\item[Reflexivity] Yes. For all $x$ in $X$, $(x, x) \in R$.
											\item[Symmetry] Yes. Each element is symmetric with itself.
											\item[Transitivity] Yes. Vacuously transitive.
											\item[Equivalence relation] $R$ is an equivalence relation.
											\item[Equivalence classes] $
												\begin{aligned}[t]
													[x] &= \{y \mid (x, y) \in R\}\\
													[c] &= \{y \mid (c, y) \in R\}\\
													&= \{c\}\\
													[b] &= \{y \mid (b, y) \in R\}\\
													&= \{b\}\\
													[a] &= \{y \mid (a, y) \in R\}\\
													&= \{a\}
												\end{aligned} $ 
										\end{description}
									\end{answer}
								\pagebreak
								\item $ X = \{a, b, c\}$ and $R = X \times X$
									\begin{answer}
										\begin{align*}
											R &= \bigl\{(a, a), (a, b), (a, c), (b, a), (b, b), (b, c), (c, a), (c, b), (c, c)\bigr\}
										\end{align*}
										\begin{description}
											\item[Reflexivity] Yes. $(a, a)$, $(b, b)$ and $(c, c)$ are all in $R$.
											\item[Symmetry] Yes. Every element in $R$ has its mirror image.
											\item[Transitivity] Yes.
											\item[Equivalence relation] $R$ is an equivalence relation.
											\item[Equivalence classes] $
												\begin{aligned}[t]
													[x] &= \{y \mid (x, y) \in R\}\\
													[a] &= \{a \mid (a, y) \in R\}\\
													&= \{a, b, c\}\\
													&= [b]\\
													&= [c]
												\end{aligned} $  
										\end{description}
									\end{answer}
								\item $X = \mathcal{P}(Y)$ where $Y = \{1, 2, 3\}$ and $R$ consists of all pairs $(C, D)$ such that $C \cap \{2\} = D \cap \{2\}$ \\
								\begin{answer}
									$
										\begin{aligned}[t]
											X &= \mathcal{P}(Y)\\
											&= \bigl\{\emptyset, \{1\}, \{2\}, \{3\}, \{1, 2\}, \{1, 3\}, \{2, 3\}, \{1, 2, 3\}\bigr\}
										\end{aligned} $
										\begin{center}
											$S \cap \{2\} = \emptyset$ if $2 \notin S$. \qquad $S \cap \{2\} = \{2\}$ if $2 \in S$.
										\end{center}
										\begin{description}
											\item[Sets without 2:] $\emptyset$, $\{1\}$, $\{3\}$, $\{1, 3\}$
												\begin{align*}
													R_{1} &= \Bigl\{(\emptyset, \emptyset), \bigl(\emptyset, \{1\}\bigr), \bigl(\emptyset, \{3\}\bigr), \bigl(\emptyset, \{1, 3\}\bigr), \\
													& \quad \bigl(\{1\}, \emptyset\bigr), \bigl(\{1\}, \{1\}\bigr), \bigl(\{1\}, \{3\}\bigr), \bigl(\{1\}, \{1, 3\}\bigr),\\
													& \quad \bigl(\{3\}, \emptyset\bigr), \bigl(\{3\}, \{1\}\bigr), \bigl(\{3\}, \{3\}\bigr), \bigl(\{3\}, \{1, 3\}\bigr)\\
													& \quad \bigl(\{1, 3\}, \emptyset\bigr), \bigl(\{1, 3\}, \{1\}\bigr), \bigl(\{1, 3\}, \{3\}\bigr), \bigl(\{1, 3\}, \{1, 3\}\bigr)\Bigr\}
												\end{align*}
											\item[Sets with 2:] $\{2\}$, $\{1, 2\}$, $\{2, 3\}$, $\{1, 2, 3\}$
												\begin{align*}
													R_{2} &= \Bigl\{\bigl(\{2\}, \{2\}\bigr), \bigl(\{2\}, \{1, 2\}\bigr), \bigl(\{2\}, \{2, 3\}\bigr), \bigl(\{2\}, \{1, 2, 3\}\bigr)\\
													& \quad \bigl(\{1, 2\}, \{2\}\bigr), \bigl(\{1, 2\}, \{1, 2\}\bigr), \bigl(\{1, 2\}, \{2, 3\}\bigr), \bigl(\{1, 2\}, \{1, 2, 3\}\bigr)\\
													& \quad \bigl(\{2, 3\}, \{2\}\bigr), \bigl(\{2, 3\}, \{1, 2\}\bigr), \bigl(\{2, 3\}, \{2, 3\}\bigr), \bigl(\{2, 3\}, \{1, 2, 3\}\bigr)\\
													& \quad \bigl(\{1, 2, 3\}, \{2\}\bigr), \bigl(\{1, 2, 3\}, \{1, 2\}\bigr), \bigl(\{1, 2, 3\}, \{2, 3\}\bigr), \bigl(\{1, 2, 3\}, \{1, 2, 3\}\bigr)\Bigr\}
												\end{align*}
										\end{description}
										\begin{description}
											\item[Reflexivity] Yes. All elements are related to themselves.
											\item[Symmetry] Yes. All elements have their mirror image.
											\item[Transitivity] Yes.
											\item[Equivalence relation] $R$ is an equivalence relation.
											\item[Equivalence classes] $
												\begin{aligned}[t]
													[X] &= \{Y \mid (X, Y) \in R\}\\
													[\emptyset] &= \bigl\{\emptyset, \{1\}, \{3\}, \{1, 3\}\bigr\}\\
													\bigl[\{2\}\bigr] &= \bigl\{\{2\}, \{1, 2\}, \{2, 3\}, \{1, 2, 3\}\bigr\}
												\end{aligned} $   
										\end{description}
								\end{answer}
							\end{questions}
						\item Let $R$ be the relation on $\mathbb{Z}$ such that $(x, y) \in R$ iff $x - y$ is a multiple of $4$.\\
							\begin{answer}
								$R = (x, y) \in \mathbb{Z}$ such that $ x - y = 4k$, where $k \in \mathbb{Z}$.
							\end{answer}
							\begin{questions}
								\item Do tests on $R$ for all the following properties: reflexivity, irreflexivity, symmetry, antisymmetry, transitivity, and trichotomy.
									\begin{answer}
										\begin{description}
											\item[Reflexivity] For every $x \in \mathbb{Z}$, is $(x, x) \in R$? Yes.
												\begin{proof}
													For all $x$, you would have $x - x = 4k$, that is $0 = 4k$, so $k = 0$. $(x, x) \in R$.
												\end{proof}
											\item[Irreflexivity] For every $x \in \mathbb{Z}$, is $(x, x) \notin R$? No.
												\begin{proof}[Counterexample]
													$(1, 1) \in R$.
												\end{proof}
											\item[Symmetry] For $x, y \in \mathbb{Z}$, if $(x, y) \in R$, is $(y, x) \in R$? Yes.
												\begin{proof}
													Suppose $(x, y) \in R$. Then $x - y = 4k$, where $k \in \mathbb{Z}$. But $y - x = -4k$. That is, $y - x = 4(-k)$. But $k$ can be any integer. Therefore, $(y, x) \in R$.
												\end{proof}
											\item[Antisymmetry] For $x, y \in \mathbb{Z}$, if $x \neq y$ and $(x, y) \in R$, is $(y, x) \notin R$? No.
												\begin{proof}[Counterexample]
													$(8, 4) \in R$ and $(4, 8) \in R$
												\end{proof}
											\item[Transitivity] For $x, y, z \in \mathbb{Z}$, if $(x, y) \in R$ and $(y, z) \in R$, is $(x, z) \in R$? Yes.
												\begin{proof}
													Suppose $(x, y) \in R$ and $(y, z) \in R$. Then $x - y = 4k$ for some $k \in \mathbb{Z}$, and $y - z = 4m$ for some $m \in \mathbb{Z}$.
													\begin{alignat*}{2}
														&\qquad & y - z &= 4m\\
														& \Rightarrow \quad&y &= 4m + z\\
														& \quad &x - y &= 4k\\
														& \Rightarrow \quad &x - (4m + z) &= 4k\\
														& \Rightarrow \quad &x - 4m - z &= 4k\\
														& \Rightarrow \quad &x - z &= 4k + 4m\\
														& \Rightarrow \quad &x - z &= 4(k + m)
													\end{alignat*}
													$\therefore (x, z) \in R$ 
												\end{proof}
											\item[Trichotomy] Is every element in $\mathbb{Z}$ related to every other element in $\mathbb{Z}$? No.
												\begin{proof}[Counterexample]
													There is no element of $R$ where $1$ is related to $2$. $(1, 2) \notin R$ and $(2, 1) \notin R$.
												\end{proof}
										\end{description}
									\end{answer}
								\item What kind of relation is $R$?\\
								{\answer $R$ is an equivalence relation.}
								\pagebreak
								\item If $R$ is an equivalence relation, give the equivalence classes of $R$ and show some members of each class.
									\begin{answer}
										\begin{align*}
											[x] &= \{y \mid (x, y) \in R\}\\
											&= \{y \mid x - y = 4k\}\\
											&= \{y \mid y = x - 4k\}\\
											[0] &= \{y = - 4k\}\\
											&= \{\ldots, 8, 4, 0, -4, -8, \ldots\}\\
											&= \{\ldots, -8, -4, 0, 4, 8, \ldots\}\\
											[1] &= \{1 - 4k\}\\
											&= \{\ldots, 9, 5, 1, -3, -7, \ldots\}\\
											&= \{\ldots, -7, -3, 1, 5, 9, \ldots\}\\
											[2] &= \{2 - 4k\}\\
											&= \{\ldots, 10, 6, 2, -2, -6, \ldots\}\\
											&= \{\ldots, -6, -2, 2, 6, 10, \ldots\}\\
											[3] &= \{3 - 4k\}\\
											&= \{\ldots, 11, 7, 3, -1, -5, \ldots\}\\
											&= \{\ldots, -5, -1, 3, 7, 11, \ldots\}
										\end{align*}
										The equivalence classes for $[4]$ and up have already been covered.
									\end{answer}
							\end{questions}
						\item Suppose $\mathbb{Q}^{+}$ is the set of all positive quotients $\frac{m}{n}$, where $m, n \in \mathbb{Z}^{+}$. That is, $\mathbb{Q}^{+}$ is the set of positive rational numbers. Let $R$ be the relation on $\mathbb{Q}^{+}$ defined by the rule $(x, y) \in R$ iff $y = \frac{a}{b}\bigl(x\bigr)$ for some $a, b \in \mathbb{Z}^{+}$. Prove that $R$ is an equivalence relation, and show the equivalence classes of $R$.\\
							\begin{answer}
								Some examples of elements of $R$: $\left(\dfrac{1}{2}, \dfrac{3}{5}\right)$, $\left(\dfrac{3}{4}, \dfrac{5}{6}\right)$
									\begin{align*}
										\frac{3}{5} &= \frac{a}{b}\left(\frac{1}{2}\right) = \left(\frac{6}{5}\right)\left(\frac{1}{2}\right) = \frac{6}{10}\tag*{$a = 6$ and $b = 5$}\\
										\frac{5}{6} &= \frac{a}{b}\left(\frac{3}{4}\right) = \left(\frac{10}{9}\right)\left(\frac{3}{4}\right) = \frac{30}{36}\tag*{$a = 10$ and $b = 9$}
									\end{align*}
								\begin{proof}
									$ $
									\begin{description}
										\item[Reflexivity] For every $x \in \mathbb{Q}^{+}$, is $(x, x) \in R$? Yes.
											\begin{subproof}[Subproof]
												For $(x, x)$ to be in $R$, it needs to satisfy: 
													\begin{align*}
														x &= \frac{a}{b}\bigl(x\bigr) \text{ for some } a, b \in \mathbb{Z}^{+}\\
														&= \frac{1}{1}\bigl(x\bigr) \tag*{$a = 1$ and $b = 1$}\\
														&= x
													\end{align*}
												$\therefore (x, x) \in R$, so $R$ is reflexive.
											\end{subproof}
										\pagebreak
										\item[Symmetry] For every $x, y \in \mathbb{Q}^{+}$, if $(x, y) \in R$, is $(y, x) \in R$? Yes.
											\begin{subproof}[Subproof]
												Suppose $(x, y) \in R$. Then $y = \dfrac{a}{b}\bigl(x\bigr)$.
												\begin{alignat*}{2}
													& &y &= \frac{a}{b}\bigl(x\bigr)\\
													& \Rightarrow \quad & by &= ax\\
													& \Rightarrow \quad & \frac{b}{a}\bigl(y\bigr) &= a\\
													& \Rightarrow \quad & x &= \frac{b}{a}\bigl(y\bigr)
												\end{alignat*}
												$\therefore (y, x) \in R$, so $R$ is symmetric.
											\end{subproof}
										\item[Transitivity] For every $x, y, z \in \mathbb{Q}^{+}$, if $(x, y) \in R$, and $(y, z) \in R$, is $(x, z) \in R$? Yes.
											\begin{subproof}[Subproof]
												Suppose $(x, y) \in R$ and $(y, z) \in R$.\\
												Then $y = \frac{a}{b}\bigl(x\bigr)$ and $z = \frac{c}{d}\bigl(y\bigr)$, where $a, b, c, d \in \mathbb{Z}^{+}$.
												\begin{align*}
													z &= \frac{c}{d}\bigl(y\bigr)\\
													&= \frac{c}{d}\bigl(\frac{a}{b}\bigr)\bigl(x\bigr)\\
													&= \frac{ab}{cd}\bigl(x\bigr)
												\end{align*}
												$\therefore (x, z) \in R$, so $R$ is transitive.
											\end{subproof}
									\end{description}
									$\therefore R$ is an equivalence relation.
								\end{proof}
								\begin{description}
									\item[Equivalence classes] $
									\begin{aligned}[t]
										[x] &= \{y \mid (x, y) \in R\} \text{ for all $x \in \mathbb{Q}^{+}$}\\
										[x] &= \Bigl\{y \mid y = \frac{a}{b}\bigl(x\bigr)\Bigr\}\\
										[1] &= \Bigl\{y \mid y = \frac{a}{b}\bigl(1\bigr)\Bigr\}\\
										&= \Bigl\{y \mid y = \frac{a}{b}\Bigr\}
									\end{aligned} $\\
									This is the only equivalence class, as every equivalence class is equal to every other equivalence class.
								\end{description}
							\end{answer}
						\pagebreak
						\item Prove that if $R$ is a relation on $\mathbb{Z}^{+}$, then $R$ is symmetric iff $R = R^{-1}$.
							\begin{answer}
								\begin{proof}
									$ $
									\begin{questions}[label=(\roman*)]
										\item If $R$ is symmetric, then $R = R^{-1}$.
											\begin{subproof}
												Assume $R$ is symmetric on $\mathbb{Z}^{+}$.
												\begin{tabbing}
													Suppose \quad \= $(x, y) \in R$.\\
													Then \> $(y, x) \in R$ \quad \=because $R$ is symmetric.\\
													Then \> $(x, y) \in R^{-1}$ \>by the definition of an inverse relation.\\
													So \>$R \subseteq R^{-1}$.
												\end{tabbing}
												\begin{tabbing}
													Suppose \quad \= $(x, y) \in R^{-1}$.\\
													Then \> $(y, x) \in R$ \quad \=by the definition of an inverse relation.\\
													Then \> $(x, y) \in R$ \>because $R$ is symmetric.\\
													So \>$R^{-1} \subseteq R$.
												\end{tabbing}
												As $R \subseteq R^{-1}$ and $R^{-1} \subseteq R$, $R = R^{-1}$
											\end{subproof}
										\item If $R = R^{-1}$, then $R$ is symmetric.
											\begin{subproof}
												Assume $R = R^{-1}$.
												\begin{tabbing}
													Suppose \quad \= $(x, y) \in R$.\\
													Then \>$(y, x) \in R^{-1}$ \quad \= by the definition of an inverse relation.\\
													So \>$(y, x) \in R$ \> because $R = R^{-1}$\\
													So \> $R$ is symmetric.
												\end{tabbing}
											\end{subproof}
									\end{questions}
									If $R$ is symmetric, then $R = R^{-1}$, and if $R = R^{-1}$, then $R$ is symmetric.\\
									$\therefore R$ is symmetric iff $R = R^{-1}$.
								\end{proof}
							\end{answer}
					\end{questions}
				\end{exercise}
				\pagebreak
				\begin{theorem}{Theorem \thechapter.1}
					\begin{enumerate}[label=(\roman*)]
						\item If $R$ is an equivalence relation to $A$, then $x \in [x]$ for each $x \in A$.\\
							In other words, every member of $A$ belongs to an equivalence class with respect to $R$.
						\item If $x\, R \, y$, then $[x] = [y]$. In other words, if two elements are equivalent with respect to $R$, they belong to the same equivalence class.
						\item If $[x] = [y]$, then $x \, R \, y$.
						\item Either $[x] = [y]$ or $[x] \cap [y] = \emptyset$
					\end{enumerate}
				\end{theorem}
			\subsection{Partitions}
				\begin{definition}{Partition}
					For a non-empty set $A$, a \concept{partition} of $A$ is a set $S = \{S_{1}, S_{2}, S_{3}\}$. The members of $S$ are subsets of $A$ (called \emph{parts} of $A$) such that:
					\begin{enumerate}
						\item For all $i$, $S_{i} \neq \emptyset$. That is, every part of the partition is not empty.
						\item For all $i$ and $j$, if $S_{i} \neq S_{j}$, then $S_{i} \cap S_{j} = \emptyset$. That is, different parts of the partition don't have common elements.
						\item $S_{1} \cup S_{2} \cup S_{3} \cup \ldots = A$. That is, every element of $A$ appears in one (and only one) part of the partition.
					\end{enumerate}
				\end{definition}
				\begin{example}
					Let $A$ = \{5, 6, 7\}. Then A can be split into two subsets, $\{5\}$ and $\{6, 7\}$. Then $\bigl\{\{5\}, \{6, 7\}\bigr\}$ is a partition of $A$, as:
					\begin{enumerate}
						\item Neither of the subsets is empty.
						\item There are no common elements between the subsets.
						\item The union of the subsets results in $A$.
					\end{enumerate}
				\end{example}
				\subsubsection{Going Backwards From a Partition}
					If one knows the original set that was partitioned, and the partitions, one can generate the original relation, using \concept{Theorem \thechapter.1}
					\begin{example}
						Given an original set $A = \{a, b, c\}$ and a partition given by $\bigl\{\{a\} \{b, c\}\bigr\}$.
						\begin{indentparagraph}
							The subset $\{a\}$ tells us that $[a] = \{a\}$, i.e. $(a, a) \in R$.\\
							The subset $\{b, c\}$ tells us that $[b] = \{b, c\} = [c]$, which means that $b$ is related to $b$ and $c$, and $c$ is related to $b$ and $c$, so the pairs $(b, b), (b, c), (c, b)$ and $(c, c)$ are in $R$.
						\end{indentparagraph}
						$R = \bigl\{(a, a), (b, b), (b, c), (c, b), (c, c)\bigr\}$
					\end{example}
					\pagebreak
					\begin{exercise}{Self Assessment Exercise \thechapter.12}
						\begin{questions}
							\item{Determine whether $P$ is a partition of $X$ in each of the following cases. If it is, describe the corresponding equivalence relation.}
							\begin{questions}
								\item $X = \{1, 2, 3\}$ and $P = \bigl\{\emptyset, \{1\}, \{2, 3\}\bigr\}$.\\
									{\answer $P$ is \emph{not} a partition of $X$, as $\emptyset$ is one of the elements of the set.}
								\item $X = \{1, 2, 3\}$ and $P = \bigl\{\{1\}, \{2\}, \{1, 3\}\bigr\}$.\\
									{\answer $P$ is \emph{not} a partition of $X$, as $1$ appears in two different elements.}
								\item $X = \{1, 2, 3\}$ and $P = \bigl\{\{1, 3\}, \{2\}\bigr\}$.\\
									\begin{answer}
										$P$ \emph{is} a partition of $X$.\\
										The part $\{2\}$ means that $(2, 2) \in R$.\\
										The part $\{1, 3\}$ means that $(1, 1)$, $(3, 3)$, $(1, 3)$ and $(3, 3)$ are elements of $R$.\\
										$R = \bigl\{(1, 1), (1, 3), (2, 2), (3, 1), (3, 3)\bigr\}$
									\end{answer}
								\item $X = \{1, 2, 3\}$ and $P = \bigl\{\{1\}, \{2\}\bigr\}$\\
									{\answer $P$ is \emph{not} a partition of $X$, as not all the members of $X$ are included.}
								\item $X = \mathbb{Z}$ and $P = \bigl\{\{0\}, \mathbb{Z}^{+}, \mathrm{Neg}\bigr\}$ where $\mathrm{Neg} = \{x \mid x \in \mathbb{Z}$ and $ x < 0\}$.\\
									\begin{answer}
										$P$ \emph{is} a partition of $X$.\\
										The equivalence relation is:
										\begin{align*}
											R = \bigl\{(x, y) \mid (x = 0 \text{ and } y = 0) \text{ or } (x \in \mathbb{Z}^{+} \text{ and } y \in \mathbb{Z}^{+}) \text{ or } (x \in \mathrm{Neg} \text{ and } y \in \mathrm{Neg})\bigr\}
										\end{align*}
									\end{answer}
								\item $X = \mathbb{Z}$ and $P = \bigl\{[0], [1], [2], [3], [4]\bigr\}$, where $[n] = \{x \mid x - n$ is divisible by $5$ with zero remainder$\}$ and $n \in \{0, 1, 2, 3, 4\}$.\\
									\begin{answer}
										$P$ \emph{is} a partition of $X$.\\
										The equivalence relation is:
										\begin{align*}
											R = \bigl\{(x, y) \mid x - y = 5k \text{ for some } k \in \mathbb{Z}\bigr\}
										\end{align*}
									\end{answer}
							\end{questions}
						\end{questions}
					\end{exercise}
		\pagebreak
		\section{Functions}
			\subsection{Functional Relation}
				\begin{definition}{Functional Relation}
					If $R$ is a relation from $X$ to $Y$, then $R$ is \concept{functional} iff any element $x$ in $X$ only appears once as a first coordinate in an ordered pair of $R$.
				\end{definition}
				\begin{example}
					Let $S$ be a relation from $\{1, 2, 3\}$ to $\{a, b, c\}$, where $S = \bigl\{(1, a), (2, c)\bigr\}$. $S$ is a functional relation as $1$ and $2$ only appear as first coordinates in distinct pairs.
				\end{example}
			\subsection{Function}
				\begin{definition}{Function}
					Suppose $R \subseteq A \times B$ is a binary relation from a set $A$ to a set $B$. $R$ is a \concept{function} from $A$ to $B$ if $R$ is functional, and the domain of $R$ is exactly the set $A$, i.e. $\mathrm{dom}(R) = A$.\\
					This is then written $R: A \rightarrow B$.
				\end{definition}
				\begin{example}
					Using the same relation as above:
						\begin{indentparagraph}
							$S$ is a relation from $\{1, 2, 3\}$ to $\{a, b, c\}$, where $S = \bigl\{(1, a), (2, c)\bigr\}$
						\end{indentparagraph}
						$S$ is functional, but not a function, as $\mathrm{dom}(S) \neq \{1, 2, 3\}$.
				\end{example}
				\begin{example}
					Prove that $f$ defined by $(x, y) \in f$ iff $y = 5x^{2} + 3$ is a function on $\mathbb{R}$.\\
					To prove this, determine whether $f$ is functional, and whether $\mathrm{dom}(f) = \mathbb{R}$.
					\begin{proof}
						$ $
						\begin{enumerate}[label=(\roman*)]
							\item $f$ is functional.
								\begin{subproof}
									Suppose $(x, y) \in f$ and $(x, z) \in f$. Is it the case that $y = z$?\\
									As $(x, y) \in f$, $y = 5x^{2} + 3$. As $(x, z) \in f$, $z = 5x^{2} + 3$.\\
									Therefore, $y = 5x^{2} + 3 = z$.\\
									So $f$ is functional. 
								\end{subproof}
							\item $\mathrm{dom}(f) = \mathbb{R}$
								\begin{subproof} $
									\begin{aligned}[t]
										\mathrm{dom}(f) &= \{x \mid \text{ for some } y \in \mathbb{R}, (x, y) \in f\}\\
										&= \{x \mid \text{ for some} y \in \mathbb{R}, y = 5x^{2} + 3\}\\
										&= \{x \mid 5x^{2} + 3 \text{ is a real number}\}\\
										&= \mathbb{R}
									\end{aligned} $\\
									Therefore the domain is equal to the input set.
								\end{subproof}
						\end{enumerate}
						As $f$ is functional, and the domain of $f$ is the same as the input set, $f$ is a function.
					\end{proof}
				\end{example}
				\begin{sidenote}{Not all functional relations are functions!}
					Every function is a functional relation, but a relation can be functional without being a function. This just means that the domain of the relation is not the same as the input set.\\
					If anything from the original set can be given to the relation to produce an output, it is a function.
				\end{sidenote}
				\begin{exercise}{Self Assessment Exercise \thechapter.14}
					\begin{questions}
						\item Give $5$ functions from $A = \{1, 2, 3, 4\}$ to $B = \{a, b, c\}$.
							\begin{answer}
								\begin{align*}
									f_{1} &= \bigl\{(1, a), (2, a), (3, a), (4, a)\bigr\}\\
									f_{2} &= \bigl\{(1, b), (2, b), (3, b), (4, b)\bigr\}\\
									f_{3} &= \bigl\{(1, c), (2, c), (3, c), (4, c)\bigr\}\\
									f_{4} &= \bigl\{(1, a), (2, b), (3, c), (4, b)\bigr\}\\
									f_{5} &= \bigl\{(1, b), (2, a), (3, b), (4, a)\bigr\}
								\end{align*}
							\end{answer}
						\item Give all the functions from $A = \{a, b\}$ to $B = \{5, 6, 7\}$.
							\begin{answer}
								\begin{alignat*}{3}
									f_{1} &= \bigl\{(a, 5), (b, 5)\bigr\} \qquad & f_{4} &= \bigl\{(a, 6), (b, 5)\bigr\} \qquad & f_{7} &= \bigl\{(a, 7), (b, 5)\bigr\}\\
									f_{2} &= \bigl\{(a, 5), (b, 6)\bigr\} \qquad & f_{5} &= \bigl\{(a, 6), (b, 6)\bigr\} \qquad & f_{8} &= \bigl\{(a, 7), (b, 6)\bigr\}\\
									f_{3} &= \bigl\{(a, 5), (b, 7)\bigr\} \qquad & f_{6} &= \bigl\{(a, 6), (b, 7)\bigr\} \qquad & f_{9} &= \bigl\{(a, 7), (b, 7)\bigr\}
								\end{alignat*}
							\end{answer}
						\item Give $3$ functions from $A \times A$ to $B$ if $A = \{a, b\}$ and $B = \{5, 6, 7\}$.
							\begin{answer}
								\begin{align*}
									A \times A &= \bigl\{(a, a), (a, b), (b, a), (b, b)\bigr\}\\
									f_{1} &= \Bigl\{\bigl((a, a), 5\bigr), \bigl((a, b), 5\bigr), \bigl((b, a), 5\bigr), \bigl((b, b), 5\bigr)\Bigr\}\\
									f_{2} &= \Bigl\{\bigl((a, a), 6\bigr), \bigl((a, b), 6\bigr), \bigl((b, a), 6\bigr), \bigl((b, b), 6\bigr)\Bigr\}\\
									f_{3} &= \Bigl\{\bigl((a, a), 5\bigr), \bigl((a, b), 6\bigr), \bigl((b, a), 7\bigr), \bigl((b, b), 6\bigr)\Bigr\}
								\end{align*}
							\end{answer}
						\item Let $R$ be a relation on $A = \bigl\{1, 2, 3, \{1\}, \{2\}\bigr\}$ defined by \\$R = \Bigl\{\bigl(1, \{1\}\bigr), \bigl(1, 3\bigr), \bigl(2, \{1\}\bigr), \bigl(2, \{2\}\bigr), \bigl(\{1\}, 3\bigr), \bigl(\{2\}, \{1\}\bigr)\Bigr\}$.
							\begin{questions}
								\item Is $R$ a function from $A$ to $A$?\\
									{\answer No. There are two elements with the same first coordinate: $\bigl(1, \{1\}\bigr)$ and $(1, 3)$, so $R$ is not a functional relation, so $R$ is not a function.}
								\item Is $\mathrm{ran}(R)$ equal to the codomain of $R$?\\
									{\answer No. $1 \in$ codomain, but $1 \notin \mathrm{ran}(R)$.}
							\end{questions}
						\pagebreak
						\item Consider the set $\mathcal{P}(A) = \bigl\{\emptyset, \{a\}, \{b\}, \{c\}, \{a, b\}, \{a, c\}, \{b, c\}, \{a, b, c\}\bigr\}$. Show that the relations $f$, $g$ and $h$ described below are functional and have as domains $\mathcal{P}(A)$, $\mathcal{P}(A) \times \mathcal{P}(A)$, and $\mathcal{P}(A) \times \mathcal{P}(A)$ respectively.
							\begin{questions}
								\item Let $f = \bigl\{(x, y) \mid x, y \in \mathcal{P}(A)$ and $y = x' \bigr\}$.
									\begin{answer}
										\begin{description}
											\item[Functional] $f$ is functional.
												\begin{proof}
													Suppose $(x, y) \in f$ and $(x, z) \in f$. ($f$ is functional iff $y = z$.)\\
													Then $y = x'$ and $z = x'$.\\
													So $y = x' = z$.\\
													So $f$ is functional.
												\end{proof}
											\item[Domain] The domain of $f$ is equal to the input set $\mathcal{P}(A)$.
												\begin{proof} $
													\begin{aligned}[t]
														\mathrm{dom}(f) &= \{x \mid \text{for some } y \in \mathcal{P}(A), (x, y) \in f\}\\
														&= \{x \mid \text{for some } y \in \mathcal{P}(A), y = x'\}\\
														&= \{x \mid x' \in \mathcal{P}(A)\}\\
														&= \mathcal{P}(A)
													\end{aligned} $\\
													Therefore $\mathrm{dom}(f)$ is equal to the input set.
												\end{proof}
										\end{description}
									\end{answer}
								\item Let $g = \Bigl\{\bigl((u, v), y\bigr) \mid (u, v) \in \mathcal{P}(A) \times \mathcal{P}(A)$ and $y = u \cup v\Bigr\}$.
									\begin{answer}
										\begin{description}
											\item[Functional] $g$ is functional.
												\begin{proof}
													Suppose $\bigl((u, v), y\bigr) \in g$ and $\bigl((u, v), z\bigr) \in g$. ($g$ is functional iff $y = z$.)\\
													Then $y = u \cup v$ and $z = u \cup v$.\\
													So $y = u \cup v = z$.\\
													So $g$ is functional.
												\end{proof}
											\item[Domain] The domain of $g$ is equal to the input set $\mathcal{P}(A) \times \mathcal{P}(A)$.
												\begin{proof} $
													\begin{aligned}[t]
														\mathrm{dom}(g) &= \Bigl\{(u, v) \mid \text{for some } y \in \mathcal{P}(A), \bigl((u, v), y\bigr) \in g\Bigr\}\\
														&= \Bigl\{(u, v) \mid \text{for some } y \in \mathcal{P}(A), y = u \cup v \in g\Bigr\}\\
														&= \Bigl\{(u, v) \mid u \cup v \in \mathcal{P}(A)\Bigr\}\\
														&= \Bigl\{u \in \mathcal{P}(A) \text{ and } v \in \mathcal{P}(A)\Bigr\}\\
														&= \mathcal{P}(A) \times \mathcal{P}(A)
													\end{aligned} $\\
													Therefore $\mathrm{dom}(g)$ is equal to the input set.
												\end{proof}
										\end{description}
									\end{answer}
								\pagebreak
								\item Let $h = \Bigl\{\bigl((u, v), y\bigr) \mid (u, v) \in \mathcal{P}(A) \times \mathcal{P}(A)$ and $y = u \cap v\Bigr\}$.
									\begin{answer}
										\begin{description}
											\item[Functional] $h$ is functional.
												\begin{proof}
													Suppose $\bigl((u, v), y\bigr) \in h$ and $\bigl((u, v), z\bigr) \in h$. ($h$ is functional iff $y = z$.)\\
													Then $y = u \cap v$ and $z = u \cap v$.\\
													So $y = u \cap v = z$.\\
													So $h$ is functional.
												\end{proof}
											\item[Domain] The domain of $h$ is equal to the input set $\mathcal{P}(A) \times \mathcal{P}(A)$.
												\begin{proof} $
													\begin{aligned}[t]
														\mathrm{dom}(h) &= \Bigl\{(u, v) \mid \text{for some } y \in \mathcal{P}(A), \bigl((u, v), y\bigr) \in h\Bigr\}\\
														&= \Bigl\{(u, v) \mid \text{for some } y \in \mathcal{P}(A), y = u \cap v \in h\Bigr\}\\
														&= \Bigl\{(u, v) \mid u \cap v \in \mathcal{P}(A)\Bigr\}\\
														&= \Bigl\{u \in \mathcal{P}(A) \text{ and } v \in \mathcal{P}(A)\Bigr\}\\
														&= \mathcal{P}(A) \times \mathcal{P}(A)
													\end{aligned} $\\
													Therefore $\mathrm{dom}(h)$ is equal to the input set.
												\end{proof}
										\end{description}
									\end{answer}
							\end{questions}
						\item For each of the following relations from $X$ to $Y$, determine whether the relation may be regarded as a function from $X$ to $Y$.
							\begin{questions}
								\item $X = Y = \mathbb{Z}$ and $R = \bigl\{(x, y) \mid y = x\bigr\}$.\\
									\begin{answer}
										$R$ is a function.
										\begin{proof}
											$ $
											\begin{description}
												\item[Functional] $R$ is functional.
													\begin{subproof}[Subproof]
														Suppose $(x, y) \in R$ and $(x, z) \in R$.\\
														Then $y = x$ and $z = x$.\\
														So $y = x = z$.\\
														So $y = z$.\\
														$\therefore R$ is functional.
													\end{subproof}
												\item[Domain] The domain of $R$ is equal to the input set: $\mathbb{Z}$.
													\begin{subproof}[Subproof] $
														\begin{aligned}[t]
															\mathrm{dom}(R) &= \bigl\{x \mid \text{for some } y \in \mathbb{Z}, (x, y) \in R\bigr\}\\
															&= \{x \mid \text{for some } y \in \mathbb{Z}, y = x\}\\
															&= \{x \mid x \in \mathbb{Z}\}\\
															&= \mathbb{Z}
														\end{aligned}$\\
														Therefore $\mathrm{dom}(R)$ is equal to the input set.
													\end{subproof}
											\end{description}
											As $R$ is functional, and the domain of $R$ is equal to the input set, $R$ is a function.
										\end{proof}
									\end{answer}
								\pagebreak
								\item $X = Y = \mathbb{Z}$ and $R = \bigl\{(x, y) \mid y = x + 1\bigr\}$.\\
									\begin{answer}
										$R$ is a function.
										\begin{proof}
											$ $
											\begin{description}
												\item[Functional] $R$ is functional.
													\begin{subproof}[Subproof]
														Suppose $(x, y) \in R$ and $(x, z) \in R$.\\
														Then $y = x + 1$ and $z = x + 1$.\\
														So $y = x + 1 = z$.\\
														So $y = z$.\\
														$\therefore R$ is functional.
													\end{subproof}
												\item[Domain] The domain of $R$ is equal to the input set: $\mathbb{Z}$.
													\begin{subproof}[Subproof] $
														\begin{aligned}[t]
															\mathrm{dom}(R) &= \bigl\{x \mid \text{for some } y \in \mathbb{Z}, (x, y) \in R\bigr\}\\
															&= \{x \mid \text{for some } y \in \mathbb{Z}, y = x + 1\}\\
															&= \{x \mid x + 1 \in \mathbb{Z}\}\\
															&= \mathbb{Z}
														\end{aligned} $\\
														Therefore $\mathrm{dom}(R)$ is equal to the input set.
													\end{subproof}
											\end{description}
											As $R$ is functional, and the domain of $R$ is equal to the input set, $R$ is a function.
										\end{proof}
									\end{answer}
								\item $X = Y = \mathbb{Z}$ and $R = \bigl\{(x, y) \mid y = 3 - x\bigr\}$.\\
									\begin{answer}
										$R$ is a function.
										\begin{proof}
											$ $
											\begin{description}
												\item[Functional] $R$ is functional.
													\begin{subproof}[Subproof]
														Suppose $(x, y) \in R$ and $(x, z) \in R$.\\
														Then $y = 3 - x$ and $z = 3 - x$.\\
														So $y = 3 - x = z$.\\
														So $y = z$.\\
														$\therefore R$ is functional.
													\end{subproof}
												\item[Domain] The domain of $R$ is equal to the input set: $\mathbb{Z}$.
													\begin{subproof}[Subproof] $
														\begin{aligned}[t]
															\mathrm{dom}(R) &= \bigl\{x \mid \text{for some } y \in \mathbb{Z}, (x, y) \in R\bigr\}\\
															&= \{x \mid \text{for some } y \in \mathbb{Z}, y = 3 - x\}\\
															&= \{x \mid 3 - x \in \mathbb{Z}\}\\
															&= \mathbb{Z}
														\end{aligned} $\\
														Therefore $\mathrm{dom}(R)$ is equal to the input set.
													\end{subproof}
											\end{description}
											As $R$ is functional, and the domain of $R$ is equal to the input set, $R$ is a function.
										\end{proof}
									\end{answer}
								\pagebreak
								\item $X = Y = \mathbb{Z}$ and $R = \bigl\{(x, y) \mid y = \sqrt{x}\bigr\}$. (That is, the positive square root of $x$.)\\
									\begin{answer}
										$R$ is \emph{not} a function.
										\begin{proof}
											$ $
											\begin{description}
												\item[Functional] $R$ is functional.
													\begin{subproof}[Subproof]
														Suppose $(x, y) \in R$ and $(x, z) \in R$.\\
														Then $y = \sqrt{x}$ and $z = \sqrt{x}$.\\
														So $y = \sqrt{x} = z$\\
														So $y = z$.\\
														So $R$ is functional.
													\end{subproof}
												\item[Domain] The domain of $R$ is \emph{not} equal to the input set.
													\begin{subproof}[Counterexample]
														$2 \in X$, but there is no integer $y$ (i.e. no $y \in Y$) where $y = \sqrt{2}$, because $\sqrt{2}$ is irrational.\\
														$-1 \in X$, but there is no integer $y$ where $y = \sqrt{-1}$.\\
														$\therefore \mathrm{dom}(R) \neq X$ 
													\end{subproof}
											\end{description}
											As the domain of $R$ is not equal to the input set, $R$ is not a function.
										\end{proof}
									\end{answer}
								\item $X = Y = \mathbb{Z}$ and $R = \bigl\{(x, y) \mid y^{2} = x\bigr\}$.\\
									\begin{answer}
										$R$ is not a function.
										\begin{proof}
											$ $
											\begin{description}
												\item[Functional] $R$ is \emph{not} functional.
													\begin{subproof}[Subproof]
														Suppose $(x, y) \in R$ and $(x, z) \in R$.\\
														Then $y^{2} = x$ and $z^{2} = x$.\\
														So $y = \pm \sqrt{x}$ and $z = \pm \sqrt{x}$.\\
														As $y$ can equal $\sqrt{x}$ and $z$ can equal $- \sqrt{x}$, $y \neq z$\\
														So $R$ is not functional.
													\end{subproof}
												\item[Domain] The domain of $R$ is \emph{not} equal to the input set.
													\begin{subproof}[Counterexample]
														$2 \in X$, but there is no integer $y$ (i.e. no $y$ in $Y$) where $y^{2} = 2$, because $\sqrt{2}$ is irrational.\\
														$-1 \in X$, but there is no integer $y$ where $y^{2} = -1$.\\
														$\therefore \mathrm{dom}(R) \neq X$ 
													\end{subproof}
											\end{description}
											As $R$ is not functional, and the domain of $R$ is not equal to the input set, $R$ is not a function.
										\end{proof}
									\end{answer}
								\pagebreak
								\item $X = Y = \mathbb{R}$ and $S = \bigl\{(x, y) \mid x^{2} + y^{2} = 1\bigr\}$.\\
									\begin{answer}
										$R$ is not a function.
										\begin{proof}
											$ $
											\begin{description}
												\item[Functional] $R$ is \emph{not} functional.
													\begin{subproof}[Subproof]
														Suppose $(x, y) \in S$ and $(x, z) \in S$.\\
														Then $x^{2} + y^{2} = 1$ and $x^{2} + z^{2} = 1$.\\
														So $y^{2} = 1 - x^{2}$ and $z^{2} = 1 - x^{2}$.\\
														So $y = \pm \sqrt{1 - x^{2}}$ and $z = \pm \sqrt{1 - x^{2}}$.\\
														As $y$ can be $\sqrt{1 - x^{2}}$ and $z$ can be $- \sqrt{1 - x^{2}}$, $y \neq z$.\\
														So $R$ is not functional.
													\end{subproof}
												\item[Domain] The domain of $S$ is \emph{not} equal to the input set.
													\begin{subproof}[Counterexample]
														$2 \in \mathbb{R}$, but there is no real number $y$ where $2^{2} + y^{2} = 1$.\\
														$\therefore \mathrm{dom}(R) \neq X$
													\end{subproof}
											\end{description}
											As $R$ is not functional, and the domain of $R$ is not equal to the input set, $R$ is not a function.
										\end{proof}
									\end{answer}
							\end{questions}
						\item Is the relation $R$ on $\mathbb{Z}^{+}$, which consists of all pairs $(x, y)$ such that $y = x - 1$, a function from $\mathbb{Z}^{+}$ to $\mathbb{Z}^{+}$?\\
							\begin{answer}
								No.
								\begin{proof}
									$ $
									\begin{description}
										\item[Functional] $R$ is functional.
											\begin{subproof}[Subproof]
												Suppose $(x, y) \in R$ and $(x, z) \in R$.\\
												Then $y = x - 1$ and $z = x - 1$.\\
												So $y = x - 1 = z$.\\
												So $y = z$.\\
												So $R$ is functional.
											\end{subproof}
										\item[Domain] The domain of $R$ is \emph{not} equal to the input set.
											\begin{subproof}[Subproof] $
												\begin{aligned}[t]
													\mathrm{dom}(R) &= \bigl\{x \mid \text{for some } y \in \mathbb{Z}^{+}, (x, y) \in R\bigr\}\\
													&= \bigl\{x \mid \text{for some } y \in \mathbb{Z}^{+}, y = x - 1\bigr\}\\
													&= \bigl\{x \mid x - 1 \in \mathbb{Z}^{+}\bigr\}\\
													&= \bigl\{x > 1 \mid x \in \mathbb{Z}^{+}\bigr\}\\
													& \neq \mathbb{Z}^{+}
												\end{aligned} $\\
												For example, $y = 1 - 1 = 0$ cannot be an element of $R$ if the domain is $\mathbb{Z}^{+}$
											\end{subproof}
									\end{description}
									As the domain of $R$ is not equal to the input set, $R$ is not a function.
								\end{proof}
							\end{answer}
						\pagebreak
						\item Let $A = \{a, b, c\}$. Consider all the equivalence relations on $A$. How many relations are also functions from $A$ to $A$?
							\begin{answer}
								\begin{description}
									\item[Equivalence Relations]
										\begin{align*}
											R_{1} &= \bigl\{(a, a), (b, b), (c, c)\bigr\}\\
											R_{2} &= \bigl\{(a, a), (b, b), (c, c), (a, b), (b, a)\bigr\}\\
											R_{3} &= \bigl\{(a, a), (b, b), (c, c), (a, c), (c, a)\bigr\}\\
											R_{4} &= \bigl\{(a, a), (b, b), (c, c), (b, c), (c, b)\bigr\}\\
											R_{5} &= \bigl\{(a, a), (b, b), (c, c), (a, b), (b, a), (a, c), (c, a), (b, c), (c, b)\bigr\}
										\end{align*}
									\item[Functions] Only $R_{1}$ is a function.
									\item[Abstract Reasoning] If $R$ is an equivalence relation, then $R$ is reflexive. So $\mathrm{dom}(R) = A$.\\
										But if $R$ is an equivalence relation where a first coordinate appears more than once, $R$ is not a function.\\
										So the first coordinates need to only appear once.\\
										In an equivalence relation, that means that the only function is the identity relation.\\
										So the only function is $\bigl\{(a, a), (b, b), (c, c)\bigr\}$
								\end{description}
							\end{answer}
						\item Let $A = \{a, b, c\}$.
							\begin{questions}
								\item How many weak partial orders on $A$ are also functions from $A$ to $A$?\\
									\begin{answer}
										If $S$ is a weak partial order on $A$, then $S$ is reflexive. So $\mathrm{dom}(S) = A$.\\
										That means that every element of $A$ appears as the first coordinate in at least one pair.\\
										For $S$ to be functional, each element of $A$ must only appear as the first coordinate in one pair.\\
										The only case for this is the identity relation.\\
										So the only weak partial order on $A$ that is a function is $\bigl\{(a, a), (b, b), (c, c)\bigr\}$.
									\end{answer}
								\item How many strict partial orders on $A$ are also functions from $A$ to $A$?\\
									\begin{answer}
										For a strict partial order $T$ to be a function on $A$, the domain of $T$ needs to be $A$, and $T$ needs to be functional.\\
										Each element of $A$ should appear as the first coordinate in exactly one pair. For the relation to be a strict partial order, it needs to be antisymmetric, irreflexive and transitive.\\
										There is no combination of pairs that satisfies all three requirements for a strict partial order that is also a function.
									\end{answer}
							\end{questions}
					\end{questions}
				\end{exercise}
	\rulechapterend
\end{document}
